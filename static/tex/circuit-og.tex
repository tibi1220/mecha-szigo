\documentclass{standalone}

\usepackage{amsmath}
\usepackage{unicode-math}
\usepackage{tikz}
\usepackage{circuitikz}

\usetikzlibrary{
  patterns, patterns.meta,
  arrows, arrows.meta,
  bending,
  positioning,
  decorations.markings, decorations.pathreplacing,
  shapes.misc, shapes.geometric
}
\tikzset{
  dot/.style = {circle, fill=red!80!gray, minimum
      size=#1, draw=black,
      inner sep=0pt, outer sep=0pt},
  dot/.default = 5pt, % size of the circle diameter
  gdot/.style = {dot=#1, fill=white},
  bdot/.style = {dot=#1, fill=black},
  gdot/.default = 5pt, % size of the circle diameter
  bdot/.default = 5pt, % size of the circle diameter
  extended line/.style={shorten >=-#1,shorten <=-#1},
  extended line/.default=3mm,
  rod/.style={double distance=#1, line cap=round, very thick,
      double=gray!20},
  rod/.default=4mm,
  var/.style={rectangle, fill=white},
  guide/.style={gray, line width=.6pt},
  dim/.style={latex-latex, teal, thick},
  edge/.style={
      postaction=decorate, decoration={
	  markings,
	  mark=at position #1 with {\arrow{Latex[round, length=3mm, bend]}}
	}
    },
  edge/.default=.55,
  generator/.style={midway, gdot=#1, fill opacity=0},
  generator/.default=10mm,
}

\begin{document}
\begin{circuitikz}[european, scale=.85]
  \node[
    rectangle, draw=black, thick,
    minimum height = 3cm, minimum width=2cm,
    text width = 2cm, align = center
  ] (fp1) at (0,0) {$M_\text{vill} = k_\text{m} i$ \\ $u_\text{ind} = k_\text{e} \omega_\text{r}$};

  \node[
    rectangle, draw=black, thick,
    minimum height = 3cm, minimum width=2cm,
    text width = 2.2cm, align = center
  ] (fp2) at (7,0) {$\omega_\text{r} = N \omega_\text{t}$ \\ $M_\text{t} = N M_\text{r}$};

  \draw ($(fp1.west)+(0,1)$)
  to[R, l_=$sL$] ++(-2,0)
  to[R, l_=$R$] ++(-2,0)
  to[vsource, l_=$u_\text{be}$] ++(0,-2) coordinate (G1)
  to[] (G1-|fp1.west)
  (G1) to[] ++(0,-.25) node[rground] {}
  ;

  \draw ($(fp1.east)+(0,1)$)
  to ++(1,0) coordinate (A)
  to ++(1.6,0) coordinate (B)
  to (B-|fp2.west)
  ;
  \draw ($(fp1.east)-(0,1)$) coordinate (C) to (C-|fp2.west);
  \draw
  (A) to[R, l=$\dfrac{1}{B_\text{r}}$] ++(0,-2) coordinate (G2)
  (B) to[R, l=$\dfrac{1}{sJ_\text{r}}$] ++(0,-2)
  (G2) to[] ++(0,-.25) node[rground] {}
  ;

  \draw ($(fp2.east)+(0,1)$)
  to ++(1,0) coordinate (D)
  to ++(1.75,0) coordinate (E)
  to[isource, l=$-M_\text{t}$] ++(0,-2) coordinate (G3)
  to[] (G3-|fp2.east)
  (D) to[R, l=$\dfrac{1}{sJ_\text{t}}$] ++(0,-2)
  (G3) to[] ++(0,-.25) node[rground] {}
  ;
\end{circuitikz}
\end{document}
