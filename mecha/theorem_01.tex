\documentclass[../main.tex]{subfiles}

\begin{document}

\section{Állapot, rendszer}

\begin{fulltheorem}
  Ismertesse a következő fogalmakat!
  (Adja meg a definícióját és rövid értelmezését.)
  \begin{itemize}
    \item Állapot, állapotváltozó, állapotjelző
    \item Állapot és kimeneti egyenlet
    \item A rendszer dimenziója
  \end{itemize}
\end{fulltheorem}

Az \kix{állapot} rendszerelméleti megközelítésénél a dinamikus rendszerek
definíciójából indulunk ki, miszerint azok memória jelleggel rendelkeznek.
Az állapot tehát a múlt összesített hatása. A rendszer állapotának a következő
két tulajdonságokkal kell rendelkeznie:
\begin{itemize}
  \item Bármely $t$ időpillanatban a \kix{kimenőjel} az adott pillanatbeli
        \kix{állapot} és \kix{bemenőjel} együttes ismeretében egyértelműen
        meghatározható legyen. (\kix{állapotegyenlet})
  \item Az állapot egy adott $t$ időpillanatban egyértelműen meghatározható
        legyen a bemenőjel a $\tau \leq t$ időtartománybeli értékének
        ismeretében. (\kix{kimeneti egyenlet})
\end{itemize}

Az \kix{állapotváltozó}[k] az \kix{állapot} egyértelmű leírására szolgálnak.
Egy rendszer állapotának egyértelmű leírásához minimálisan szükséges
állapotváltozók számát a \kw{rendszer dimenziójának}
\index{renszer dimenziója} szokás nevezni. Az állapotváltozók szokásos jelölése:
$\rvec x(t) \in \mathbb R^n$.

\end{document}
