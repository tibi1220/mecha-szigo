\documentclass[../main.tex]{subfiles}

\begin{document}

\section{Rendszerek}

\begin{fulltheorem}
  Ismertesse a következő fogalmakat!
  (adja meg a definícióját és rövid értelmezését)
  \begin{itemize}
    \item Valós fizikai rendszer
    \item A jel
    \item A be- és kimenetek
  \end{itemize}
\end{fulltheorem}

\subsection{Valós fizikai rendszerek}

A \kix{valós fizikai rendszer} egy olyan fizikai objektum, amely mérhető külső
kényszer hatására mérhető módon megváltozik.

A \kix{rendszer} a fizikai valóság elkülönített része, amely bemenő mennyiségekből
egy folyamat hatására kimenő mennyiségeket hoz létre úgy, hogy közben kapcsolatban
van áll a környezettel.

\begin{figure}[H]
  \centering
  \begin{tikzpicture}[thick]
    \node[
      ellipse,
      minimum width=5cm, minimum height=3cm,
      draw=black, fill=red!5, dashed,
      decorate, decoration={random steps,segment length=3pt,amplitude=1pt},
    ]
    (k) at (0,0) {};
    \node[
      ellipse,
      align=center,
      draw=cyan!50!black, fill=yellow!10,
      decorate, decoration={random steps,segment length=3pt,amplitude=1pt},
    ]
    (e) at (0,0) {rendszer \\ (folyamat)};

    % noindent
    \foreach \i/\j in {245/260,270/270,295/280}{
        \draw[teal!50!gray, -{Latex[round]}, ultra thick]
        (e.\i) -- ++(\j:1.5);
      }
    \foreach \i in {-15,0,15,165,180,195}{
        \draw[yellow!50!black, {Latex[round]}-, ultra thick]
        (e.\i) -- ++(\i:1.5);
      }
    % indent

    \node[align=center] at (0,-3) {kimenő\\mennyiség};
    \node at (0,1.1) {környezet};
    \node[align=center] at (-4,0) {bemenő\\mennyiség};
  \end{tikzpicture}
  \caption{A rendszer definíciója grafikusan}
  \label{fig:system}
\end{figure}

\subsection{A jel}

A \kix{jel} egy változó \kix{fizikai mennyiség} absztrakt információ tartalma.

\subsection{Be- és kimenetek}

A valós fizikai rendszerre ható és időben változni képes kényszereket nevezzük
\kix{fizikai bemenet}[eknek]. A bementek számunkra fontos, mérhető mennyiségek.
Jele: $u(t)$.

A valós fizikai rendszernek a fizikai kényszerek hatására bekövetkező bármely
változása lehet \kix{fizikai kimenet}, ezek közül azt tekintjük
\kix{fizikai kimenet}[nek], amelyet az adott vizsgálatban közvetlenül vagy
közvetve mérünk. Jele: $y(t)$.

\end{document}
