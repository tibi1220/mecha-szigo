\documentclass[../main.tex]{subfiles}

\begin{document}

\section{Vizsgálójelek}

\label{sec:3-14}

\begin{fulltheorem}
  Milyen vizsgálójeleket ismer? Mi a feltétele annak, hogy egy rendszer
  működését vizsgáló jelek segítségével analizáljunk? Mi e módszer lényege?
  Egy magyarázó ábra segítségével ismertesse, hogy a súlyfüggvény
  (impulzus válasz) ismeretében miként számítható ki egy tetszőleges bemenő
  jelre adott válasz a folytonos időtartományban (konvolúció).
\end{fulltheorem}

A \kix{vizsgálójel}[ek] lehetnek:
\begin{itemize}
  \item $\hc(t)$ -- Egységugrás (Heaviside) függvény
  \item $\dirac(t)$ -- Dirac-delta
\end{itemize}

Ezek definíciója diszkrét időben:
\[
  \dirac[k] = \begin{cases}
    1 \text{, ha } k = 0    \\
    0 \text{, ha } k \neq 0 \\
  \end{cases}
  \qquad
  \hc[k] = \begin{cases}
    1 \text{, ha } k \geq 0 \\
    0 \text{, ha } k < 0    \\
  \end{cases}
\]

Definíció folytonos időben:
\[
  \dirac(t) = \begin{cases}
    + \infty \text{, ha } t = 0 \\
    0 \text{ egyébként}
  \end{cases}
  \qquad
  \hc(t) = \begin{cases}
    1 \text{, ha } t \geq 0 \\
    0 \text{, ha } t < 0    \\
  \end{cases}
\]

A vizsgálójelek Laplace-transzformáltja:
\[
  \Laplace \left\{ \dirac(t) \right\}(s) = 1
  \qquad
  \Laplace \left\{ \hc(t) \right\} (s) = 1 / s
\]

\subsection{A súlyfüggvény}

A \kix{súlyfüggvény} a rendszer \kix{Dirac-delta} függvényre adott válasza.
Az \kix{átviteli függvény} segítségével egyszerűen számolható:
\[
  w(t) = \Laplace^{-1} \left\{ W(s) \right\}(t)
\]

\kix{Egytárolós tag} \kix{időállandós  alak}[jából] számítva:
\[
  w(t)
  = \Laplace^{-1} \left\{ \frac{A}{1 + sT} \right\} (t)
  = \Laplace^{-1} \left\{ \frac{A}{T} \frac{1}{s + 1/T} \right\} (t)
  = \frac{A}{T} e^{-t / T}
\]

\kix{Kéttárolós tag} esetén:
\[
  w(t)
  = \Laplace^{-1} \left\{ \frac{A \omega_n^2}{s^2 + 2 \zeta \omega_n  s + \omega_n^2} \right\} (t)
  = \frac{A \omega_n^2}{\omega_d} e^{-\zeta \omega_n t } \sin \omega_d t
\]

\subsection{Az átmeneti függvény}

Az \kix{átmeneti függvény} a rendszer \kix{egységugrás} függvényre adott válasza.
Az \kix{átviteli függvény} segítségével egyszerűen számolható:
\[
  v(t) = \Laplace^{-1} \left\{ W(s) / s\right\}(t)
\]

\kix{Egytárolós tag} \kix{időállandós  alak}[jából] számítva:
\[
  v(t)
  = \Laplace^{-1} \left\{ \frac{1}{s} \frac{A}{1 + sT} \right\} (t)
  = \Laplace^{-1} \left\{ \frac{A}{s} -  \frac{AT}{1 + sT} \right\} (t)
  = A \left( 1 - e^{-t/T} \right)
\]

\kix{Kéttárolós tag} esetén:
\[
  v(t)
  = \Laplace^{-1} \left\{ \frac{1}{s} \frac{A \omega_n^2}{s^2 + 2 \zeta \omega_n	s + \omega_n^2} \right\} (t)
  = A \left( 1 - e^{-\zeta \omega_n t} \left( \frac{\zeta}{\sqrt{1 - \zeta^2}} \sin \omega_d t + \cos \omega_d t
  \right) \right)
\]\[
  v(t) = A \left( 1 - \frac{e^{-\zeta \omega_n t}}{\sqrt{1 - \zeta^2}} \sin \left( \omega_d t + \varphi \right)
  \right)
  \qquad
  \varphi = \arccos \zeta
\]

\subsection{Kimenet megadása a súlyfüggvény segítségével}

Tetszőleges \kix{bemenet}[re] felírhatjuk a \kix{kimenet}[et] a súlyfüggvény
segítségével a konvolúció segítségével.

\subsubsection*{Operátor tartományban}
\[
  Y(s) = W(s)U(s)
\]

\subsubsection*{Folytonos időben}
\[
  y(t) = (w * u)(t) = \int_{0^-}^\infty w(t - \tau) u(\tau) \, \mathrm{d} \tau
\]

\subsubsection*{Diszkrét időben}
\[
  y[k] = \sum_{i=0}^k w [k - i] u[i]
\]

\end{document}
