\documentclass[../main.tex]{subfiles}

\begin{document}

\section{Hipermátrixok}

\begin{fulltheorem}
	Egy összefüggő irányított egyszerű gráf esetén maximálisan hány lineárisan
	független hurok-, illetve vágategyenleteket leíró részgráf választható ki?
	Egy példa segítségével mutassa be a kiválasztás módját. Lássa be, hogy a
	csomóponti mátrix és a hurok mátrix transzponáltjának szorzata egy zéró
	elemekből álló mátrix.
\end{fulltheorem}

\end{document}
