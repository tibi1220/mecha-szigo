\documentclass[../main.tex]{subfiles}

\begin{document}

\section{ARMA rendszerek}

\begin{fulltheorem}
  Ismertesse a következő fogalmakat!
  (adja meg a definícióját és rövid értelmezését)
  \begin{itemize}
    \item ARMA rendszer
    \item Eltolási operátor
    \item Impulzus átviteli függvény
  \end{itemize}
  Mi és hol van kódolva az ARMA rendszerek paramétereiben?
  Ismertesse a kapcsolatot az impulzus átviteli függvény és az átviteli függvény között.
\end{fulltheorem}

\subsection{ARMA rendszer definíciója}

Az \kix{ARMA rendszer}[ekben] (Autoregresszív Mozgó Átlag / AutoRegressive Moving
Average) az \kw{AR} alrész arra utal, hogy a \kix{kimenőjel} korábbi értékei visszahatnak
a kimenőjel korábbi értékeire (\kix{memóriajelleg}). Ezt \kix{visszacsatolás}[sal]
valósítjuk meg. A \kw{MA} alrész pedig azt jelenti, hogy a \kix{bemenőjel} korábbi
értékei (mozgó átlaga) is hatással lehet a jelenlegi jelre.

Az \kix{ARMA} rendszer $a_i$ együtthatói a kimenőjelek, $b_i$ együtthatói a bemenőjelek
hatását kódolják magukban. Az \kix{ARMA egyenlet} folytonos és diszkrét időben:
\begin{alignat*}{9}
   & \sum_{i=0}^n  a_i y^{(i)}(t) & = & \sum_{i=0}^r b_i u^{(i)}(t)
  \\
   & \sum_{i=0}^n  a_{di} y[k-i]  & = & \sum_{i=0}^r b_{di} u[k-i]
\end{alignat*}

\subsection{Eltolási operátor}

\kix{Z-transzformáció} esetén a \kix{$z$ operátor} az időben való eltolást
valósítja meg. Unilaterális Z-transzformáció definíciója:
\[
  X(z)=\mathcal Z \left\{ x[n] \right\} = \sum_{k=0}^{\infty} x[k]z^{-k}
\]


\subsection{Impulzus átviteli függvény}

Az \kix{impulzus átviteli függvény} formális definíciója hasonlít az
\kix{átviteli függvény}[éhez], a kimenő jel transzformáltját kell elosztanunk
a bemenő jel transzformáltjával. Ez \kix{Z-transzformálás} esetén:
\[
  W(z) = \frac{Y(z)}{U(z)}
\]

\end{document}
