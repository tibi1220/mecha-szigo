\documentclass[../main.tex]{subfiles}

\begin{document}

\section{Fourier sorfejtés III}

\begin{fulltheorem}
  Mutassa be, hogy a Fourier transzformáció miként általánosítható,
  hogy eljussunk a Laplace transzformációhoz. Milyen feltételeknek megfelelő
  időfüggvények esetén alkalmazható a Fourier, illetve a Laplace transzformáció.
  Ismertesse a kezdeti és végérték tételt. Mi az alkalmazásuk feltétele?
\end{fulltheorem}

Nem tanultuk.

\end{document}
