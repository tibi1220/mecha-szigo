\documentclass[../main.tex]{subfiles}

\begin{document}

\section{Átviteli karakterisztikák}

\begin{fulltheorem}
  Ismertesse, hogy mit ábrázol a Bode és Nyquist diagram. Méréssel hogyan
  állítható elő? Rajzolja fel az egy-, két- és háromtárolós, valamint
  integráló és a holtidős tag Bode és Nyquist diagramját.
\end{fulltheorem}

A \kix{Bode diagram} és a \kix{Nyquist diagram} segítségével
frekvencia-átviteli karakterisztikákat ábrázolhatunk.

\subsection{A Bode diagram}

A \kix{Bode diagram}[on] az amplitudót és a szögeltolást ábrázolhatjuk a frekvencia
logaritmusának függvényében. Az ábrázolásmód olyan tekintetben hasznos,
hogy átviteli függvények szorzata a logaritmikus azonosságok miatt
összeadássá redukálódik.

Az ábrázolandó függvényeket megkapjuk az $s = \iu \omega$ helyettesítéssel:
\[
  A_r = \left| W(\iu \omega) \right|
  \quad \rightarrow \quad
  B = 20 \lg A_r
\]\[
  \varphi = \arg W(\iu \omega)
\]

A $W(\iu \omega)$ \kix{frekvencia átviteli függvény} egy olyan komplex függvény,
amely a nem negatív körfrekvenciákhoz egy olyan komplex számot rendel hozzá,
amelynek az abszolút értéke a kimenőjel állandósult állapotbeli amplitúdójának
és bemenőjel amplitúdójának aránya, a szöge megegyezik a kimenőjel állandósult
állapotbeli fáziseltolódásával a bemenőjelhez képest (az adott körfrekvencián).

\begin{figure}[htb]
  \centering
  \includegraphics{../static/tex/build/bode-general.pdf} 
  \caption{Bode diagram általános alakja}
  \label{fig:bode-general}
\end{figure}

\subsection{A Nyquist diagram}

A \kix{Nyquist diagram}[on] a frekvencia átviteli függvény valós és képzetes
részét ábrázoljuk.

\begin{figure}[H]
  \centering
  \includegraphics{../static/tex/build/nyquist-general.pdf} 
  \caption{A Nyquist diagram általános alakja}
  \label{fig:nyquist}
\end{figure}

\subsubsection*{Arányos tag}

\kix{Arányos tag} esetén az átviteli függvény: $W(s) = P \; \rightarrow \; W(\iu \omega) = P$.
\[
  |W(\iu \omega)| = P
  \qquad
  \arg W(\iu \omega) = 0^\circ
\]

\begin{figure}[H]
  \centering
  \includegraphics{../static/tex/build/bode-P.pdf} 
  \caption{Arányos tag Bode diagramja ($P=10 \; \rightarrow \; 20\lg P = 20$)}
  \label{fig:bode-P}
\end{figure}
% \begin{figure}[H]
%   \centering
%   % noindent
%   \NyquistZPK[%
%     plot/{black,samples=1000}
%   ]%
%   { z/{0}, p/{0}, k/10 }
%   {-30}{30}
%   % indent
%   \caption{Arányos tag Nyquist diagramja}
%   \label{fig:nyquist-P}
% \end{figure}


\subsubsection*{Deriváló tag}

\kix{Deriváló tag} esetén az átviteli függvény: $W(s) = s \; \rightarrow \; W(\iu \omega) = \iu \omega$.
\[
  |W(\iu \omega)| = \omega
  \qquad
  \arg W(\iu \omega) = 90^\circ
\]

\begin{figure}[H]
  \centering
  \includegraphics{../static/tex/build/bode-diff.pdf} 
  \caption{Deriváló tag Bode diagramja}
  \label{fig:bode-diff}
\end{figure}

\subsubsection*{Integráló tag}

\kix{Integráló tag} esetén az átviteli függvény: $W(s) = 1/s \; \rightarrow \; W(\iu \omega) = -\iu / \omega$.
\[
  |W(\iu \omega)| = 1 / \omega
  \qquad
  \arg W(\iu \omega) = -90^\circ
\]

\begin{figure}[H]
  \centering
  \includegraphics{../static/tex/build/bode-int.pdf} 
  \caption{Integráló tag Bode diagramja}
  \label{fig:bode-int}
\end{figure}

\subsubsection*{Egytárolós tag}

% \paragraph*{Tiszta zérus}

\kix{Tiszta zérus} esetén az átviteli függvény: $W(s) = 1 + sT \; \rightarrow \; W(\iu \omega) = 1 + \iu \omega T$.
\[
  |W(\iu \omega)| = \sqrt{1 + \omega^2 T^2}
\]

\begin{figure}[H]
  \centering
  \includegraphics{../static/tex/build/bode-z1.pdf} 
  \caption{Tiszta zérus Bode diagramja ($T = 1$)}
  \label{fig:bode-1z}
\end{figure}

% \paragraph*{Tiszta pólus}

\kix{Tiszta pólus} esetén az átviteli függvény: $W(s) = 1/(1 + sT) \; \rightarrow \; W(\iu \omega) =  1/(1 + \iu \omega T)$.
\[
  |W(\iu \omega)| = 1 / \sqrt{1 + \omega^2 T^2}
\]

\begin{figure}[H]
  \centering
  \includegraphics{../static/tex/build/bode-p1.pdf} 
  \caption{Tiszta pólus Bode diagramja ($T = 1$)}
  \label{fig:bode-1p}
\end{figure}

\end{document}
