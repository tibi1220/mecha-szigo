\documentclass[../main.tex]{subfiles}

\begin{document}

\section{LTI rendszerek}

\begin{fulltheorem}
  Adja meg a lineáris idő invariáns rendszerek állapottér egyenletének
  szokásos mátrixos alakját. Laplace transzformáció segítségével vezesse le
  egy lineáris idő invariáns rendszer állapottér egyenletének megoldását az
  időtartományban. Miként vehetők figyelembe a kezdeti értékek?
\end{fulltheorem}

Az \kix{LTI} rendszerekre teljesül a \kix{szuperpozíció elve} (linearitás),
valamint teljesül, hogy késleltetett bemenetre késleltetett választ ad.
\begin{gather*}
  u(t) \; \rightarrow \; y(t)
  \qquad
  u(t + \tau) \; \rightarrow \; y(t + \tau)
  \\[2mm]
  u_1(t) \;\rightarrow\; y_1(t)
  \text{ és }
  u_2(t) \;\rightarrow\; y_2(t)
  \\
  \lambda u_1(t) + \mu u_2(t)
  \;\rightarrow\;
  \lambda y_1(t) + \mu y_2(t)
\end{gather*}

\subsection{Diszkrét időben}

A kimeneti és állapotegyenleteket rekurzió segítségével írhatjuk fel:
\begin{alignat*}{9}
   & \rvec x [k+1] & = & \rmat A_d \rvec x[k] & + & \rmat B_d \rvec u[k] \\
   & \rvec y [k]   & = & \rmat C_d \rvec x[k] & + & \rmat D_d \rvec u[k]
\end{alignat*}

\begin{figure}[htb]
  \centering
  \begin{tikzpicture}[
      thick,
      arr/.style={ultra thick, draw=cyan!50!black},
      box/.style={rectangle, minimum width=1.2cm, minimum  height=.75cm, draw=black, },
      cross/.style={path picture={
              \draw[black]
              (path picture bounding box.south east) -- (path picture bounding box.north west) (path picture bounding
              box.south west)
              -- (path picture bounding box.north east);
            }},
      sum/.style={circle, minimum size=.75cm, draw=black, cross},
    ]
    \node[box] (z) at (0,0) {$z^{-1}$};
    \node[box] (a) at (0,-1.5) {$\rmat A_d$};
    \node[box] (b) at (-4,0) {$\rmat B_d$};
    \node[box] (c) at (2.5,0) {$\rmat C_d$};
    \node[box] (d) at (0,-3) {$\rmat D_d$};

    \node[sum] (s1) at (-2,0) {};
    \node[sum] (s2) at (4.5,0) {};

    \draw[arr,to-] (b) -- ++(-2,0);
    \draw[arr,-to] (b) -- (s1);
    \draw[arr,-to] (s1) -- (z);
    \draw[arr,-to] (z) -- (c);
    \draw[arr,-to] (c) -- (s2);
    \draw[arr,-to] (s2) --  ++(1.5,0);

    \draw[arr,-to] (a) -| (s1);
    \draw[arr,-to] (d) -| (s2);
    \draw[arr,-to] ($(b)-(1.25,0)$) |- (d);
    \draw[arr,-to] ($(c)-(1.25,0)$) |- (a);
  \end{tikzpicture}
  \caption{Állapottérmodell hatásvázlata diszkrét időben}
  \label{fig:dstate}
\end{figure}


\subsection{Folytonos időben}

A diszkrét modellből levezetve, a differenciahányadosaink
differenciálhányadosokká alakulnak.
\begin{alignat*}{9}
   & \dot{\rvec x} (t) & = & \rmat A \rvec x(t) & + & \rmat B \rvec u(t) \\
   & \rvec y (t)       & = & \rmat C \rvec x(t) & + & \rmat D \rvec u(t)
\end{alignat*}

\bgroup
\def\arraystretch{1.2}
\begin{table}[H]
  \begin{center}
    \begin{tabular}{| c | c |}
      \hline
      \begin{tabular}{ c c }
        $\rvec x(t) \in \mathbb R^n$ & \kix{állapotvektor}   \\
        $\rvec u(t) \in \mathbb R^m$ & \kix{bemeneti vektor} \\
        $\rvec y(t) \in \mathbb R^p$ & \kix{kimeneti vektor} \\
      \end{tabular}
       &
      \begin{tabular}{ c c }
        $\rmat A \in \mathcal{M}_{n \times n}$ & \kix{rendszermátrix}        \\
        $\rmat B \in \mathcal{M}_{n \times m}$ & \kix{bemeneti mátrix}       \\
        $\rmat C \in \mathcal{M}_{p \times n}$ & \kix{kimeneti mátrix}       \\
        $\rmat D \in \mathcal{M}_{p \times m}$ & \kix{előrecsatolási mátrix} \\
      \end{tabular}
      \\\hline
    \end{tabular}
  \end{center}
  \caption{Az állapotegyenletekben szereplő mennyiségek}
  \label{table:state-variables}
\end{table}
\egroup

\begin{figure}[htb]
  \centering
  \begin{tikzpicture}[
      thick,
      arr/.style={ultra thick, draw=cyan!50!black},
      box/.style={rectangle, minimum width=1.2cm, minimum  height=.75cm, draw=black, },
      cross/.style={path picture={
              \draw[black]
              (path picture bounding box.south east) -- (path picture bounding box.north west) (path picture bounding
              box.south west)
              -- (path picture bounding box.north east);
            }},
      sum/.style={circle, minimum size=.75cm, draw=black, cross},
    ]
    \node[box] (z) at (0,0) {$\displaystyle\int$};
    \node[box] (a) at (0,-1.5) {$\rmat A$};
    \node[box] (b) at (-4,0) {$\rmat B$};
    \node[box] (c) at (2.5,0) {$\rmat C$};
    \node[box] (d) at (0,-3) {$\rmat D$};

    \node[sum] (s1) at (-2,0) {};
    \node[sum] (s2) at (4.5,0) {};

    \draw[arr,to-] (b) -- ++(-2,0);
    \draw[arr,-to] (b) -- (s1);
    \draw[arr,-to] (s1) -- (z);
    \draw[arr,-to] (z) -- (c);
    \draw[arr,-to] (c) -- (s2);
    \draw[arr,-to] (s2) --  ++(1.5,0);

    \draw[arr,-to] (a) -| (s1);
    \draw[arr,-to] (d) -| (s2);
    \draw[arr,-to] ($(b)-(1.25,0)$) |- (d);
    \draw[arr,-to] ($(c)-(1.25,0)$) |- (a);
  \end{tikzpicture}
  \caption{Állapottérmodell hatásvázlata folytonos időben}
  \label{fig:cstate}
\end{figure}

A differenciálegyenlet-renszer megoldásához végezzük el a \kix{Laplace transzformáció}[t].
\begin{alignat*}{9}
  s & \rvec X(s) - \rvec x_0 & = & \rmat A \rvec X(s) & + & \rmat B \rvec U(s) \\
    & \rvec Y (s)            & = & \rmat C \rvec X(s) & + & \rmat D \rvec U(s)
\end{alignat*}
\begin{gather*}
  (s \rmat I - \rmat A) \rvec X(s) = \rvec x_0 + \rmat B \rvec U(s) \\
  \rvec X(s) = (s \rmat I - \rmat A)^{-1} \rvec x_0 + (s \rmat I - \rmat A)^{-1} \rmat B \rvec U(s) \\
  \rvec Y(s)
  = \underbracket{\rmat C(s \rmat I - \rmat A)^{-1} \rvec x_0}_{\text{tranziens}}
  +  \underbracket{\left( \rmat C ( s \rmat I - \rmat A )^{-1} \rmat B + \rmat D \right)}_{\rmat W(s) \text{ --
      átviteli mátrix}} \rvec U(s)
\end{gather*}

\kix{SISO} rendszer esetén $W(s)$ átviteli függvényt kapunk. Ebből
\kix{inverz Laplace transzformáció} segítségével meghatározható az
\kix{időtartomány}[beli] megoldás. A \kix{kezdeti érték tétel}[lel]
és a \kix{végértéktétel}[lel] meghatározhatjuk a kimenet kezdeti értékét
és állandósult állapotát. (Utóbbi csak abban az esetben használható,
ha $y(t)$ konvergens.)
\begin{alignat*}{9}
   & y(0)      & = & \lim_{s \to \infty} & s W(s) \\
   & y(\infty) & = & \,\lim_{s \to 0}    & s W(s)
\end{alignat*}

Az egy- és kéttárolós tagok vizsgálójelekre adott válaszát a \ref{sec:3-14}
fejezetben tárgyaljuk.



\end{document}
