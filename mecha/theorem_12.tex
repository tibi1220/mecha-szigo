\documentclass[../main.tex]{subfiles}

\begin{document}

\section{Struktúragráf}

\begin{fulltheorem}
  Egy adott, tanult példa kapcsán ismertesse a struktúra gráf felrajzolásának
  lépéseit. Milyen feltételek teljesülése esetén és hogyan lehet csatolt
  kétpólus elemmel összekapcsolt rendszereket egy oldalra redukálni?
\end{fulltheorem}

A \kix{struktúragráf} egz irányított, összefüggő gráf.
Felrajzolásakor először állapítsuk meg, hogy a modellezendő renszernek milyen
potenciáljai vannak, ezek lesznek a csomópontok. Vegyünk például egy villamos hajtást
áttétellel. A villamos oldalt egy feszültségforrás, tekercs és ellenállás
sorbakapcsolásával modellezhetjük, tehát máris 3 potenciálunk van a referencián
kívül. Mivel a hajtás oldalon a rugalmasságot elhanyagoljuk, ezért áttét előtt
és után is csak egy-egy potenciált kell felvennünk (minden ugyanolyan sebességgel
forog). A rendszerek között váltó kapcsolat van, vagyis 2 transzformátorra is
szükségünk lesz.
\begin{figure}[H]
  \centering
  \includegraphics[scale=.85]{../static/tex/build/structure-graph-og.pdf}
  \caption{A modell struktúragráfja}
  \label{fig:structure-graph-og}
\end{figure}

A gráf és a tanult analógiák alapján felrajzolható az impedanciahálózat.
\begin{figure}[H]
  \centering
  \includegraphics[scale=.85]{../static/tex/build/circuit-og.pdf}
  \caption{A modell impedanciahálózata}
  \label{fig:circuit-og}
\end{figure}

A hálózat a mindenkori áramköri átalakításokkal egyszerűsíthető.
\begin{figure}[H]
  \centering
  \includegraphics[scale=.85]{../static/tex/build/circuit-simpler.pdf}
  \caption{Az egyszerűsített impedanciahálózat}
  \label{fig:circuit-simpler}
\end{figure}

Tudjuk, hogy: $\itOmega = k_m I$ és $M = U / k_e$. Ebből következik, hogy:
\[
  Z_\text{mech}
  = \frac{\itOmega}{M}
  = \frac{1}{k_m k_e} \frac{U}{I}
  = \frac{Z_\text{vill}}{k_m k_e}
\]

Vagyis, ha az elektromos elemeket az első négypólus jobb oldalára akarjuk
áthozni, akkor az áramköri elemek a változók közötti kapcsolat arányában
változnak meg.
\begin{figure}[H]
  \centering
  \includegraphics{../static/tex/build/circuit-elimination.pdf}
  \caption{Első négypólus eliminálása}
  \label{fig:circuit-elimination}
\end{figure}

\end{document}
