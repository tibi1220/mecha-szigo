\documentclass[../main.tex]{subfiles}

\begin{document}

\section{Átviteli függvények I}

\begin{fulltheorem}
  Milyen összefüggés van az átviteli és frekvencia-átviteli függvény között?
  Milyen feltétele van annak, hogy az átviteli függvényt egyszerű
  helyettesítéssel átírjuk frekvencia átviteli függvénnyé?
\end{fulltheorem}

Az \kix{átviteli függvény} alkalmas egy lineáris időinvariáns rendszer legfontosabb
tulajdonságainak leírására, de a fizikai hátterét azért nehéz közvetlenül
elemezni, mert egy komplex változós komplex függvény. Az átviteli függvény
megegyezik a súlyfüggvény Laplace-transzformáltjával. Ez azért egy fontos
megállapítás, mert a $\sum a_i s^i = \sum b_j s^j$ alakú differenciálegyenlettel
leírható rendszerek impulzus válasza (súlyfüggvénye) garantáltan lecsengő,
ezért létezik Fourier-transzformáltja, vagyis a \kix{frekvencia átviteli függvény}[e]
(frekvencia karakterisztikája) az átviteli függvényből egyszerű $s = \iu \omega$
helyettesítéssel megkapható. A frekvencia átviteli függvény fizikailag
értelmezhető.
\[
  W(s) \quad \overset{s = \iu \omega}{\longrightarrow} \quad W(\iu \omega)
\]

\end{document}
