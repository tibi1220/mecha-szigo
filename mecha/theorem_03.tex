\documentclass[../main.tex]{subfiles}

\begin{document}

\section{Leírási módok}

\begin{fulltheorem}
	Mi az alapvető különbség az átviteli függvény és állapotváltozós leírási mód
	között? Mutassa be a lineáris idő invariáns rendszerek esetén az állapottér
	egyenletek rendszermátrixának sajátértékei és a rendszer átviteli
	függvényének pólusai közötti összefüggést.
\end{fulltheorem}

\end{document}
