\documentclass[../../main.tex]{subfiles}

\begin{document}

\section{Adatszerkezetek}

\begin{fulltheorem}
	Adatszerkezetek. Tömbök, kapcsolt listák, gráf, fa, verem, sor.
\end{fulltheorem}

Az \kix{adatszerkezetek} egyszerű, vagy összetett alapadatok rendszerének
matematikai, logikai modellje, amely elég egyszerű a kezeléshez, és elég jó
a valós kapcsolatok tükrözésére.

\subsection{Tömbök}

\begin{itemize}
	\item Egy \kix{tömb} lehet lineáris, egy vagy többdimenziós.

	\item $n$ darab azonos típusú \kix{adatelem}[ből] áll.

	\item Az elemekre \kix{indexhalmaz}[zal] hivatkozunk.

	\item Az elemeket egymást követő memóriahelyek tárolják.

	\item Az elemekhez bejárás nélkül férünk hozzá.
\end{itemize}

\subsection{Kapcsolt listák}

\begin{itemize}
	\item A \kix{kapcsolt lista} vagy \kix{egyirányú lista}
	      adatelemek, vagy csomópontok lineáris gyűjteménye,
	      ahol az elemek sorrendjét \kix{mutatók} rögzítik.
	      \begin{itemize}
		      \item A mutatókat tároló elemeket \kix{kapcsolómező}[nek] hívjuk.
	      \end{itemize}

	\item A \kix{kétirányú lista} mindkét irányban bejárható.

	\item A \kix{ciklikusan kapcsolt lista} nem rendelkezik
	      első és utolsó elemmel, hiszen az "első" az "utolsóra" mutat.
\end{itemize}

\subsection{Gráf}

\begin{itemize}
	\item A \kix{gráf} két halmazzal jellemezhető adatszerkezet.
	      \begin{itemize}
		      \item a \kix{csomópont}[ok] sorszámozott halmazza

		      \item az elemeket összekötő számpárral jellemzett \kix{él}[ek] halmaza
	      \end{itemize}

	\item Az összekötött csomópontokat \kw{szomszéd}oknak hívjuk.

	\item $\deg(u)$ -- a csomópont \kw{fok}a a befutó élek száma.
	      \begin{itemize}
		      \item Ha $\deg(u) = 0$, akkor a csomópont \kw{izolált}.
	      \end{itemize}

	\item A $v_0$-ból $v_n$-be mutató élek halmazát
	      útnak nevezzük – $P \left( v_0; v_1; \dots ;v_n \right)$.
	      \begin{itemize}
		      \item Az út \kw{zárt}, ha $v_0 = v_n$
		      \item Az út \kw{egyszerű}, ha minden pontja különbözik
		      \item Az út \kw{kör}, ha legalább 3 hosszú, egyszerű
	      \end{itemize}

	\item Egy $G$ gráf \kw{összefüggő},
	      ha bármely 2 pontja között létezik egyszerű út.

	\item Egy $G$ gráf \kw{teljes},
	      ha minden csomópontja minden csomópontjával össze van kötve.

	\item Egy $G$ gráf \kw{címkézett}, ha éleihez adatokat rendelünk.
	      \begin{itemize}
		      \item egy $G$ gráf \kw{súlyozott},
		            ha éleihez rendelt adatok nemnegatívak
	      \end{itemize}

	\item Egy $G$ gráf \kw{irányított}, ha az éleknek irányítottságuk van.

	\item Rendelhetünk hozzájuk mátrixokat:
	      \begin{itemize}
		      \item \kix{szomszédsági mátrix} – $a_{ij} = 1$,
		            ha $i$-ből $j$ felé halad él

		      \item \kix{útmátrix} – $a_{ij} = 1$,
		            ha $i$-ből $j$-be halad valamilyen út
	      \end{itemize}
\end{itemize}

\subsection{Fa}

\begin{itemize}
	\item A \kix{fa} köröket nem tartalmazó összefüggő gráf.

	\item A \kix{bináris fa} elemek véges halmaza, mely:
	      \begin{itemize}
		      \item vagy üres, vagy egyetlen $\mathbf{T}$ elemhez (\kw{gyökér})
		            kapcsolt két diszjunkt $\mathbf{T}_1$ és $\mathbf{T}_2$
		            részfa alkotja. (\kix{szukcesszor})

		      \item A \kw{zárócsomópont}nak nincsen szukcesszora
		            (\kw{levél}), az utolsó élet \kw{ág}nak nevezzük.

		      \item Egy \kw{generáció}ba az azonos \kw{szintszámú} elemek tartoznak.
		            (gyökér szintszáma 0)

		      \item A \kw{mélység} az azonos ágon elhelyezkedő elemek maximális száma.

		      \item A fa \kw{teljes}, ha az utolsó szintet kivéve
		            a csomópontok száma maximális.

		      \item Ábrázolhatóak:
		            \begin{itemize}
			            \item[$\circ$] kapcsolt szerkezettel,
			            \item[$\circ$] tömbökkel,
			            \item[$\circ$] szekvenciálisan. ($2k$ helyek eltolva)
		            \end{itemize}

		      \item A bejárás történhet pl. irányítás szerint.
	      \end{itemize}

	\item Az \kix{általános fa} esetén nem csak 2 szukcesszor engedélyezett.
	      \begin{itemize}
		      \item Elemek véges halmaza ($\mathbf{T}$), amely \dots
		            \begin{itemize}
			            \item[$\circ$] tartalmaz egy kitüntetett
				            $\mathbf{R}$ gyökérelemet,
			            \item[$\circ$] a többi elem nem nulla diszjunkt
				            részfája $\mathbf{T}$-nek.
		            \end{itemize}
	      \end{itemize}

	\item A \kix{minimális feszítőfa}[ probléma]
	      (minimum spanning tree)
	      \begin{itemize}
		      \item minden csúcsot érintő, összefüggő,
		            körmentes élhalmaz

		      \item bemenet: összefüggő, súlyozott,
		            irányítatlan $G = (V; E)$ gráf, ahol $k(u; v)$
		            a súly, az $(u;v)$ az élköltséget fejezi ki

		      \item kimenet: egy $\mathbf{F}$ feszítőfa,
		            melyre az élköltség minimális:
		            \[k(\mathbf{F}) = \sum_{(u;v) \in \mathbf{F}} k(u;v)\]
		      \item megoldás \kix{Kruskal algoritmus}[sal], amely egy
		            \kix{mohó algoritmus}
		            \begin{itemize}
			            \item[$\circ$] minden pillanatban a leghatékonyabb megoldást válassza ki
		            \end{itemize}
	      \end{itemize}
\end{itemize}

\subsection{Verem}

\begin{itemize}
	\item A \kix{verem} (\cix{stack}) \tc{LIFO}
	      típusú tároló.

	\item \cix{push} és \cix{pop} függvényekkel rendelkezik.
\end{itemize}

\subsection{Sor}

\begin{itemize}
	\item A \kix{sor} (\cix{queue}) \tc{FIFO}
	      típusú tároló.

	\item \cix{push} és \cix{pop} függvényekkel rendelkezik
\end{itemize}

\end{document}
