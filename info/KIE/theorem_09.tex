
\documentclass[../../main.tex]{subfiles}

\begin{document}

\section{Az adatbázisok alapjai}

\begin{fulltheorem}
	Az adatbázisok alapjai. Adatmodellezés. Egyed-kapcsolat modell. Kapcsolatok típusai.
\end{fulltheorem}

\subsection{Az adatbázis}

Az \kix{adat}[ok] nyers tények, feldolgozatlan információ. Az
\kix{információ} feldolgozott adat, az információs rendszerek hozzák létre,
keresik vissza, dolgozzák fel.

Az \kix{adatbázis} adatok gyűjteménye, amelyet egy adatbázis-kezelő rendszer
kezel, tehát nemcsak az adatok rendezett tárolását, hanem azok kezelését is
lehetővé teszi, mert kapcsolatok nélkül az adatok eltérően értelmezhetőek.

Alapfunkciói: létrehozás, adatok mentése, lekérdezések, adatvédelem.
Az adatbázis-kezelő rendszer (DBMS) segítségével lehetséges a tárolt
adatok definiálása, kezelése, karbantartása, felügyelete.

Követelmények egy adatbázissal szemben:
\begin{itemize}
	\item \tc{DDL} -- Új adatok létrehozása adatdefiníciós
	      nyelv segítségével.

	\item \tc{SQL} – Meglévő adatok lekérdezése, módosítása
	      lekérdező vagy adatmanipulációs nyelvvel.

	\item Támogassa az adatok hosszú távú, biztonságos tárolását.

	\item Felügyelje a több felhasználó által egy
	      időben történő hozzáférését.

	\item \kw{adatintegritás} \tabto{4.5cm} -- \tabto{5.5cm}
	      érvényesség, helyesség, ellentmondás-mentesség

	\item \kw{rugalmasság} \tabto{4.5cm} -- \tabto{5.5cm}
	      adatok egyszerűen módosíthatóak

	\item \kw{hatékonyság} \tabto{4.5cm} -- \tabto{5.5cm}
	      gyors és hatékony keresés, módosíthatóság

	\item \kw{adatfüggetlenség} \tabto{4.5cm} -- \tabto{5.5cm}
	      hardver és szoftverfüggetlenség

	\item \kw{adatbiztonság} \tabto{4.5cm} -- \tabto{5.5cm}
	      védelem a hardver és szoftverhibák ellen

	\item \kw{adatvédelem} \tabto{4.5cm} -- \tabto{5.5cm}
	      illetéktelen felhasználókkal szemben

	\item \kw{osztott hozzáférés} \tabto{4.5cm} -- \tabto{5.5cm}
	      több felhasználó egyidejű hozzáférése

	\item \kw{integritási kényszerek} \tabto{4.5cm} -- \tabto{5.5cm}
	      szabályok, amiket figyelembe kell venni

	\item \kw{tranzakciók} \tabto{4.5cm} -- \tabto{5.5cm}
	      felhasználó általi változtatás nem végleges rögtön
\end{itemize}

\subsection{Adatmodellezés}
Az \kix{adatmodellezés} segítséget nyújt a környezővilág megértésében és
leképezésében, a lényeges jellemzők kiemelésében. Az adatmodell az \kix{adat}[ok]
és az azok közötti összefüggések leírására szolgál.

A \kix{modell} olyan mesteséges rendszer, amely a felépítésében és
viselkedésében megegyezik a vizsgált környező rendszerrel.

Adatmodellnek nevezzük az adatok struktúrájának (felépítésének) leírására
szolgáló modelleket.

\subsection{Egyed--kapcsolat modell}

Az \kix{ETK modell} az alábbi három fogalom együttese:
\begin{itemize}
	\item \kix{Egyed} -- Az információs rendszert felépítő személyek,
	      tárgyak, események.
	      \begin{itemize}
		      \item Nem elszigetelten, hanem kapcsolatban állnak
		            egymással és más objetumokkal.
	      \end{itemize}

	\item \kix{Tulajdonság} -- Az egyedeket jellemzi, egy érték,
	      melynek tulajdonságtípusa van.
	      \begin{itemize}
		      \item \kw{azonosító}
		            \tabto{5.5cm} – \tabto{6.5cm}
		            egyedi, nem ismétlődhet (\blackc{ID})

		      \item \kw{leíró tulajdonság}
		            \tabto{5.5cm} – \tabto{6.5cm}
		            ismétlődhet (\blackc{név})

		      \item \kw{gyengén jellemző tulajdonság}
		            \tabto{5.5cm} – \tabto{6.5cm}
		            lehet üres is (\blackc{kedvenc szín})

		      \item \kw{kapcsoló tulajdonság}
		            \tabto{5.5cm} – \tabto{6.5cm}
		            itt leíró, ott azonosító (\blackc{szül hely})
	      \end{itemize}

	\item \kix{Kapcsolat} -- Két egyed közötti viszony.
	      \begin{itemize}
		      \item Lehet többszintű, bonyolult, melyben
		            több rendszer is kapcsolatban áll egymással.
	      \end{itemize}
\end{itemize}

\subsection{Kapcsolatok típusai}

\begin{itemize}
	\item (\texttt{1-1}) – "{egy az egyhez}"
	      \begin{itemize}
		      \item Egy egyedtípus egy egyedéhez egy másik
		            egyedtípus csak egyetlen egyede kapcsolódhat,
		            és fordítva is igaz.
		            (\texttt{osztály–osztályfőnök})
	      \end{itemize}

	\item (\texttt{1-N}) – "{egy a többhöz}"
	      \begin{itemize}
		      \item Egy egyedtípus egy egyedéhez egy másik
		            egyedtípus több egyede is kapcsolódhat,
		            de fordítva nem igaz.
		            (\texttt{osztály–tanuló})
	      \end{itemize}

	\item (\texttt{N-M}) – "{több a többhöz}"
	      \begin{itemize}
		      \item Egy egyedtípus egy egyedéhez egy másik
		            egyedtípus több egyede is kapcsolódhat,
		            de fordítva is igaz.
		            (\texttt{osztály–tanár})
	      \end{itemize}

	\item Az \kix{adatbázis} fogalma a kapcsolatok alapján:
	      \begin{itemize}
		      \item Véges számú egyedek és azok véges számú tulajdonságainak
		            és kapcsolatainak adatmodell szerinti szervezett együttese.
	      \end{itemize}

	      \begin{figure}[H]
		      \flushright
		      \begin{tikzpicture}[node distance=8em]
			      \node[entity] (ent1) {Egyed${}_1$};
			      \node[relationship] (rel1) [right of=ent1] {Kapcsolat${}_1$} edge[thick] (ent1);
			      \node[attribute] (attr1) [below left of=rel1] {\underline{Kulcstul}} edge[<-, thick] (rel1);
			      \node[attribute] (attr2) [below of=rel1] {Többértékű} edge[<<-, thick] (rel1);
			      \node[attribute] (attr3) [below right of=rel1] {\xcancel{Összetett}} edge[<-,thick] (rel1);
			      \node[entity] (ent2) [right of=rel1] {Egyed${}_2$} edge[thick] (rel1);
			      \node[relationship] (rel2) [right of=ent2] {Kapcsolat${}_2$} edge[thick] (ent2);
			      \node[attribute] (attr4) [below of=rel2] {Egyszerű} edge[<-, thick] (rel2);
			      \node[attribute] (attr5) [below right of=rel2] {Egyértékű} edge[<-, thick] (rel2);
		      \end{tikzpicture}
		      \caption{A kapcsolatok szemléltetése}
	      \end{figure} %tikz

	\item a \kix{kapcsolat foka} lehet:
\end{itemize}
\begin{table}[H]
	\centering\begin{tabular}{|c c|}
		\hline
		\kix{Unáris}   & rekurzív         \\
		\kix{Bináris}  & két résztvevős   \\
		\kix{Trináris} & három résztvevős \\
		\hline
	\end{tabular}
	\caption{Adatok közötti kapcsolatok foka}
	\label{fig:datadeg}
\end{table}

\end{document}
