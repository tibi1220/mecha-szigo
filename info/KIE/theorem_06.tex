\documentclass[../../main.tex]{subfiles}

\begin{document}

\section{A számítástudomány alapjai}

\begin{fulltheorem}
  A számítástudomány alapjai. Turing gép. Eljárások, algoritmusok.
\end{fulltheorem}

\subsection{Turing gép}

A \kix{Turing gép} jellemzői

\begin{itemize}
  \item Külső adat és tárolóterület:
        végtelen szalag, amelynek egymás után cellái vannak,
        amelyek vagy üresek, vagy jelöltek.

  \item A gép egyszerre egy cellával foglalkozik.
        (Az író/olvasó feje egy cellán áll.)

  \item A szalagon tud jobbra-balra lépni,
        tud jelet olvasni, törölni és írni.

  \item A bevitel, a számítás és a kivitel minden
        konkrét esetben véges marad, ezen túl a szalag üres.

  \item A gép belső állapotait számozzuk.

  \item A gép működését megadja egy explicit helyettesítési táblázat.

  \item Ha egy \kix{algoritmus} elég mechanikus és világos,
        akkor található olyan Turing gép, amely azt végrehajtja.

  \item A Turing gép definiálja mindazt,
        amit matematikailag algoritmikus eljárás alatt értünk.

  \item Minden más algoritmikus eljárást végrehajtó rendszer
        ekvivalens valamelyik Turing géppel

  \item \kix{Megállási probléma}
        -- Nem tudjuk, hogy egy adott programmal megáll-e.

  \item Nincs arra bizonyítási módszerünk,
        hogy egy eljárás biztosan algoritmus.
\end{itemize}

\subsection{Eljárások}

Egy \kix{eljárás} esetén nem garantálható, hogy véges lépésben,
tehát valaha is választ kapjunk kérdésünkre.

\subsection{Algoritmusok}

Egy \kix{algoritmus} esetén a választ véges számú lépés után mindenképpen
megkapjuk. Egy véges \kix{utasítássorozat}, amely bármely input esetén
véges lépésszám után megáll, eredményt ad.
Minden algoritmus leírható az alábbi logikai struktúrákkal:
\begin{multicols}{3}
  \begin{itemize}
    \item rákövetkeztetés \\ (\kix{konkatenáció})
    \item választás \\ (\kix{alternáció})
    \item ciklus \\ (\kix{iteráció})
  \end{itemize}
\end{multicols}
Maga a Turing gép matematikai leírása az algoritmus fogalmának formális
definíciója. Minden probléma, amelyre eljárás, procedúra szerkeszthető,
Turing géppel megoldható. Az ember azokra és csakis azokra a kérdésekre
tud választ adni, amelyekre a Turing-gép is képes. A tartalmazás kérdése
algoritmikusan eldönthetetlen.
\begin{multicols}{2}
  \begin{itemize}
    \item \kix{bejárás} – elemek keresése

    \item \kix{keresés} – adott értéknek megfelelő elemek kiválasztása

    \item \kix{beszúrás} – új adat beillesztése

    \item \kix{törlés} – adatelem eltávolítása

    \item \kix{rendezés} – elemeket logikai sorrendbe

    \item \kix{összeválogatás} – különböző rendezett
          adathalmazokból új elemhalmaz kialakítása
  \end{itemize}
\end{multicols}

\end{document}
