\documentclass[../../main.tex]{subfiles}

\begin{document}

\section{Relációs adatbázismodell}

\begin{fulltheorem}
	Relációs adatbázismodell. Relációk jellemzői. A relációs algebra műveletei. SQL alapok.
\end{fulltheorem}

\subsection{Adatbázis szerkezetek}

A leggyakoribb adatszerkezetek: hierarchikus, hálós, relációs.

\kix{Hierarchikus} esetben csak egy a többhöz (\texttt{1-N}) kapcsolatok
képezhetőek le. Az adatok a tárolt hierarchia szerint érhetőek el.
Fastruktúrával szemléltethető, hiszen az adatokat alá-fölérendeltségi
viszonnyal meghatározható szerkezettel írjuk le.

\kix{Hálós} adatbázismodell esetén a kapcsolatok gráfokkal írhatóak le,
ahol a csomópontokat élekkel kötjük össze. Csak a tárolt kapcsolat mentén
járható be. A modellel egy a többhöz (\texttt{1-N}) és több a többhöz
(\texttt{N-M}) kapcsolatot és leírhatunk.

\subsection{A relációk jellemzői}

\kix{Relációs} modell esetén az adatokat táblázatos formában tároljuk.
Nincsenek előre meghatározott kapcsolatok. Előnyei, hogy egyszerűen
értelmezhető és átlátható, rugalmas és könnyen kezelhető, valamint
a \kix{reláció}[kat] a \kix{relációs algebra} segítségével írhatjuk le.

A reláció egy adattábla, melynek elemei:
\begin{table}[H]
	\centering\begin{tabular}{|c c|}
		\hline
		\kix{rekord}    & a táblázat sorai              \\
		\kix{attibútum} & a táblázat oszlopai           \\
		\kix{mező}      & a sorok és oszlopok metszetei \\ \hline
	\end{tabular}
	\caption{A reláció részei}
	\label{fig:relation}
\end{table}

A reláció rekordjaiban tároljuk a logikailag összefüggő adatokat.
A relációban tárolt rekordok száma a reláció számossága.
Az oszlopokban azonos tulajdonságokra vonatkozó adatok jelennek meg.
Egy tábla nem tartalmazhat azonos nevű oszlopot.
Az oszlopok száma a reláció foka.
Egy relációra vonatkozó követelmények:
\begin{itemize}
	\item Minden rekordja különböző.
	\item Nincs két azonos attribútum.
	\item Minden rekord mezőszerkezete azonos.
	\item A rekordok és attribútumok sorrendje tetszőleges.
\end{itemize}

\subsection{A relációs algebra műveletei}

A műveletek lehetnek:

\begin{enumerate}
	\item \kix{Adatkezelő}[ műveletek] --
	      adatbevitel, törlés, módosítás.

	\item \kix{Adatlekérdező}[ műveletek] --
	      relációs algebra műveleteivel.
\end{enumerate}

A \kix{relációs algebra} műveletei lehetnek egy-, vagy többoperandusúak.
Az előbbit egy, az utóbbit pedif több változón végezzük el.

\begin{table}[H]
	\centering\begin{tabular}{|c c|}
		\hline
		\kix{Szelekció}     & rekordokat választunk ki (kiválasztás)         \\
		\kix{Projekció}     & attribútumokat választunk ki (vetítés)         \\
		\kix{Kiterjesztés}  & matematikai műveletekkel új oszlop             \\
		\kix{Csoportosítás} & attribútumok alapján csoport, majd hozzá érték \\
		\hline
	\end{tabular}
	\caption{Egyoperandusú relációs algebra műveletek}
	\label{fig:rel1}
\end{table}

\begin{table}[H]
	\centering\begin{tabular}{|c c|}
		\hline
		\kix{Descartes-szorzat}   & 2 reláció sorainak összes kombinációja \\
		\kix{Összekapcsolás}      & összekapcsolás attribútum alapján      \\
		\kix{Unió}, \kix{Metszet} & kommutatív, mint halmazelmélet         \\
		\kix{Különbség}           & nem kommutatív, mint halmazelmélet     \\
		\hline
	\end{tabular}
	\caption{Többoperandusú relációs algebra műveletek}
	\label{table:rel+}
\end{table}

Az azonosítás történhet \kix{kulcs} alapján, amely egyértelműen
azonosítja az egyedet az egyedhalmazon belül. Amennyiben egyetlen
attribútumból áll, akkor egyszerű, egyébként összetett.
Megadható több kulcs is, amire szükségünk van az adott feladatnál,
azt \kix{elsődleges kulcs}[nak] nevezzük, a többi mind másodlagos.
\kix{Idegen kulcs} egy reláció olyan attribútumai,
melyek egy másikban elsődlegesek.

\subsection{SQL alapok}

Hogy elkerüljük az \kw{anomáliákat}, az adatbázisokat
\kw{normalizálni} szokták. Ennek lényege, hogy a
változtatási anomáliák megszűnjenek. (módosítási, beírási,
törlési) A normalizálásnak több szintje létezik.
Minden \kix{relációs séma} megköveteli legalább
az első normálformát. Gyakorlatban a harmadikig szokták.
\begin{itemize}
	\item {1. normál forma}
	      \begin{itemize}
		      \item Ha a mezők függéseinek rendszerében
		            létezik egy olyan kulcs, amelytől minden
		            más mező függ, azaz minden mezője
		            funkcionálisan függ a kulcsmező csoporttól.
	      \end{itemize}

	\item {2. normál forma}
	      \begin{itemize}
		      \item Nincs benne részleges függés, azaz bármely nem
		            kulcs mező a teljes kulcstól függ, de nem függ a
		            kulcs bármely részhalmazától.
	      \end{itemize}

	\item {3. normál forma}
	      \begin{itemize}
		      \item Nem áll fenn tranzitív függőség, azaz
		            a nem kulcs mezők nem függnek egymástól,
		            tehát nincs funkcionális függőség a \textbf{nem}
		            elsődleges attribútumok között.
	      \end{itemize}
\end{itemize}

\subsubsection*{Definíció}
\begin{itemize}
	\item \bluec{CREATE TABLE} \blackc{table\textunderscore{}name
		      (column1 datatype cond,\dots);}
	      \\ \hspace*{\fill}
	      \orangec{objektum létrehozása}

	\item \bluec{DROP TABLE} \blackc{table\textunderscore{}name;}
	      \\ \hspace*{\fill}
	      \orangec{objektum megszűntetése}

	\item \bluec{ALTER TABLE} \blackc{table\textunderscore{}name}
	      \bluec{ADD|MODIFY} \blackc{(column1 datatype cond|cond);}
	      \\ \hspace*{\fill}
	      \orangec{objektum séma módosítása}
\end{itemize}

\begin{minipage}[c]{0.5\textwidth}
	\begin{table}[H]
		\centering\begin{tabular}{|c|}
			\hline
			\blackc{PRIMARY KEY}                            \\
			\blackc{NOT NULL}                               \\
			\blackc{UNIQUE}                                 \\
			\blackc{CHECK cond}                             \\
			\blackc{REFERENCING table\textunderscore{}name} \\
			\hline
		\end{tabular}
		\caption{SQL attribútum feltételek}
		\label{table:sql-conditions}
	\end{table}
\end{minipage}\hfill
\begin{minipage}[c]{0.5\textwidth}
	\begin{table}[H]
		\centering\begin{tabular}{|c|}
			\hline
			\blackc{CHAR(n)}
			\\
			\blackc{NUMBER(n,m)}
			\\
			\blackc{DATE}
			\\ \hline
		\end{tabular}
		\caption{SQL függvények}
		\label{table:sql-functions}
	\end{table}
\end{minipage}\hfill

\subsubsection*{Módosítás}
\begin{itemize}
	\item \bluec{INSERT INTO} \blackc{table\textunderscore{}name}
	      \bluec{VALUES} \blackc{(field=value);}
	      \\ \hspace*{\fill}
	      \orangec{rekord felvitele}

	\item \bluec{DELETE FROM} \blackc{table\textunderscore{}name}
	      \bluec{WHERE} \blackc{cond;}
	      \\ \hspace*{\fill}
	      \orangec{rekord törlése}

	\item \bluec{UPDATE} \blackc{table\textunderscore{}name}
	      \bluec{SET} \blackc{field=value,\dots} \bluec{WHERE} \blackc{cond;}
	      \\ \hspace*{\fill}
	      \orangec{rekord módosítása}
\end{itemize}

\subsubsection*{Adatok lekérdezése}
\begin{itemize}
	\item \bluec{SELECT} \blackc{column1, column2, \dots}
	      \bluec{FROM} \blackc{table\textunderscore{}name1, table\textunderscore{}name2,}
	      \bluec{WHERE} \blackc{cond;}

	      \vspace{.5em}
	      \begin{table}[H]
		      \centering\begin{tabular}{|l r|}
			      \hline
			      \bluec{GROUP BY}             & csoportosítás            \\
			      \bluec{HAVING} \blackc{cond} & \hspace{2em} megszorítás \\
			      \bluec{ORDER BY}             & rendezés                 \\
			      \hline
		      \end{tabular}
		      \caption{SQL rendezések}
		      \label{table:sql-order}
	      \end{table}

	\item \bluec{SELECT} \blackc{column} \bluec{FROM}
	      \blackc{table\textunderscore{}name;}
	      \hfill \orangec{projekció}

	\item \bluec{SELECT} \blackc{column} \bluec{FROM}
	      \blackc{table\textunderscore{}name}
	      \bluec{WHERE} \blackc{cond;}
	      \hfill \orangec{szelekció}

	\item \bluec{SELECT} \blackc{*} \bluec{FROM}
	      \blackc{table\textunderscore{}name1, table\textunderscore{}name2;}
	      \hfill \orangec{Descartes-szorzat}

	\item \bluec{SELECT} \blackc{expr column,\dots}
	      \bluec{FROM} \blackc{table\textunderscore{}name;}
	      \hfill \orangec{kiterjesztés}

	\item \bluec{SELECT} \blackc{aggregation}
	      \bluec{FROM} \blackc{table\textunderscore{}name;}
	      \hfill \orangec{aggregáció megadása}

	      \vspace{.5em}
	      \begin{table}[H]
		      \centering\begin{tabular}{|c|}
			      \hline
			      \bluec{SUM}\blackc{(expr)}   \\
			      \bluec{COUNT}\blackc{(expr)} \\
			      \bluec{MIN}\blackc{(expr)}   \\
			      \bluec{AVG}\blackc{(expr)}   \\
			      \bluec{MAX}\blackc{(expr)}   \\
			      \hline
		      \end{tabular}
		      \caption{SQL aggregációs függvények}
		      \label{table:sql-aggregation}
	      \end{table}

	\item \bluec{SELECT} \blackc{aggregation}
	      \bluec{FROM} \blackc{table\textunderscore{}name;}
	      \bluec{GROUP BY} \blackc{expr;}
	      \\ \hspace*{\fill}
	      \orangec{aggregáció, csoportképzés}

	\item \bluec{SELECT} \blackc{column}
	      \bluec{FROM} \blackc{table\textunderscore{}name;}
	      \bluec{ORDER BY} \blackc{column1 mode1, \dots;}
	      \\ \hspace*{\fill}
	      \orangec{eredmény rekordok rendezése}

	      \begin{table}[H]
		      \centering\begin{tabular}{|c|}
			      \hline
			      \bluec{ASC}  \\
			      \bluec{DESC} \\
			      \hline
		      \end{tabular}
		      \caption{SQL rendezési sorrendek}
		      \label{table:sql-ascdesc}
	      \end{table}
\end{itemize}

\subsubsection*{Adatok lekérdezése}

\begin{table}[H]
	\centering\begin{tabular}{|l r|}
		\hline
		\bluec{=}                               & egyenlő         \\
		\bluec{\textless\textgreater,\string^=} & nem egyenlő     \\
		\bluec{\textgreater}                    & nagyobb         \\
		\bluec{\textgreater=}                   & nagyobb egyenlő \\
		\bluec{\textless}                       & kisebb          \\
		\bluec{\textless=}                      & kisebb egyenlő  \\
		\hline
	\end{tabular}
	\caption{SQL operátorok}
	\label{table:sql-operators}
\end{table}
\begin{table}[H]
	\centering\begin{tabular}{|l r|}
		\hline
		\bluec{NOT} & tagadás \\
		\bluec{AND} & és      \\
		\bluec{OR}  & vagy    \\
		\hline
	\end{tabular}
	\caption{SQL logikai műveletke}
	\label{table:sql-logic}
\end{table}
\begin{table}[H]
	\centering\begin{tabular}{|l r|}
		\hline
		\bluec{LIKE} \bluec{'a\%'}
		 & \bluec{'a'}-val kezdődik                     \\
		\bluec{LIKE} \bluec{'x\textunderscore'}
		 & \bluec{'x'}-val kezdődik, 2 betű             \\
		\bluec{LIKE} \bluec{'\%a\%'}
		 & \bluec{'a'}-t tartalmaz                      \\
		\bluec{LIKE} \bluec{'\textunderscore{}a\%x'}
		 & 2. betű \bluec{'a'}, \bluec{'x'}-re végződik \\
		\hline
	\end{tabular}
	\caption{SQL szöveg összehasonlítása}
	\label{table:sql-string}
\end{table}
\begin{table}[H]
	\centering\begin{tabular}{|l r|}
		\hline
		\bluec{BETWEEN}\blackc{ x }\bluec{AND}\blackc{ y}
		                                   & \hspace{.25em} adott értékek közé esik \\
		\bluec{IN}\blackc{(a, b, c,\dots)} & értékek között van                     \\
		\bluec{LIKE} \blackc{sample}       & hasonlít a mintára                     \\
		\hline
	\end{tabular}
	\caption{SQL összehasonlító operátor halmazokra}
	\label{table:sql-comp}
\end{table}

\end{document}
