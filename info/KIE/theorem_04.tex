\documentclass[../../main.tex]{subfiles}

\begin{document}

\section{Az objektum-orientált programozás alapjai IV}

\begin{fulltheorem}
	Az objektum-orientált programozás alapjai. Polimorfizmus. Virtuális
	alaposztályok. Abstract osztály. Általánosított osztályok.
\end{fulltheorem}

\subsection{Polimorfizmus}

A szó jelentése: ugyanaz a metódus más- és másképpen működik a családfa
szintjein. A \kix{tagfüggvény}[ek] \kix{újradefiniálható}[ak]. Lehet a
fordításkor korai, statikus kötés, vagy futás közben késői, vagy dinamikus
kötés. Ősosztály-típusú mutatóval meghívható az ős metódusa a
leszármazottra. (mint \cix{super})

\begin{minted}[bgcolor=bg, linenos]{cpp}
class Parent {
  int x;
public:
  Parent(int x) : x(x) {}
};

class Child : public Parent {
public:
  Child(int x) : Parent(x) {}
};
\end{minted}

\subsection{Virtuális alaposztályok}

A \cix{virtual} kulcsszót használva a tagfüggvények \kix{átdefiniálható}[ak]
lesznek a leszármazottakban. \tc{"=0"} esetén tisztán \kix{virtuális} lesz.

\begin{minted}[bgcolor=bg, linenos]{cpp}
class MyClass {
  virtual void func() = 0;
};
\end{minted}

\subsection{Abstract osztály}

Tisztán virtuális függvényt tartalmazó osztály nem példányosítható.
Ezeket \kix{abstract osztály}[oknak] hívjuk. Az \kix{interface} olyan absztrakt
osztály, amely csak tisztán \kix{virtuális függvény}[eket] tartalmaz.

\subsection{Általánosított osztályok}

Az \kix{általánosított osztály}[ok] a \cix{template} kulcsszó segítségével
hozhatóak létre.

\begin{minted}[bgcolor=bg, linenos]{cpp}
template<typename T>
class ClassName{
  T myFunc (T myVar){
    // { ... }
    return myVar;
  }
};
\end{minted}

A \cix{template} minden használó \kix{forrásállomány}[ban] kell.

\end{document}
