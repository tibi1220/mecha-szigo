\documentclass[../main.tex]{subfiles}

\begin{document}

\section{Az operációs rendszerek alapjai III}

\begin{fulltheorem}
	Az operációs rendszerek alapjai. Ütemezési algoritmusok.
	Előbb jött -- előbb fut algoritmus. A legrövidebb előnyben algoritmus.
	Körbenforgó algoritmus.
\end{fulltheorem}

\subsection{Ütemezési algoritmusok}

\kix{Többfeladatos} (\cix{multitask}) rendszereknél a folyamatok közötti
átkapcsolást, azaz a környezetváltást az alacsony szintű ütemezési algoritmusok
végzik. Általában a gyakorlatban többféle módszer kombinációját alkalmazzák.
Alap algoritmusok:
\begin{itemize}
	\item \cix{FCFS} \tabto{1.2cm} – \tabto{1.8cm}
	      First Come First Served

	\item \cix{SJF} \tabto{1.1cm} – \tabto{1.8cm}
	      Shortest Job First

	\item \cix{RR} \tabto{1cm} – \tabto{1.8cm}
	      Round Robin
\end{itemize}

\subsection{Előbb jött -- előbb fut algoritmus}

Az \cix{FCFS} algoritmus esetén a folyamatok érkezési sorrendjükben kapják meg
a processzort. Előnye, hogy ez a legegyszerűbb. Hátránya, hogy a várakozási
idő nagymértékben függ az érkezési időtől.
(csorda hatás, lassú kamion effektus)

\begin{table}[H]
	\centering
	\begin{tabular}{|c|c|c|c|c|c|}
		% {|p{1cm} | p{1.5cm}| p{1.5cm}| p{1.8cm}| p{1.8cm}| p{2.2cm}|}
		\hline
		            & \texttt{érk. idő} & \texttt{CPU igény} & \texttt{kezd. időpont} & \texttt{bef. időpont} & \texttt{várakozási idő}
		\\ \hline
		\texttt{P1} & \texttt{0}        & \texttt{14}        & \texttt{0}             & \texttt{14}           & \texttt{0}              \\
		\texttt{P2} & \texttt{7}        & \texttt{8}         & \texttt{14}            & \texttt{22}           & \texttt{7}              \\
		\texttt{P3} & \texttt{11}       & \texttt{36}        & \texttt{22}            & \texttt{58}           & \texttt{11}             \\
		\texttt{P4} & \texttt{20}       & \texttt{10}        & \texttt{58}            & \texttt{68}           & \texttt{38}             \\
		\hline
	\end{tabular}
	\caption{Az \cix{FCFS} algoritmus működése}
	\label{table:fcfs}
\end{table}

Jelen esetben az átlagos várakozási idő:
$\mathtt{0 + 7 + 11 + 38 = 56}
	\hspace*{1em}\rightarrow\hspace*{1em}
	\mathtt{56/4 = 14}$.

\subsection{A legrövidebb előnyben algortmus}

Az \cix{SJF} algoritmus esetén a \cix{CPU}-t egy folyamat befejezése után a
legrövidebbnek adja oda. Előnye, hogy így a legrövidebb a várakozási idő.
Hátránya, hogy tudni kell előre a folyamatok hosszát, illetve hogy
\kix{kiéheztet}[i] a hosszú folyamatokat.

\begin{table}[H]
	\centering
	\begin{tabular}{|c|c|c|c|c|c|}
		% {|p{1cm} | p{1.5cm}| p{1.5cm}| p{1.8cm}| p{1.8cm}| p{2.2cm}|}
		\hline
		            & \texttt{érk. idő} & \texttt{CPU igény} & \texttt{kezd. időpont} & \texttt{bef. időpont} & \texttt{várakozási idő}
		\\ \hline
		\texttt{P1} & \texttt{0}        & \texttt{14}        & \texttt{0}             & \texttt{14}           & \texttt{0}              \\
		\texttt{P2} & \texttt{7}        & \texttt{8}         & \texttt{14}            & \texttt{22}           & \texttt{7}              \\
		\texttt{P4} & \texttt{20}       & \texttt{10}        & \texttt{22}            & \texttt{32}           & \texttt{2}              \\
		\texttt{P3} & \texttt{11}       & \texttt{36}        & \texttt{32}            & \texttt{68}           & \texttt{21}             \\
		\hline
	\end{tabular}
	\caption{A \cix{SJF} algoritmus működése}
	\label{table:sjf}
\end{table}

Jelen esetben az átlagos várakozási idő:
$\mathtt{0 + 7 + 2 + 21 = 30}
	\hspace*{1em}\rightarrow\hspace*{1em}
	\mathtt{30/4 = 7,5}$.

\subsection{Körbenforgó algoritmus}

Az \cix{RR} algoritmus esetén a folyamatokat egy zárt körbe szervezzük, és
minden folyamat egy előre rögzített maximális időre kapja meg a processzort,
majd visszaáll a sor végére. Kombinálható prioritások bevezetésével.
(minden prioritási szintnek "saját köre" van) Előnye, hogy egyszerű, és
nincsen kiéheztetés. Hátránya, hogy az időszeletek lejártakor a folyamat
állapotát el kell menteni. (időveszteség)

\begin{table}[H]
	\centering
	\begin{tabular}{|c|c|c|c|c|c|}
		% {|p{1cm} | p{1.5cm}| p{1.5cm}| p{1.8cm}| p{1.8cm}| p{2.2cm}|}
		\hline
		              & \texttt{érk. idő} & \texttt{CPU igény} & \texttt{kezd. időpont} & \texttt{bef. időpont} & \texttt{várakozási idő}
		\\ \hline
		\texttt{P1}   & \texttt{0}        & \texttt{14}        & \texttt{0}             & \texttt{10}           & \texttt{0}              \\
		\texttt{P2}   & \texttt{7}        & \texttt{8}         & \texttt{10}            & \texttt{18}           & \texttt{3}              \\
		(\texttt{P1}) & \texttt{10}       & \texttt{4}         & \texttt{18}            & \texttt{22}           & \texttt{8}              \\
		\texttt{P3}   & \texttt{11}       & \texttt{36}        & \texttt{22}            & \texttt{32}           & \texttt{11}             \\
		\texttt{P4}   & \texttt{20}       & \texttt{10}        & \texttt{32}            & \texttt{42}           & \texttt{12}             \\
		(\texttt{P3}) & \texttt{32}       & \texttt{26}        & \texttt{42}            & \texttt{52}           & \texttt{10}             \\
		(\texttt{P3}) & \texttt{42}       & \texttt{16}        & \texttt{52}            & \texttt{62}           & \texttt{0}              \\
		(\texttt{P3}) & \texttt{52}       & \texttt{6}         & \texttt{62}            & \texttt{68}           & \texttt{0}              \\
		\hline
	\end{tabular}
	\caption{A \cix{RR} algoritmus működése}
	\label{table:rr}
\end{table}

Jelen esetben az átlagos várakozási idő:
$\mathtt{44/4 = 11}$. (Az időszelet: \texttt{10})

\end{document}
