\documentclass[../../main.tex]{subfiles}

\begin{document}

\section{Az objektum-orientált programozás alapjai V}

\begin{fulltheorem}
	Standard Template Library. Tárolók. Bejárók. Algoritmusok. Függvényobjektumok.
\end{fulltheorem}

\subsection{Standard Template Library}

Adatstruktúrákat és algoritmusokat tartalmaz \cix{C++} nyelvhez.

\subsection{Tárolók}

Soros és asszociatív tárolókat tartalmaz az \cix{STL}.
A \kix{soros tároló}[k] jellemzője, hogy megőrzik az elemek beviteli
sorrendjét. Az \kix{asszociatív tároló}[k] elemei egy kulcs alapján érhetőek
el. Lehetnek rendezettek, illetve rendezetlenek. Mindegyiknek van \cix{default}
és \cix{copy} konstruktora, \blackc{=} operátora.

\subsubsection*{Soros tárolók}

\paragraph*{Array}

Az \cix{array} adott méretű és típusú tárolótömb folytonos memóriaterületen.
\cix{for} ciklussal bejárható.

\begin{minted}[bgcolor=bg, linenos]{cpp}
#include <array>

array<type, size> myArr { {myVar1, myVar2, myVar3} };

myArr[0] = 12; // feltöltés [] operátorral

front(); back(), data(); at(); []         // elem elérés
begin(); end(); rbegin(); rend();         // iterator
empty(); size(); max_size()               // méretek
swap(); fill(); reverse(); accumulate();  // műveletek
\end{minted}

\paragraph*{Allocator}

Az allokátorobjectumok (\cix{allocator}) a tárolók számára lefoglalt
memóriaterületet kezelik.

\begin{minted}[bgcolor=bg, linenos]{cpp}
#include <memory>

allocator<type> myAlloc;

allocate(<size_t> n);
construct(<pointer> p, <const reference> r);
deallocate(<pointer> p, <size_t> n);
destroy();
\end{minted}

\paragraph*{Vector}

A \cix{vector} \kix{dinamikus tömb}[ökkel] megvalósított soros tároló.
Folytonos adatterületen pointerrel és bejáróval is bejárható.
Automatikusan növekszik és csökken a tárolási mérete. Az elemek könnyen
elérhetők pozíció alapján (állandó idővel). Elemek sorban bejárhatóak
(lineáris idővel). Elemek illeszthetők/törölhetők a végéről (konstans idővel).
Elemek be is illeszthetőek/törölhetőek, de erre más tárolók (\cix{deque},
\cix{list}) jobb időt produkálnak.

\begin{minted}[bgcolor=bg, linenos]{cpp}
#include <vector>

vector<type> myVector1 (count, myVar);
vector<type> myVector2 (myVector1.begin(), myVector1.end());
vector<type> myVector3 (myVector1);

begin(); end(); rbegin(); rend();   // bejárok
[]; at(); front(); back();          // tulajdonságok
size(); resize();                   // méretek (össz)
capacity(), reverse();              // méretek (hátralévő)
push_back(), pop_back();            // módosítók
iterator insert();                  // beszúrás (nem hatékony)

// Bejáras
for(i = myVec.begin(); i != myVec.end(); i++);
for(int i = 0; i < myVec.size(); i++);
for(auto i : myVec);
for(auto& i : myVec);
\end{minted}

\paragraph*{String} A \cix{string} is STL tároló.

\begin{minted}[bgcolor=bg, linenos]{cpp}
#include <string>

at(); []; front(); back(); data();
begin(); end(); rbegin(); rend();
empty(); size(), max_size();
push_back(); copy(); c_str();
find(); find_first(); insert(); replace();
\end{minted}

\paragraph*{Deque}

A \cix{deque} egy kétvégű sor, amely mindkét végén növelhető.
Konstans idő alatt adhatunk hozzá, illetve távolíthatunk el elemet a sor végeiről.
Nem folytonos adatterületen helyezkedik el. (nincs \texttt{\tc{capacity}()} és
\texttt{\tc{reverse}()}) Egydimenziós tömböt tartalmazó listában tárolódik.
Az elemek könnyen elérhetőek pozíció alapján. (állandó idővel)
(\texttt{[], \tc{at}(), \tc{front}(), \tc{back}()}) Elemek sorban bejárhatóak
(lineáris idővel). Lassabb, mint a sor (\cix{queue}).

\begin{minted}[bgcolor=bg, linenos]{cpp}
#include <deque>

deque<type> myDQ1(count, myVar);
deque<type> myDQ2{myVar1, myVar2, ...};
deque<type> myDQ3(myDQ1.begin(), myDQ1.end());
deque<type> myDQ4(myDQ1);

// Konstans idővel elem hozzáadása / törlése
push_front(); pop_front();
push_back(); pop_back();
\end{minted}

\paragraph*{List}

A listák (\cix{list}) esetén nincsen direkt elérés (\texttt{\tc{at}(), []}).
Kétirányban láncolt lista. Gyors beszúrás törlés, sorbarakás összefésülés.
\begin{minted}[bgcolor=bg, linenos]{cpp}
#include <list>

front(); back(); push_front(); push_back(); pop_front(); pop_back();
iterator insert();
sort(); merge(); splice();
\end{minted}

\paragraph*{Forward{}\textunderscore{}list}

Az egy irányban láncolt listák (\cix{forward{}\textunderscore{}list})
esetén nincs \texttt{\tc{push{}\textunderscore{}back}()} és
\texttt{\tc{pop{}\textunderscore{}back}()}, viszont van
\texttt{\tc{insert{}\textunderscore{}after}()}.

\begin{minted}[bgcolor=bg, linenos]{cpp}
#include <forward_list>
\end{minted}

\subsubsection*{Asszociatív tárolók}

Az asszociatív tárolók (\cix{map}, \cix{set}) absztrakt adattípusok.

\paragraph*{Set}

A \cix{set} egyetlen elem sort tartalmaz, ami a kulcs egyben. Nincsen
\blackc{[]} operátor.

\begin{minted}[bgcolor=bg, linenos]{cpp}
#include <set>

type array[] = {var1, var2, var3}
set<type> mySet1(array, array+3)
set<type> mySet2(mySet1.begin(), mySet1.end())
set<type> mySet3(mySet1)

begin(); end(); rbegin(); rend();
::iterator; size(); max_size(); insert();
empty(); erase(); clear(); find();
\end{minted}

\paragraph*{Map}

A \cix{map} kulcs--érték adatpárokat tartalmaz. (a \cix{pair} sablon szerint)
Lehet definiálni az összehasonlítást kulcsra és értékre is. Van \blackc{[]}
operátor (kulcs kell).

\begin{minted}[bgcolor=bg, linenos]{cpp}
#include<map>

map<type1, type2> myMap1;
map<type1, type2> myMap2(myMap1.begin(), myMap1.end());
map<type1, type2> myMap3(myMap1);

begin(); end(); rbegin(); rend();
::iterator; size(); max_size(); insert();
empty(); erase(); clear(); find();
\end{minted}

Léteznek kulcsismétlést megengedő változatok
(\cix{multiset}, \cix{multimap}),
a keresés ebben az esetben lineáris végrehajtási idejű.
Mindegyiknek van rendezetlen változata is.
(\tc{unordered\textunderscore{}xxx})

\subsubsection*{Konténer adapterek}

\paragraph*{Stack}

A verem (\cix{stack}) \tc{FILO} típusú tároló.
Interface minden standardnak, aminek van
\texttt{\tc{back}()}, \texttt{\tc{push\textunderscore{}back}()}
és \texttt{\tc{pop\textunderscore{}back}()} művelete.
(\cix{vector}, \cix{deque}, \cix{list})

\begin{minted}[bgcolor=bg, linenos]{cpp}
 #include <stack>

stack<type, container<type>>;

empty(); push(); pop(); top(); size()
\end{minted}

\paragraph*{Queue}

A sor (\cix{queue}) \tc{FIFO} típusú tároló.
Interface minden standardnak, aminek van
\texttt{\tc{back}()}, \texttt{\tc{push\textunderscore{}back}()}
és \texttt{\tc{pop\textunderscore{}back}()} művelete.
(\cix{vector}, \cix{deque}, \cix{list})

\begin{minted}[bgcolor=bg, linenos]{cpp}
#include <queue>

queue<type, container<type>>;

empty(); push(); pop(); front(); back(); size();
\end{minted}

\paragraph*{Priority{}\textunderscore{}queue}

A prioritásos sor (\cix{priotity\textunderscore{}queue})
interface minden standardnak, aminek van
\texttt{\tc{back}()}, \texttt{\tc{push\textunderscore{}back}()}
és \texttt{\tc{pop\textunderscore{}back}()} művelete.
(\cix{vector}, \cix{deque}, \cix{list})

\begin{minted}[bgcolor=bg, linenos]{cpp}
#include <queue>

queue<type, container<type>>;

empty(); push(); pop(); front(); size();
\end{minted}

\subsection{Bejárók}

A \kix{bejáró}[k] (\cix{iterator}) \cix{pointer} szerű, tároló független
objektumok. Teljes bejárás esetén a \texttt{\tc{begin}()}/\texttt{\tc{rbegin}()}
és \texttt{\tc{end}()}/\texttt{\tc{rend}()} használatos. Egy pozíciót
határoznak meg a tárolóban. Iteratorokra alkalmazható függvénysablonok:

\begin{minted}[bgcolor=bg, linenos]{cpp}
begin(); end();
advance(<int>count);
distance(iterator1, iterator2);
next(iterator, <int>count=1);
prev(iterator, <int>count=1);
\end{minted}

\subsection{Algoritmusok}

\kix{Algoritmus}[ok] a konténerek adatainak kezelésére
(sorbarakás, keresés, szélsőértékek \dots)

\subsection{Függvényobjektumok}

\kix{Függvényobjektum}[ok] megvalósítják a függvényhívás \blackc{()} operátorát
az algoritmusokban. Hívásakor az osztály objektumát adjuk át.

\begin{minted}[bgcolor=bg, linenos]{cpp}
#include <vector>

class Even {
public:
  bool operator() (int i) const {
    return (x % 2) == 0;
  }
};
vector<int> numbers{2, 3, 6, 9};
vector<int>::iterator i = find_if(
  begin(numbers), end(numbers), Even
);
\end{minted}

\cix{C++11} óta általánosított függvényobjektumok is elérhetőek.
Szintén ettől a szabványtól használatosak a névtelen (\kix{lambda})
függvények.
\bgroup
\def\arraystretch{1.2}
\begin{table}[H]
	\centering\begin{tabular}{|c c|}
		\hline
		\blackc{[]\{\};}             &
		nem használ a hívó hatókörében változókat
		\\
		\blackc{[=]\{\};}            &
		a hívó hatókörében definiált változók másolatát látja
		\\
		\blackc{[\&]\{\};}           &
		a hívó hatókörében definiált változók referenciát látja
		\\
		\blackc{[\&,\tc{i}]\{\};}    &
		a hívó hatókörében definiált változók referenciát,
		\tc{i} másolatát látja
		\\
		\blackc{[=,\&{}\tc{i}]\{\};} &
		a hívó hatókörében definiált változók másolatát,
		\tc{i} referenciát látja
		\\ \hline
	\end{tabular}
	\caption{Lambda függvények}
	\label{fig:anonymus-fns}
\end{table}
\egroup

\end{document}
