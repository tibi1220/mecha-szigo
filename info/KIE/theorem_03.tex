\documentclass[../../main.tex]{subfiles}

\begin{document}

\section{Az objektum-orientált programozás alapjai III}

\begin{fulltheorem}
  Az objektum-orientált programozás alapjai. Öröklődés. Egységbe zárás.
  Protected osztálytagok. Kompozíció. Aggregáció. Többszörös öröklődés.
  Barátság.
\end{fulltheorem}

\subsection{Öröklődés}

\begin{table}[H]
  \centering\begin{tabular}{| c || c | c | c |}
    \hline
                        & \bluec{public}      & \orangec{protected} & \blackc{private} \\
    \hline \hline
    \bluec{public}      & \bluec{public}      & \orangec{protected} & \blackc{-}       \\
    \hline
    \orangec{protected} & \orangec{protected} & \orangec{protected} & \blackc{-}       \\
    \hline
    \redc{private}      & \redc{private}      & \redc{private}      & \blackc{-}       \\
    \hline
  \end{tabular}
\end{table}

\begin{itemize}
  \item Az \kix{osztály} más osztályok tulajdonságait/viselkedését
        is magába integrálja.

  \item Módosított viselkedésű osztály az eredeti kód másolata,
        hivatkozása nélkül.

  \item Minden változás automatikusan végigmegy a hierarchián.

  \item Az ős neve a bázis osztály. (\tc{base/parent class})

  \item Az utód neve leszármazott osztály. (\tc{derived/child class})

  \item A konstruktor, destruktor, barát függvények, illetve
        az \tc{operator=} túlterhelése nem öröklődnek,
        öröklődnek viszont az adattagok, tagfüggvények és
        a többi túlterhelt operátor.
\end{itemize}

\subsection{Egységbe zárás}

Az \kix{objektum} egységbe zárja (\cix{encapsulation}) az adatokat és a programokat.
Magába foglalja a külvilág felé mutatott viselkedést. A belső struktúrája,
állapota, adatai és kezelőfüggvényei kívülről nem láthatóak. (\cix{data hiding})

\subsection{Protected osztálytagok}

A \cix{protected} \kix{tagváltozók} \kix{öröklődés} esetén nyilvánosak az
utódosztály számára, de kívülről nem elérhetőek.

\subsection{Kompozíció és Aggregáció}

\kix{Kompozíció} esetén egy meglévő osztályt tagobjektumként használunk fel egy
másik osztályban. (\kix{statikus példány}) \kix{Aggregáció} esetén
\kix{pointer} vagy \kix{referencia} használatos. (\kix{dinamikus példány})
A különbség akkor lép fek, ha a felhasznált osztályt módosítani kezdjük.

\begin{minted}[bgcolor=bg, linenos]{cpp}
class X {};

// Composition
class Y {
  X x;   // 'has a' realtion
};

// Aggregation
class Z {
  X& x;  // only reference
  X* px; // only pointer
};
\end{minted}

\subsection{Többszörös öröklődés}

A \kix{többszörös öröklődés} esetén több ős van.

\subsection{Barátság}

Az alaposztály \kix{barát}ja az utódban az öröklött tagokat éri el.
Az utód barátja az ős public és protected tagjait éri el. 3 fajta
\cix{friend} lehet.

\begin{minted}[bgcolor=bg, linenos]{cpp}
friend type myFunc();           // Külső függvény
friend type MyClass::myFunc();  // Másik osztály publikus tagfüggvénye
friend MyClass;                 // Másik osztály összes függvénye
\end{minted}

\end{document}
