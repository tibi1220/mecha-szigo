\documentclass[../main.tex]{subfiles}

\begin{document}

\section{Az információelmélet alapjai II}

\begin{fulltheorem}
  Az információelmélet alapjai. Forráskódolás. Huffman-kód. Csatornakódolás.
  Hamming -- távolság. Hibajelzés.
\end{fulltheorem}

\subsection{Forráskódolás}

\subsubsection*{Kódszavak átlagos szóhossza}

Az olyan $f: A \rightarrow \mathcal{B}$ kódokat, melyek különböző $A$-beli
szimbólumokhoz más-más hosszúságú kódszavakat rendelnek,
\kix{változó szóhosszúságú} kódoknak nevezzük.
$f(A_i) = B^{(1)} \; B^{(2)} \; \dots \; B^{(\ell_i)}$
$\mathcal{B}$-beli sorozat (kódszó) hossza: ${\ell_i}$.
Egy $f$ kód \kix{átlagos szóhossz}[a] $\ell_i$ \kix{várható érték}[e]:
\[
  L(A) = \sum_{i=1}^n p(A_i) \cdot \ell_i =
  \sum_{i=1}^n p_i \cdot\ell_i
\]

\subsubsection*{Shannon forráskódolási tétele}

Minden $A = \left\{ A_1, A_2, \dots , A_n \right\}$
véges forrásábcéjű forráshoz található olyan $s$ elemű
kódábécével rendelkező $f: A \rightarrow \mathcal{B}$
kód, amely az egyes forrásszimbólumokhoz rendre
$\ell_1, \ell_2, \dots, \ell_n$ szóhosszúságú szavakat
rendel, és \dots
\[
  \frac{H(A)}{\log_2(s)}
  \leq L(A)
  < \frac{H(A)}{\log_2(s)} + 1
\]
Az olyan kódok, melyekre ez teljesül, azok \kix{optimális kód}[ok].

A jól tömöríthető eljárásokra igaz, hogy ha $p_i \geq p_j$,
akkor $\ell_i \leq \ell_j$. Ha az $f$ \kix{bináris kód} \kix{prefix},
akkor \dots
\begin{itemize}
  \item A leggyakoribb forrásábécébeli elemhez
        fog a legrövidebb kódszó tartozni.

  \item A második leggyakoribbhoz eggyel hosszabb kódszó.

  \item $\vdots$

  \item A két legritkábban előforduló betűhöz
        pedig azonosan hosszú kódszó fog tartozni,
        és csak az utolsó karakterben fog e két szó különbözni.
\end{itemize}

\subsection{Huffman-kód}

A \kix{Huffman-kód} a legrövidebb átlagos
szóhosszú \kix{bináris prefix kód}.
\begin{enumerate}
  \item Valószínűségek szerint sorba rendez.

  \item A két legkisebb valószínűséhű szimbólumot
        összevonja. Az összevont szimbólum valószínűsége
        az eredeti két valószínűség összege.

  \item Az első 2 lépés ismételgetése.

  \item A kapott gráf minden csomópontja előtti
        két élt megcímkézi $0$-val és $1$-gyel.

  \item A kódfa gyökerétől elindulba megkeresi
        az adott szimbólumhoz tartozó útvonalat, kiolvassa
        az éleknek megfelelő biteket. A kapott bitsorozatot
        rendeli a szimbólumhoz kódszóként.
\end{enumerate}

\subsection{Csatornakódolás}

$\mathbb{C}^n$ vektortér, $\rvec{c} \in \mathbb{C}^n$
és $\rvec{v} \in \mathbb{C}^n$ vektorok. A csatorna a rá
bocsájtott $\rvec{c} = c^{(1)}, c^{(2)}, \dots, c^{(n)}$
szimbólumsorozatból egy $\rvec{v} = v^{(1)}, v^{(2)},
  \dots, v^{(n)}$ szimbólumsorozatot csinál.

\subsection{Hamming-távolság}

A \kix{Hamming-távolság} $\rvec{c}$ és $\rvec{v}$
eltérésének mérésére definiált távolság. Alatta
azon $i$ pozíciók számát értjük, ahol
$c^{(i)} \neq v^{(i)}$.
Jele: $\dist \left( \rvec{c}, \rvec{v} \right)$.
Teljesülnek az alábbiak:
\[
  \mathrm{d}\left( \rvec{c}, \rvec{v} \right) \geq 0,
  \hspace{2.5em}
  \mathrm{d}\left( \rvec{c}, \rvec{c} \right) = 0
  \hspace{2.5em}
  \mathrm{d}\left( \rvec{c}, \rvec{v} \right) =
  \mathrm{d}\left( \rvec{v}, \rvec{c} \right)
  \hspace{2.5em}
  \mathrm{d}\left( \rvec{c}, \rvec{v} \right) \leq
  \mathrm{d}\left( \rvec{c}, \rvec{w} \right) +
  \mathrm{d}\left( \rvec{w}, \rvec{v} \right)
\]

\subsection{Hibajelzés}

Az \kix{egyszerű hibázás} esetén nem tudjuk, hogy melyik pozíciókban rontott
a csatorna, csak azt, hogy hány darab hiba van.

\kix{törléses hiba} esetén ismerjük a hibázások helyét is, csak azt nem,
hogy mennyire romlott el azokon a helyeken a jel.

Legyen $K$ a lehetséges kódszavak halmaza. Ekkor egy $K$ kód kódtávolsága
a kódszavak közötti \kix{Hamming-távolság} minimuma.

\[
  d_\mathrm{{min}} = \min_{
    \rvec{c} \neq \rvec{c'};\rvec{c}, \rvec{c'} \in K
  } \left\{ \dist \left( \rvec{c}, \rvec{c'} \right) \right\}
\]

\kix{Hibajelzés} lehetséges, ha a $\rvec{c}$ kódszavunkból
keletkezett $\rvec{v}$ nem egy másik érvényes kódszó. Ha $\nu$
a hibák száma, akkor $\nu < d_\mathrm{min}$ hibát lehet mindenképp
jelezni. Hibajelzés után általában megismétlik az üzenetet.

\subsubsection*{Törléses hiba javítása}

Ebben az esetben tudjuk a hibák helyét.
A $\rvec{v}$ hibásan vett vektort abba a kódszóba javítjuk,
amelyik a hibás pozícióktól eltekintve azonos $\rvec{v}$-vel.
Ha több ilyen van, nem tudunk javítani.
Ha a két legközelebbi kódszóból $d_\mathrm{min}$ komponenst
a megfelelő helyről törlünk, akkor azonos maradékot kapunk,
ennél kevesebb elem törlésével sehogy sem kaphatunk azonos
maradékot. Így $\nu \leq d_\mathrm{min} - 1$ törléses hiba javítható.

\subsubsection*{Egyszerű hiba javítása}

Ebben az esetben nem tudjuk a hibák helyét.
A $\rvec{v}$ hibás vett vektort abba a $\rvec{c}$
szóba javítjuk amelyre $\mathrm{d} \left\{
  \rvec{c}, \rvec{v}
  \right\}$ a legkisebb. Ha több ilyen van,
akkor nem tudunk javítani. A javítóság feltétele:
\begin{equation*}
  \mathrm{d} \left( \rvec{c}, \rvec{v} \right)
  < \mathrm{d} \left( \rvec{c'}, \rvec{v} \right)
  \hspace*{1.5em} \rightarrow \hspace*{1.5em}
  \nu \leq \frac{d_\mathrm{min} - 1}{2}
\end{equation*}

\end{document}
