\documentclass[../../main.tex]{subfiles}

\begin{document}

\section{Algoritmusok}

\begin{fulltheorem}
  Algoritmusok. Bejárás, keresés, rendezés.
  Algoritmusok bonyolultsága. Rekurzió.
\end{fulltheorem}

\subsection{Bejárás}

Tömb bejárása \cix{for} ciklussal lehetséges. Bonyolultsága: $\mathcal{O}(n)$.

\subsection{Keresés}

Ha a tömb nincsen rendezve, akkor \kix{szekvenciális keresés}[t] tudunk
alkalmazni, amely egy sima \cix{for} ciklusos keresés, bonyolultsága
ennélfogva: $\mathcal{O}(n)$.

Ha a tömb rendezve van, akkor \kix{bináris keresés}[t] is tudunk alkalmazni.
Ez egy felezéses keresési algoritmus, bonyolultsága: $\mathcal{O}(\log_2n)$.

\subsection{Rendezés}

A legegyszerűbb \kix{rendezés}[i algoritmus] a \kix{buborékrendezés}.
Ez egy dupla \cix{for} ciklusos rendezés, bonyolultsága: $\mathcal{O}(n^2)$.

Ha effektívebben szeretnénk a mintákat rendezni, akkor választhatjuk
például a \kix{quick sort} algoritmust. Ez egy \cix{for} ciklusos felezgetős
rendezés, bonyolultsága: $\mathcal{O}(n \, \log_2n)$.

\subsection{Algoritmusok bonyolultsága}

A $B(n)$ \kix{bonyolultság} algoritmusokat jellemez futási idő, vagy
helyigény szempontjából az adatok számának függvényében. Az eddig tárgyalt
algoritmusokat a futási idő bonyolultságával írtuk le. Mivel egyik
algoritmus sem tárhely igényes, ezért ebből a szempontból minegyik
$\mathcal{O}(1)$-es.

Problémáink lehetnek továbbá:
\begin{itemize}
  \item \kix{polinomidőben megoldható}[ak] $(P)$, vagyis a gyakorlatban
        hatékonyan megoldható problémák, valamint
  \item \kix{polinomidőben verifikálható}[ak] $(NP)$, vagyis a lehetséges megoldások
        polinomidőben előállíthatók nemdeterminisztikus Turing géppel,
        majd determinisztikusan polinomidőben ellenőrizhető az,
        hogy az előállított lehetséges megoldás valóban megoldás-e.
\end{itemize}
Megoldatlan probléma, hogy $N \overset{?}{=} NP$.

Az \kix{utazó ügynök probléma}:
\begin{itemize}
  \item Célja a legkisebb költségű út megtalálása városok között, minden
        várost pontosan egyszer érintve és visszatérve a kiindulási pontba.

  \item A városok közötti költség az Euklideszi távolságon alapul,
        a probléma szimmetrikus, a költségek konstansok.

  \item $NP$ nehéz probléma, azaz ha találnánk egy hatékony algoritmust,
        amely optimális megoldást nyújt polinom számú lépésben, akkor hatékony
        algoritmust tudnánk találni más bonyolult $NP$-beli problémákra.
\end{itemize}

\subsection{Rekurzió}

\kix{Rekurzió} esetén a függvényben, általában a visszatérésénél meghívjuk
saját magát. Számos probléma megoldható rekurzívan, ám ez általában növeli
az algoritmusunk bonyolultságát:

\begin{minted}[bgcolor=bg, linenos]{cpp}
// Faktoriális
int factorial(int n){
  if(n == 0) return 1;
  return n * factorial(n-1);
}

// Az aranymetszés
int golden_ratio(int n){
  if(n == 1 || n == 2) return 1;
  return golden_ratio(n-1) + golden_ratio(n-2);
}

// Hanoi tornyai
void hanoi(int n, char from, char to, char tmp){
  if(n == 1){
    cout << "\nTedd at az 1. lemezt a(z) "
         << from << "rudrol a(z)" << to << " rudra.";
  }
  hanoi(n-1, from, tmp, to);
  cout << "\nTedd at a(z) " << n << " lemezt a(z) "
        << from << "rudrol a(z)" << to << " rudra.";
  hanoi(n-1, tmp, to, from);
}
\end{minted}

\end{document}
