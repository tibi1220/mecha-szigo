\documentclass[../main.tex]{subfiles}

\begin{document}

\section{Az operációs rendszerek alapjai II}

\begin{fulltheorem}
  Az operációs rendszerek alapjai. Holtpont. Holtpont kezelése.
  Holtpont észlelése. Holtpont megelőzés. Bankár algoritmus.
\end{fulltheorem}

\subsection{Holtpont}

A \kix{holtpont} egy rendszernek egy olyan állapota, ahonnan külső beavatkozás
nélkül nem tud elmozdulni. Holtpont akkor fordulhat elő, amikor a folyamatok
egy adott halmazában minden egyes elem leköt néhány erőforrást, és ugyanakkor
várakozik is másokra.

\subsection{Holtpont kezelése}

\begin{itemize}
  \item \kix{Strucc algoritmus}:
        nem teszünk semmit.

  \item \kix{Detektálás} és \kix{feloldás}:
        észrevesszük, ha holtpont alakult ki, és megpróbáljuk feloldani.

  \item \kix{Megelőzés}:
        strukturálisan holtpontmentes rendszert tervezünk.
\end{itemize}

\subsection{Holtpont észlelése}

Tegyük fel, hogy \texttt{4} \texttt{folyamat}unk és \texttt{10}
egység \texttt{erőforrás}unk van. A helyzet a következő:
\begin{table}[H]
  \centering\begin{tabular}{|p{1.5cm} |p{1.5cm} |p{1.5cm} |}
    \hline
                           & \texttt{foglal} & \texttt{kér}
    \\ \hline
    \texttt{\color{red}P1} & \texttt{4}      & \texttt{\color{red}4} \\
    \texttt{P2}            & \texttt{1}      & \texttt{0}            \\
    \texttt{\color{red}P3} & \texttt{3}      & \texttt{\color{red}4} \\
    \texttt{P4}            & \texttt{1}      & \texttt{2}            \\
    \hline
  \end{tabular}
  \caption{Holtponti helyzet}
  \label{table:deadlock}
\end{table}

\kix{Radikális} lépés, ha az összes folyamatot felszámoljuk.
\kix{Kiméletes}, ha megnézzük, van-e menthető folyamat, esetleg
prioritás, vagy éppen mekkora része lett már az adott
folyamat feladatának elvűgezve. Minden esetben biztosítani
kell a folyamatok visszaállíthatóságát.

\subsection{Holtpont megelőzése}

\kix{Biztonságos}[nak] nevezzük azokat a folyamat-erőforrás rendszereket,
amelyekben létezik a folyamatoknak (legalább egy) olyan sorrendje,
amely szerint végrehajtva őket, azok maximális erőforrás igénye is kielégíthető.
Biztonságos állapotban nem lehetséges holtponti állapot kialakulása.
A biztonságos állapotot \kix{bankár algoritmus}[sal] ellenőrizhetjük.
(bankban is ilyen módszert alkalmaznak)

\subsection{Bankár algoritmus}

\begin{enumerate}
  \item adatok mátrixos felírása
  \item az igények és a szabad erőforrások kiszámítása
  \item megnézzük, hogy valamelyik folyamat kielégíthető-e
  \item újraszámítás, folytatás\dots
\end{enumerate}
Ha találunk egy olyan sorrendet,
amelyben a folyamatok erőforrás igénye kielégíthető,
akkor a rendszer \kix{biztonságos állapot}[ban] van.

\end{document}
