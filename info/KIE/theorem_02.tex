\documentclass[../../main.tex]{subfiles}

\begin{document}

\section{Az objektum-orientált programozás alapjai II}

\begin{fulltheorem}
	Az objektum-orientált programozás alapjai. Operátorok túlterhelése.
	C++ IO. new, delete. Osztály hierarchiák.
\end{fulltheorem}

\subsection{Az operátorok túlterhelése}

Szinte minden \kix{operátor} \kix{túlterhelhető}, kivéve: \tc{::}, \tc{.},
\tc{.*}, \tc{?}, \texttt{\tc{sizeof}()}, \texttt{\tc{typeid}()}. Új műveleti
jel nem hozható létre. Egyoperandusú művelet esetén:

\begin{minted}[bgcolor=bg, linenos]{cpp}
int operator++ ()    { /* ... */ }  // i++   --   Postincrementing
int operator++ (int) { /* ... */ }  // ++i   --   Preincrementing
\end{minted}

\subsection{C++ IO}

Az \tc{std} \cix{namespace}-t használjuk, az alábbi fejlécekkel:
\cix{iostream}, \cix{iomanip}.

\begin{minted}[bgcolor=bg, linenos]{cpp}
#include <iostream>
#include <iomanip>
#include <string>

using namespace std;

cin.get();
cout.put('\n');

cout << "Hello World"
string tmp;
cin >> tmp;

cout.flags(ios_base::hex)
  // (no)boolalpha, left, right
  // dec, hex, oct, fixed, scientific
cout.precision(2)

cout << setw(5) << setprecision(2) << 12.345;

// Overloading
friend ostream& operator<< (ostream& os, const type& myVar);
friend istream& operator>> (istream& is, type& myVar);
\end{minted}

\subsection{new és delete operátorok}

A \cix{new} és \cix{delete} kulcsszavak segítségével \kix{dinamikus példány}okat
hozhatunk létre, és törölhetünk. Ezek is túlterhelhetőek:

\begin{minted}[bgcolor=bg, linenos]{cpp}
void* operator new(size_t size){
  // { ... }
  return new type[size];
}

void operator delete (void* p, size_t size){
  // { ... }
  ::delete p;
}
\end{minted}

\subsection{Osztály hierarchiák}

Az újrahasznosítható szoftver alapja. Többszörös \kix{öröklődés}
(\cix{inheritance}) lehetséges.

\end{document}
