\documentclass[../../main.tex]{subfiles}

\begin{document}

\section{A számítógép architektúrák alapjai I}

\begin{fulltheorem}
  A számítógép architektúrák alapjai. Boole függvények. Logikai kapuk.
  Kombinációs logikai hálózatok. Szekvenciális logikai hálózatok. Flip-flop.
\end{fulltheorem}

\subsection{Boole függvények}

Két változós \kix{Boole függvény}[ekből] 16 darab van.
$n$ változósból ${2^2}^n$. \cix{AND}, \cix{OR}, \cix{NOT} függvényekből
az összes Boole függvény előállítható. A \cix{NAND} és a \cix{NOR} önmagukban
képesek az összeset előállítani.

\subsection{Logikai kapuk}

A \kix{digitális áramkör}[ök] esetén az áramkör
bármely pontján mérhető jeleknek csak 2 állapotát
külünbüztetjük meg.
A digitális áramköröket \kix{logikai áramkör}[ökkel]
modellezzük. Leírásukhoz \kw{Boole algebrát} használunk.
A \kix{logikai kapu}[k] a logikai áramkörök építőkockái,
logikai alapműveleteket valósítanak meg. Ezek kombinációjával
további áramköröket tudunk felépíteni.

\begin{figure}[H]
  \centering
  \begin{circuitikz}[american]
    \node [and port](myand) at (0,0) {};
    \node [below=4pt] (andnode) at (myand.south) {\cix{AND}};

    \node [nand port](mynand) at (2.5,0) {};
    \node [below=4pt] (nandnode) at (mynand.south) {\cix{NAND}};

    \node [or port](myor) at (5,0) {};
    \node [below=4pt] (ornode) at (myor.south) {\cix{OR}};

    \node [nor port](mynor) at (7.5,0) {};
    \node [below=4pt] (nornode) at (mynor.south) {\cix{NOR}};

    \node [xor port](myxor) at (10,0) {};
    \node [below=4pt] (xornode) at (myxor.south) {\cix{XOR}};

    \node [not port](mynot) at (-3,0) {};
    \node [below=4pt] (notnode) at (mynot.south) {\cix{NOT}};
  \end{circuitikz}
  \caption{A logikai kapuk}
  \label{fig:logic-gates}
\end{figure}

\begin{table}[H]
  \begin{minipage}[t]{0.166\textwidth}
    \begin{center}
      \begin{tabular}{|c|c|}
        \hline
        $in$ & $out$
        \\ \hline \hline
        0    & 1
        \\ \hline
        1    & 0
        \\ \hline
      \end{tabular}
    \end{center}
  \end{minipage}\hfill
  \begin{minipage}[t]{0.166\textwidth}
    \begin{center}
      \begin{tabular}{|c|c|c|}
        \hline
        $i_1$ & $i_2$ & $o$
        \\ \hline \hline
        0     & 0     & 0
        \\ \hline
        0     & 1     & 0
        \\ \hline
        1     & 0     & 0
        \\ \hline
        1     & 1     & 1
        \\ \hline
      \end{tabular}
    \end{center}
  \end{minipage}\hfill
  \begin{minipage}[t]{0.166\textwidth}
    \begin{center}
      \begin{tabular}{|c|c|c|}
        \hline
        $i_1$ & $i_2$ & $o$
        \\ \hline \hline
        0     & 0     & 1
        \\ \hline
        0     & 1     & 1
        \\ \hline
        1     & 0     & 1
        \\ \hline
        1     & 1     & 0
        \\ \hline
      \end{tabular}
    \end{center}
  \end{minipage}\hfill
  \begin{minipage}[t]{0.166\textwidth}
    \begin{center}
      \begin{tabular}{|c|c|c|}
        \hline
        $i_1$ & $i_2$ & $o$
        \\ \hline \hline
        0     & 0     & 0
        \\ \hline
        0     & 1     & 1
        \\ \hline
        1     & 0     & 1
        \\ \hline
        1     & 1     & 1
        \\ \hline
      \end{tabular}
    \end{center}
  \end{minipage}\hfill
  \begin{minipage}[t]{0.166\textwidth}
    \begin{center}
      \begin{tabular}{|c|c|c|}
        \hline
        $i_1$ & $i_2$ & $o$
        \\ \hline \hline
        0     & 0     & 1
        \\ \hline
        0     & 1     & 0
        \\ \hline
        1     & 0     & 0
        \\ \hline
        1     & 1     & 0
        \\ \hline
      \end{tabular}
    \end{center}
  \end{minipage}\hfill
  \begin{minipage}[t]{0.166\textwidth}
    \begin{center}
      \begin{tabular}{|c|c|c|}
        \hline
        $i_1$ & $i_2$ & $o$
        \\ \hline \hline
        0     & 0     & 0
        \\ \hline
        0     & 1     & 1
        \\ \hline
        1     & 0     & 1
        \\ \hline
        1     & 1     & 0
        \\ \hline
      \end{tabular}
    \end{center}
  \end{minipage}\hfill
  \caption{A logikai kapuk igazságtáblája}
  \label{table:truth-tables}
\end{table}

\subsection{Kombinációs logikai hálózatok}

\kix{Kombinációs hálózatok} esetén a kimeneti jelek értékei csak a bemeneti
jelek pillanatnyi értékétől függenek. A kimenetek egy-egy függvénykapcsolattal
írhatóak le.

\subsubsection{Félösszeadó}

Egy \cix{XOR} és \cix{AND} kapuval megvalósítható. Feladata 2 bit összeadása.
\[
  S = \overline{A}B + A\overline{B} = A \oplus B
  \qquad \text{és} \qquad
  C = AB
\]

\hfill
\begin{minipage}[b]{0.45\textwidth}
  \begin{figure}[H]
    \centering
    \begin{circuitikz}[american]
      \draw (0, 4)node[xor port] (xorone){}
      (0, 2)node[and port] (and){}
      (xorone.in 1) node[left=1cm](a) {$A$}
      (xorone.in 2) node[left=1cm](b) {$B$}
      (xorone.out) node[right=.5cm](s) {$S$}
      (and.out) node[right=.5cm](c) {$C$}

      (a.east) to[short,-*] (xorone.in 1) |- (and.in 1)
      (b.east) to[short,-*] ($(b.east)!.5!(xorone.in 2)$) coordinate (branch) -- (xorone.in 2)
      (c.west) to (and.out)
      (s.west) to (xorone.out)
      (branch) |- (and.in 2);
    \end{circuitikz}
    \caption{Félösszeadó megvalósítása}
    \label{fig:half-adder}
  \end{figure}
\end{minipage}\hfill
\begin{minipage}[b]{0.5\textwidth}
  \begin{table}[H]
    \centering
    \begin{tabular}{| c | c || c | c |}
      \hline
      $A$ & $B$ & $S$ & $C$
      \\ \hline \hline
      0   & 0   & 0   & 0
      \\ \hline
      0   & 1   & 1   & 0
      \\ \hline
      1   & 0   & 1   & 0
      \\ \hline
      1   & 1   & 0   & 1
      \\ \hline
    \end{tabular}
    \caption{Félösszeadó igazsátáblája}
    \label{table:half-adder}
  \end{table}
\end{minipage}
\hfill

\subsubsection{Teljes összeadó}

Feladata két bit és az előző helyi értékből származó maradék összeadása.

Bemenetei: $A$, $B$, $C_\text{in}$

Kimenetei: $S$, $C_\text{out}$

\[
  S = A \oplus B \oplus C_\text{in}
  \qquad \text{és} \qquad
  C_\text{out} = AB + AC_\text{in} + BC_\text{in}
\]

\begin{figure}[htpb]
  \centering
  \begin{circuitikz}[american,scale=.8]
    \ctikzset{
      tripoles/american xor port/height=1.2,
      tripoles/american or port/height=1.2,
      tripoles/american and port/height=.70,
    }
    \node[xor port, number inputs=3] (xor) at (0,0) {};

    \node[and port] (or1) at (0,-1.9) {};
    \node[and port] (or2) at (0,-3.25) {};
    \node[and port] (or3) at (0,-4.6) {};

    \node[or port, number inputs=3] (and) at (3, -3.25) {};

    \draw[]
    (or1.out) -- ++(0.5,0) |- (and.in 1)
    (or2.out) -- ++(0.5,0) |- (and.in 2)
    (or3.out) -- ++(0.5,0) |- (and.in 3)

    (xor.out) to[short, -o] ++(4,0) node[above left] {$S$}
    (and.out) to[short, -o] ++(1,0) node[above left] {$C_\text{out}$}

    (xor.in 1) to[short, -o] ++(-2,0) node[left] {$A$}
    (xor.in 2) to[short, -o] ++(-2,0) node[left] {$B$}
    (xor.in 3) to[short, -o] ++(-2,0) node[left] {$C_\text{in}$}

    (or1.in 1) to ++(-.5,0) coordinate(H) to[short, -*] (H|-xor.in 1)
    (or2.in 1) to ++(-.5,0) coordinate(H) to[short, -*] (H|-xor.in 1)

    (or1.in 2) to ++(-1,0) coordinate(H) to[short, -*] (H|-xor.in 2)
    (or3.in 1) to ++(-1,0) coordinate(H) to[short, -*] (H|-xor.in 2)

    (or2.in 2) to ++(-1.5,0) coordinate(H) to[short, -*] (H|-xor.in 3)
    (or3.in 2) to ++(-1.5,0) coordinate(H) to[short, -*] (H|-xor.in 3)
    ;
  \end{circuitikz}
  \caption{Teljes összeadó megvalósítása}
  \label{fig:full-adder}
\end{figure}

\subsubsection*{Multiplexer}

\hfill
\begin{minipage}[b]{0.45\textwidth}
  \begin{figure}[H]
    \centering
    \begin{circuitikz}[american]
      \tikzset{mux 4by2/.style={muxdemux,
            muxdemux def={Lh=5, NL=4, Rh=3,
                NB=2, w=2, square pins=1}}}

      \draw (0,0) node [mux 4by2] (mux1) {\small\cix{MUX}}

      (mux1.lpin 1) node [left=6](in1) {$D_0$}
      (mux1.lpin 2) node [left=6](in2) {$D_1$}
      (mux1.lpin 3) node [left=6](in3) {$D_2$}
      (mux1.lpin 4) node [left=6](in4) {$D_3$}

      (mux1.rpin 1) node [right](y) {$Y$}

      (mux1.bpin 1) node [below](a) {$A$}
      (mux1.bpin 2) node [below](b) {$B$}

      (in1.east) to (mux1.lpin 1)
      (in2.east) to (mux1.lpin 2)
      (in3.east) to (mux1.lpin 3)
      (in4.east) to (mux1.lpin 4)

      (a.north) to (mux1.bpin 1)
      (b.north) to (mux1.bpin 2)

      (y.west) to (mux1.rpin 1);
    \end{circuitikz}
    \caption{4 az 1-hez multiplexer}
    \label{fig:mux}
  \end{figure}
\end{minipage}\hfill
\begin{minipage}[b]{0.5\textwidth}
  \begin{table}[H]
    \centering
    \begin{tabular}{|c|c|c|}
      \hline
      $A$ & $B$ & $Y$
      \\ \hline \hline
      0   & 0   & $D_0$
      \\ \hline
      0   & 1   & $D_1$
      \\ \hline
      1   & 0   & $D_2$
      \\ \hline
      1   & 1   & $D_3$
      \\ \hline
    \end{tabular}
    \caption{Multiplexer igazságtáblája}
    \label{table:mux}
  \end{table}
\end{minipage}
\hfill

Feladata több bemenő jel közül az egyik kiválasztása, valamint
lehet még párhuzamos–soros adatkonverter.

Bemenetek: $2^n$ db

Vezérlőbemenetek: $n$ db

Kimenetek: $1$ db

\subsubsection*{Demultiplexer}

\hfill
\begin{minipage}[b]{0.40\textwidth}
  \begin{figure}[H]
    \centering
    \begin{circuitikz}[american]
      \tikzset{demux 2by4/.style={muxdemux,
            muxdemux def={Lh=3, NL=1, Rh=5,
                NR=4, NB=2, w=2, square pins=1}}}

      \draw (0,0) node [demux 2by4] (demux1) {\small\cix{DEMUX}}

      (demux1.rpin 1) node [right=6](out1) {$D_0$}
      (demux1.rpin 2) node [right=6](out2) {$D_1$}
      (demux1.rpin 3) node [right=6](out3) {$D_2$}
      (demux1.rpin 4) node [right=6](out4) {$D_3$}

      (demux1.lpin 1) node [left](y) {$Y$}

      (demux1.bpin 1) node [below](outa) {$A$}
      (demux1.bpin 2) node [below](outb) {$B$}

      (out1.west) to (demux1.rpin 1)
      (out2.west) to (demux1.rpin 2)
      (out3.west) to (demux1.rpin 3)
      (out4.west) to (demux1.rpin 4)

      (outa.north) to (demux1.bpin 1)
      (outb.north) to (demux1.bpin 2)

      (y.east) to (demux1.lpin 1);
    \end{circuitikz}
    \caption{1 az 4-hez demultiplexer}
    \label{fig:demux}
  \end{figure}
\end{minipage}\hfill
\begin{minipage}[b]{0.55\textwidth}
  \begin{table}[H]
    \centering
    \begin{tabular}{|c|c||c|c|c|c|}
      \hline
      $A$ & $B$ & $D_1$ & $D_2$ & $D_3$ & $D_4$
      \\ \hline \hline
      0   & 0   & $Y$   & 0     & 0     & 0
      \\ \hline
      0   & 1   & 0     & $Y$   & 0     & 0
      \\ \hline
      1   & 0   & 0     & 0     & $Y$   & 0
      \\ \hline
      1   & 1   & 0     & 0     & 0     & $Y$
      \\ \hline
    \end{tabular}
    \caption{Demultiplexer igazságtáblája}
    \label{table:demux}
  \end{table}
\end{minipage}\hfill

Feladata egy jel kapcsolása választható kimenetre, valamint
lehet még párhuzamos–soros adatkonverter.

Bemenetek: $1$ db

Vezérlőbemenetek: $n$ db

Kimenetek: $2^n$ db

\subsubsection*{Dekódoló}

\hfill
\begin{minipage}[b]{0.45\textwidth}
  \begin{figure}[H]
    \centering
    \begin{circuitikz}[american]
      \tikzset{decoder/.style={muxdemux,
            muxdemux def={Lh=7, NL=3, Rh=7,
                NR=8, NB=0, w=3.5, square pins=1}}}

      \draw (0,0) node [decoder] (decoder1) {\small\cix{DECODER}}

      (decoder1.lpin 1) node [left](in1) {$A$}
      (decoder1.lpin 2) node [left](in2) {$B$}
      (decoder1.lpin 3) node [left](in3) {$C$}

      (decoder1.rpin 1) node [right](out1) {\footnotesize $D_0$}
      (decoder1.rpin 2) node [right](out2) {\footnotesize $D_1$}
      (decoder1.rpin 3) node [right](out3) {\footnotesize $D_2$}
      (decoder1.rpin 4) node [right](out4) {\footnotesize $D_3$}
      (decoder1.rpin 5) node [right](out5) {\footnotesize $D_4$}
      (decoder1.rpin 6) node [right](out6) {\footnotesize $D_5$}
      (decoder1.rpin 7) node [right](out7) {\footnotesize $D_6$}
      (decoder1.rpin 8) node [right](out8) {\footnotesize $D_7$};
    \end{circuitikz}
    \caption{3 bites Dekódoló}
    \label{fig:decoder}
  \end{figure}
\end{minipage}\hfill
\begin{minipage}[b]{0.5\textwidth}
  \begin{table}[H]
    \centering
    \begin{tabular}{|c|c|c|c|}
      \hline
      $A$ & $B$ & $C$ & $Y$
      \\ \hline \hline
      0   & 0   & 0   & $D_0$
      \\ \hline
      0   & 0   & 1   & $D_1$
      \\ \hline
      0   & 1   & 0   & $D_2$
      \\ \hline
      0   & 1   & 1   & $D_3$
      \\ \hline
      1   & 0   & 0   & $D_4$
      \\ \hline
      1   & 0   & 1   & $D_5$
      \\ \hline
      1   & 1   & 0   & $D_6$
      \\ \hline
      1   & 1   & 1   & $D_7$
      \\ \hline
    \end{tabular}
    \caption{Dekódoló igazságtáblája}
    \label{table:decoder}
  \end{table}
\end{minipage}\hfill

Feladata cím dekódolása.

Bemenet: $n$ bites szám

Kimenet: $2^n$-ből választ ki $1$-et

\subsection{Szekvenciális logikai hálózatok}

\kix{Szekvenciális hálózatok} esetén a kimenet nemcsak a bemeneti
jelkombinációtól, hanem a hálózat állapotától is függ.
(azaz a a hálózatra megelőzően ható jelkombinációktól)
Léteznek \kix{szinkron} (órajel) és \kix{aszinkron} sorrendi hálózatok.

\subsection{Flip-flop}

A \kix{flip-flop}[ok] elemi szekvenciális hálózatok, melyeknek két stabil
állapotuk van, amely a kimenettel egyezik meg. \kix{Billenő elemek}[nek]
is szokták őket nevezni. \kix{SR flipflop} megvalósítása \cix{NOR} kapukkal:

\hfill
\begin{minipage}[b]{0.3\textwidth}
  \begin{figure}[H]
    \centering
    \begin{circuitikz}[american]
      \ctikzset{american or shape=roundy}

      \node[nor port] (nor1) at (0, 1.2) {};
      \node[nor port] (nor2) at (0,-1.2) {};

      \draw (nor1.in 2) -| ++ (-0.3,-0.5) -- ++(2.15,-0.8) coordinate(a)|- (nor2.out);
      \draw (nor2.in 1) -| ++ (-0.3, 0.5) -- ++(2.15, 0.8) |- (nor1.out);

      \draw (nor1.out -| a) to [short, *-] ++(0.45,0) node[right]{$Q$};
      \draw (nor2.out -| a) to [short, *-] ++(0.45,0) node[right]{$\overline{Q}$};

      \draw (nor1.in 1) -- ++ (-.5, 0) node[left]{$S$};
      \draw (nor2.in 2) -- ++ (-.5, 0) node[left]{$R$};
    \end{circuitikz}
    \caption{SR flip-flop}
    \label{fig:SR}
  \end{figure}
\end{minipage}\hfill
\begin{minipage}[b]{0.5\textwidth}
  \begin{table}[H]
    \centering
    \begin{tabular}{|c|c|c|}
      \hline
      $S$ & $R$ & $Q$
      \\ \hline \hline
      0   & 0   & $Q_0$
      \\ \hline
      0   & 1   & 0
      \\ \hline
      1   & 0   & 1
      \\ \hline
      1   & 1   & ?
      \\ \hline
    \end{tabular}
    \caption{SR flip-flop igazságtáblája}
    \label{table:SR}
  \end{table}
\end{minipage}\hfill

\end{document}
