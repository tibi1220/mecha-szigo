\documentclass[../main.tex]{subfiles}

\begin{document}

\section{Az információelmélet alapjai I}

\begin{fulltheorem}
  Az információelmélet alapjai. Shannon hírközlési modellje. Az információ.
  Az entrópia. Forráskódok. Egyértelműen dekódolható kódok. Prefix kód.
\end{fulltheorem}

\subsection{Shannon hírközlési modellje}

A hírközlés során egy üzenetet juttatunk el
egy tér- és időbeli pontból egy másikba.
\begin{figure}[htb]
  \centering
  \begin{tikzpicture}[scale=.7]
    \draw[thick, fill=yellow!20] (0 ,0) rectangle ++ (3,1)
    node[midway]{\texttt{forrás/adó}};

    \draw[thick, fill=orange!20] (4 ,0) rectangle ++ (3,1)
    node[midway]{\texttt{kódoló}};

    \draw[thick, fill=bgreen!20] (8 ,0) rectangle ++ (3,1)
    node[midway]{\texttt{csatorna}};

    \draw[thick, fill=orange!20] (12,0) rectangle ++ (3,1)
    node[midway]{\texttt{dekódoló}};

    \draw[thick, fill=yellow!20] (16,0) rectangle ++ (3,1)
    node[midway]{\texttt{vevő/nyelő}};

    \draw[thick, ->] (3,.5) -- ++(1,0) node[above, pos=.5] {$b$};
    \draw[thick, ->] (7,.5) -- ++(1,0) node[above, pos=.5] {$c$};
    \draw[thick, ->] (11,.5) -- ++(1,0) node[above, pos=.5] {$y$};
    \draw[thick, ->] (15,.5) -- ++(1,0) node[above, pos=.5] {$z$};

    \begin{scope}[xshift=0cm, yshift=-6.833cm]
      \draw[thick, fill=yellow!10, rounded corners=10]
      (0 ,1) rectangle ++ (10, 5);

      \draw[thick, fill=teal!20] (0.4 ,3.5) rectangle ++ (2.7, 2)
      node[midway, text width=2.2cm, align=center]
        {\texttt{tényleges forrás}};

      \draw[thick, fill=teal!20] (4.4 ,3.5) rectangle ++ (4.2, 2)
      node[midway, text width=4cm, align=center]
        {\texttt{mintavételezés kvantálás forráskódolás}};

      \draw[thick, ->] (3.1,4.5) -- ++(1.3,0) node[above, pos=.5] {$a$};
      \draw[thick, ->] (8.6,4.5) -- ++(1.1,0) node[above, pos=.5] {$b$};

      \draw node[text width=2.5cm, align=center] at(1.75, 2.25)
      {lehet folytonos jel};

      \draw node[text width=5cm, align=center] at(6.35, 2.25)
      {a forrás jelét diszkrét \\ jellé alakítja, tömöríti};
    \end{scope}

    \begin{scope}[xshift=11cm, yshift=-6.833cm]
      \draw[thick, fill=orange!10, rounded corners=10]
      (0 ,1) rectangle ++ (8, 5)
      node[midway, text width=5cm, align=justify]
        {a \texttt{csatornakódolás} (hibajavító)
          lehetővé teszi a zajos csatornán való biztonságos(abb)
          üzenetátvitelt, a keletkező hibák jelzését és javítását};
    \end{scope}

    \begin{scope}[xshift=0cm, yshift=-11.666cm]
      \draw[thick, fill=bgreen!10, rounded corners=10]
      (0 ,0) rectangle ++ (19, 5);

      \draw[thick, fill=teal!20] (2.5 ,3) rectangle ++ (3.5, 1.5)
      node[midway, text width=2.2cm, align=center]
        {\texttt{modulátor}};

      \draw[thick, fill=teal!20] (7.75 ,3) rectangle ++ (3.5, 1.5)
      node[midway, text width=4cm, align=center]
        {\texttt{csatorna}};

      \draw[thick, fill=teal!20] (13 ,3) rectangle ++ (3.8, 1.5)
      node[midway, text width=3cm, align=center]
        {\texttt{demodulátor, döntő}};

      \draw[thick, ->] (.5, 3.75) -- ++(2   , 0) node[above, pos=.5] {$c$};
      \draw[thick, ->] (6 , 3.75) -- ++(1.75, 0) node[above, pos=.5] {};
      \draw[thick, ->] (11.25, 3.75) -- ++(1.75, 0) node[above, pos=.5] {$x$};
      \draw[thick, ->] (16.8, 3.75) -- ++(1.7, 0) node[above, pos=.5] {$y$};

      \draw[thick, ->] (9.5, 2) -- ++(0, 1)
      node[above, pos=.5, rotate=90] {zaj}
      node[below, pos=0, yshift=-.15cm]{torzul a jel};

      \draw node[text width=4cm, align=center] at(4.25, 1.5)
      {átalakítja a kódolt üzenetet a csatornán átvihető jellé};

      \draw node[text width=4cm, align=center] at(14.75, 1.5)
      {eldönti, hogy a lehetséges leadott
        jelalakok közül melyiket adhatták};
    \end{scope}

    \begin{scope}[xshift=0cm, yshift=-17.5cm]
      \draw[thick, fill=orange!10, rounded corners=10]
      (0 ,0) rectangle ++ (6, 5)
      node[midway, text width=3.5cm, align=justify]
        {a \texttt{dekódoló} kijavítja, és / vagy jelzi a vett jelek hibáit.
          Elvégzi a \texttt{csatornakódolás} inverz műveletét.};
    \end{scope}

    \begin{scope}[xshift=7cm, yshift=-17.5cm]
      \draw[thick, fill=yellow!10, rounded corners=10]
      (0 ,0) rectangle ++ (12, 5);

      \draw[thick, fill=teal!20] (2.5 ,2.5) rectangle ++ (4, 2)
      node[midway, text width=3cm, align=center]
        {\texttt{forráskódolás inverze}};

      \draw[thick, fill=teal!20] (8.5 ,3) rectangle ++ (2, 1)
      node[midway, text width=4cm, align=center]
        {\texttt{vevő}};

      \draw[thick, ->] (.5,3.5) -- ++(2,0) node[above, pos=.5] {$z$};
      \draw[thick, ->] (6.5,3.5) -- ++(2,0) node[above, pos=.5] {$a'$};

      \draw node[text width=4cm, align=center] at(4.5, 1.5)
      {a helyreállított üzenetet \texttt{kitömöríti}};

      \draw node[text width=2.5cm, align=center] at(9.5, 1.5)
      {\texttt{értelmezi} az üzenetet};
    \end{scope}
  \end{tikzpicture}
  \caption{Shannon modellje}
  \label{fig:shannon}
\end{figure}

\subsection{Az információ}

Az \kix{információ} valamely véges számú, előre ismert esemény közül annak a
megnevezése, hogy melyik következett be.
Értéke azonos azzal a bizonytalansággal, melyet megszűntet.

\subsubsection*{Hartley}

$m$ számú, azonos valúszínűségű esemény közül
egy megnevezésével nyert információ:
\[
  I = \log_2(m)
\]

\subsubsection*{Shannon}

Shannon szerint minél váratlanabb az esemény, bekövetkezése
annál több információt jelent. Legyen
$A = \left\{ A_1, A_2, \dots, A_m \right\}$,
$A_i$ esemény valószínűsége $p_i$.
Az $A_i$ esemény megnevezésével nyert információ ekkor:
\[
  I \left( A_i \right) = -\log_2 \left( p_i \right)
\]

\subsection{Az entrópia}

Az \kix{entrópia} az információ várható értéke.
\[
  H(p_1, p_2, \dots, p_m) =
  \left\langle I(A) \right\rangle =
  \sum_{i=1}^m p_i \cdot I \left( A_i \right) =
  - \sum_{i=1}^m p_i \cdot \log_2(p_i)
\]
Az entrópia tulajdonképpen annak a kijelentésnek az információtartalma,
hogy az $m$ db egymást kizáró esemény közül az egyik bekövetkezett.

\subsection{Forráskódolás}

\begin{itemize}
  \item a \kix{forrás} kimenetén véges sok
        elemből álló $A = \left\{ A_1, A_2, \dots, A_m \right\}$ halmaz
        elemei jelenhetnek meg

  \item \kix{forrásábcé} – maga az $A$ halmaz

  \item \kix{üzenet} – az $A$ elemeiből képzett véges
        $A^{(1)} \; A^{(2)} \; \dots \; A^{(m)}$ sorozatok

  \item $\mathcal{A}$ – a lehetséges üzenetek halmaza

  \item a kódolt üzenetek egy
        $B = \left\{ B_1, B_2, \dots, B_s \right\}$
        szintén véges halmaz elemeiből épülnek fel

  \item \kix{kódábécé} – maga a $B$ halmaz

  \item \kix{kódszavak} – a $B$ elemeiből képzett véges
        $B^{(1)} \; B^{(2)} \; \dots \; B^{(s)}$ sorozatok

  \item $\mathcal{B}$ – a lehetséges kódszavak halmaza

  \item az $f: A \rightarrow \mathcal{B}$ illetve
        $F: \mathcal{A} \rightarrow \mathcal{B}$ függvényeket
        \kix{forráskód}oknak nevezzük. Az $f$ leképezés
        a forrás egy-egy szimbólumához rendel egy-egy szót.
\end{itemize}

\subsection{Egyértelműen dekódolható kódok}

Egy $f$ forráskód \kix{egyértelműen dekódolható},
ha minden egyes $B$-beli sorozatot csak egyféle
$A$-beli sorozatból állít elő, azaz a neki
megfelelő $F$ invertálható. Nem elég, hogy $f$
invertálható legyen.
\begin{itemize}
  \item $! \; A = \left\{a, b, c\right\}$, $B = \left\{ 0, 1\right\}$
        és $f(a) = 0$, $f(b) = 1$, $f(c) = 01$. Ekkor $f$ invertálható,
        de a 01 kódszót dekódolhatjuk $f(a) f(b) = 01$ ezerint $ab$-nek,
        vagy $f(c)=01$ szerint $c$-nek is.
\end{itemize}
Az \kix{állandó kódszóhosszú}
kódok egyértelműen dekódolhatóak, megfejthetőek,
de nem elég gazdaságosak.

\subsection{Prefix kód}

Az $f$ kód \kix{prefix}, ha a lehetséges kódszavak közül
egyik sem folytatása a másiknak, vagyis bármely kódszó
végéből bármekkora szegmenst levágva nem kapunk egy másik
kódszót. Prefix kód \kix{egyértelműen dekódolható}.
\begin{itemize}
  \item $! \; A = \left\{a, b, c\right\}$, $B = \left\{ 0, 1\right\}$
        és $f(a) = 0$, $f(b) = 10$, $f(c) = 110$. Ha az $abccab$ üzenetet
        kódoljuk, akkor a $010110110010$ kódsorozatot kapjuk. A kódból
        az üzenet visszafejtése nagyon egyszerű a prefix tulajdonság miatt.

  \item  $! \; A = \left\{a, b, c, d\right\}$, $B = \left\{ 0, 1\right\}$
        és $f(a) = 0$, $f(b) = 01$, $f(c) = 011$, $f(d)=0111$.
        Ez a kód nem prefix, de egyértelműen dekódolható, hiszem a $0$
        karakter egy új kódszó kezdetét jelzi.
\end{itemize}

\end{document}
