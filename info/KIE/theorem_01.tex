\documentclass[../../main.tex]{subfiles}

\begin{document}

\section{Az objektum-orientált programozás alapjai I}

\begin{fulltheorem}
	Az objektum-orientált programozás alapjai. Az objektum fogalma. Az osztály
	fogalma. Struktúrák. Tagfüggvények. Konstruktor. Destruktor. Statikus tagok.
	Friend függvények.
\end{fulltheorem}

\subsection{Az objektum fogalma}

Az \kix{objektum} (\cix{object}) egy rendszer egyedileg azonosítható szereplője
adatokkal és működéssel.

\subsection{Az osztály fogalma}

Az \kix{osztály} (\cix{class}) megegyező szerkezetű, hasonló viselkedésű
egyedek mintája. Felhasználó által definiált adattípus. Az osztályhoz tartozó
\kix{példány} (\cix{instance}) az objektum. Adatszerkezetekből és
tagfüggvényekből áll, melyek alaphelyzetben nem láthatóak. (\cix{data hiding})

\begin{minted}[bgcolor=bg, linenos]{cpp}
class MyClass {
  // private:
  int hidden_variable;
  int hidden_function() { return 42; }

public:
  int public_variable;
  int public_function() { return 42; }
};
\end{minted}

\subsection{Struktúrák}

A \kix{struktúra} (\cix{struct}) adattagjai az osztályokkal ellentétben
alaphelyzetben láthatóak.

\begin{minted}[bgcolor=bg, linenos]{cpp}
struct MyStruct {
  // public:
  int public_variable;
  int public_function() { return 42; }
};
\end{minted}

\subsection{Tagfüggvények}
Az \kix{adattípus}sal kapcsolatban álló (az \kix{adattag}okon működő)
\kix{tagfüggvény}ek lehetnek belül, vagy kívül definiáltak. Lehetnek továbbá
\cix{inline} függvények. (Hívás helyén lesznek lefordítva.)
\begin{minted}[bgcolor=bg, linenos]{cpp}
class MyClass {
public:
  int outer();
  int inner() { return 42; }
  int inline_outer();
  inline int inline_inner() { return 42; }
};

int MyClass::outer() { return 42; }
inline int MyClass::inline_outer() { return 42; }
\end{minted}

\subsection{Konstruktor}

A \kix{konstruktor} függvény az objektumok előkészítésére, esetleg
inicializálása. Ugyanaz a neve mint az osztálynak, nincs típusa, sem
visszatérési értéke. Létrehozás után nem hívható. Ha nem adjuk meg akkor
is létezik alapértelmezett. (\cix{default}) Másoló konstruktor megadása
konstans referenciával lehetséges.

\subsection{Destruktor}

Az objektumok megszűnésekor automatikusan meghívódó függvény.
Nincsenek paraméterei. Nem terhelhető túl. Statikus \cix{instance} esetén a
\cix{scope}-on kívül, dinamikus esetben a \cix{delete} kulcsszó esetén
hívódik meg.

\begin{minted}[bgcolor=bg, linenos]{cpp}
class MyClass {
  const x;
public:
  MyClass() : x(42) {}
  MyClass(int x) : x(x) {}
  MyClass(int x, int y) : x(x) { this->y = y; }
  ~MyClass() {}
};

MyClass* dinamic_instance = new MyClass();
MyClass static_instance();
delete dinamic_instance;
\end{minted}

\subsection{Statikus tagok}

A \kix{statikus} (\cix{static}) tagok nem az objektumhoz, hanem az osztályhoz
vannak kötve. Az osztályban nem definiálható, \kix{névtér} (\cix{namespace})
szintjén kell ezt megtenni.

\begin{minted}[bgcolor=bg, linenos]{cpp}
class MyClass {
public:
  static int x;
};

int MyClass::x = 42;
\end{minted}

\subsection{Friend függvények}

A \cix{friend} függvények segítségével az osztály rejtett tagjai is elérhetőek.
Lehet külső függvény, egy másik osztály függvénye, vagy akár egy mésik osztály
is.

\begin{minted}[bgcolor=bg, linenos]{cpp}
class MyClass {
  int x;

public:
  friend int func(MyClass &r);
  friend int OtherClass::class_func(MyClass &r);
  friend class OtherClass;
};

class OtherClass {
public:
  int class_func(MyClass &r) { return r.x; }
};

int func(MyClass &r) { return r.x; }
\end{minted}

\end{document}
