\documentclass[../main.tex]{subfiles}

\begin{document}

\section{A számítógép architektúrák alapjai II}

\begin{fulltheorem}
  A számítógép architektúrák alapjai. A számítógép felépítése. Memóriák.
  CPU részei. Utasítás ciklus. Szubrutinhívás. Interrupt.
  Közvetlen memória hozzáférés.
\end{fulltheorem}

\subsection{A számítógép felépítése}

A \kix{digitális számítógép} olyan gép, amely a neki címzett utasítások
alapján problémákat old meg. Az utasítássorozatot, amely leírja, hogy hogyan
oldjunk meg egy feladatot, programnak nevezünk.

\bgroup
\def\arraystretch{1.2}
\begin{table}[H]
  \centering\begin{tabular}{| m{2cm}  >{\centering\arraybackslash}m{10cm} |}
    \hline
    \kix{Hardver}  &
    elektromos áramkörök, mechanikus berendezések,
    kábelek, csatlakozók, perifériák, önmagában nem működőképesek
    \\ \hline
    \kix{Szoftver} &
    számítógépet működőképessé tevő programok és azok dokumentációi
    \\ \hline
    \kix{Firmware} &
    célprogram, mikrokóddal írt, készülék specifikus,
    hardverbe ágyazott szoftver, gyakran \kix{Flash ROM}
    \\ \hline
  \end{tabular}
  \caption{A számítógép részei}
  \label{fig:comp-parts}
\end{table}
\egroup

\subsubsection*{Gépi, nyelvi szintek}

\begin{enumerate}
  \setcounter{enumi}{-1}
  \item digitális logikai szint
        \begin{itemize}
          \item kapuk (\cix{gate})
        \end{itemize}

  \item mikroarchitektúra  szintje
        \begin{itemize}
          \item értelmezi a második szintet
          \item \cix{ALU}, regiszterek
        \end{itemize}

  \item gépi utasítás szintje (elektronikus áramkörök)
        \begin{itemize}
          \item itt dől el a kompatibilitás kérdése
        \end{itemize}

  \item operációs rendszer gépi szintje
        \begin{itemize}
          \item általában értelmezés
          \item az utasításait az oprendszer,
                vagy közvetlen a 2. szint hajtja végre
        \end{itemize}

  \item assembly nyelvi szint (assembler)
        \begin{itemize}
          \item szimbolikus leírás
        \end{itemize}

  \item probléma orientált nyelvi szint (fordító program)
        \begin{itemize}
          \item ezek tényleges nyelvek (\cix{C}, \cix{C++})
        \end{itemize}
\end{enumerate}

\subsection*{A Neumann elvű számítógép felépítése}

\begin{itemize}
  \item A \kix{CPU}
        általános vezérlő, műveletvégző, adat-mozgató egység,
        végrehajtja a futó programok utasításait.

  \item A \kix{memória}
        a futó programok kódját, adatait tartalmazza.

  \item A \kix{háttértárolók}
        lehet mágneslemez, merevlemez, optikai tároló,
        szalagos tároló, félvezezős tároló.
        (flash memória chip)

  \item A \kix{perifériák}:
        monitor, billentyűzet, egér,
        nyomtató, kommunikációs vonalak, stb.
\end{itemize}
\begin{figure}[H]
  \centering
  \begin{tikzpicture}

    \draw[thick] (0,0) rectangle ++(4,.75);
    \draw[thick] (0,-1.5) rectangle ++(4,.75);
    \draw[] (0,-1.5) rectangle ++(2,.75);
    \draw[thick] (0,-3) rectangle ++(4,.75);

    \draw[implies-implies,double equal sign distance, semithick] (2,0) -- (2,-0.75);
    \draw[implies-implies,double equal sign distance, semithick] (2,-1.5) -- (2,-2.25);

    \node[] at (2, .375) {\kix{háttértárolók}};
    \node[] at (1, -1.125) {\kix{CPU}};
    \node[] at (3, -1.125) {\kix{memória}};
    \node[] at (2, -2.625) {\kix{I/O eszközök}};
  \end{tikzpicture}
  \caption{Neumann-elvű számítógép}
  \label{fig:neumann}
\end{figure}

\subsubsection{A számítógépek szokásos felépítése}

A részegységek egy rendszersínen (\kix{rendszerbusz}) keresztül kapcsolódnak
egymáshoz. A \kix{rendszerbusz}, \kix{mikroprocesszor}, \kix{memória}, valamint
az \kix{eszközvezérlő}[k] nagy része tipikusan az \kix{alaplap}[on] helyezkedik el.
Bővítőkártyák is tartalmazhatnak eszközvezérlőket.
Az eszközvezérlő képes lehet \kix{DMA}-t végezni; ha kész, \kix{megszakítás}[t]
vált ki. Három típusú információ áramolhat: cím, adat, vezérlő.

\subsubsection*{Buszok}

A \kix{busz}[ok] jellemzésére az adat- és címvonalak számát, az adatátvitel
jellemzőit, időzítés adatait, a vezérlőjelek típusait, funkcióit kell megadni.
A cím lehet memóriacím, vagy IO eszköz is címezhető.
a vezérlőjelek lehetnek \dots
\begin{itemize}
  \item adatátvitelt vezérlő jelek
        \begin{itemize}
          \item[$\circ$] cím a sínen
            \tabto{2.7cm} – \tabto{3.3cm}
            memória, periféria (\tc{M}/\tc{IO})

          \item[$\circ$] adat a sínen
            \tabto{2.7cm} – \tabto{3.3cm}
            írás, olvasás (\tc{R}/\tc{W})

          \item[$\circ$] átvitel vége
            \tabto{2.7cm} – \tabto{3.3cm}
            szó, byte átvitel (\tc{WD}/\tc{B})
        \end{itemize}

  \item megszakítást vezérlő jelek
  \item sínvezérlő jelek (kérés, foglalás, visszaigazolás)
  \item egyéb (órajel, ütemezés, táp)
\end{itemize}

\subsection{Memóriák}

\subsubsection*{Memória hierarchia}

\begin{center}
  regiszter \\
  gyorsítótár \\
  központi memória \\
  mágneslemez \\
  szalag, optikai lemez
\end{center}

\subsubsection*{Csoportosítás}

\begin{enumerate}
  \item információ elérése alapján
        \begin{itemize}
          \item cím szerinti hozzáférés

          \item tartalom szerinti hozzáférés (\cix{cache})
        \end{itemize}

  \item hozzáférés belső szervezés alapján
        \begin{itemize}
          \item szekvenciális memóriák

          \item tetszőleges sorrendben címezhető memóriák
                \begin{itemize}
                  \item csak olvasható memóriák (\cix{ROM})

                  \item írható-olvasható memóriák (\cix{RAM})
                \end{itemize}
        \end{itemize}
\end{enumerate}

\subsubsection*{ROM típusok}

Minden egyes típus egyedi karakterisztikával bír, de két dologban közösek.
Az eltárolt adatok ezekben a lapkákban nem illékonyak, azaz nem vesznek el,
amikor kikapcsoljuk az áramot. Az eltárolt adatok megváltoztathatatlanok,
vagy speciális műveletet igényel a változtatás.
(Ellentétben a RAM-mal, melynél könnyű a változtatás)

\begin{itemize}
  \item \cix{ROM} \tabto{1.7cm} – \tabto{2.5cm}
        Read Only Memory
        \begin{itemize}
          \item a gyártó programozza
        \end{itemize}

  \item \cix{PROM} \tabto{1.7cm} – \tabto{2.5cm}
        Programmable Read Only Memory
        \begin{itemize}
          \item felhasználó egyszer programozhatja,
                azaz megfelelő készülékkel kiégetheti
                a cellákban lévő tranzisztorok bekötő vezetékeit
        \end{itemize}

  \item \cix{EPROM} \tabto{1.7cm} – \tabto{2.5cm}
        Erasable Programmable Read Only Memory
        \begin{itemize}
          \item UV fénnyel törölhető, majd külön
                készülékkel újra írható a tartalma

          \item régebben a \cix{ROM BIOS} ilyen
                memóriában helyezkedett el
        \end{itemize}

  \item \cix{EEPROM} \tabto{1.7cm} – \tabto{2.5cm}
        Electrically Erasable Programmable Read Only Memory
        \begin{itemize}
          \item elektromosan törölhető,majd külön
                készülékkel újra írható a tartalma
        \end{itemize}

  \item \cix{FLASH} \tabto{1.7cm} – \tabto{2.5cm}
        Flash/Villanó Memória
        \begin{itemize}
          \item Olyan \cix{EEPROM}, melyet számítógép
                is képes törölni, majd újraírni.

          \item Pendrive-okban, fényképezőgépekben
        \end{itemize}
\end{itemize}

\subsubsection*{RAM típusok}

\begin{itemize}
  \item \cix{SRAM} \tabto{1.7cm} – \tabto{2.5cm}
        Static Random Access Memory
        \begin{itemize}
          \item  a tápfeszültség biztosításával korlátlan
                ideig megőrzi az információt

          \item a memóriacellában egy \cix{flip-flop} található

          \item kisebb integráltságú
                (nagyobb méretű egy cella, mint a \cix{DRAM} esetén)

          \item nagyon gyors: \cix{cache}
        \end{itemize}

  \item \cix{DRAM} \tabto{1.7cm} – \tabto{2.5cm}
        Dynamic Random Access Memory
        \begin{itemize}
          \item az információt egy pici
                \kix{kondenzátor} tárolja

          \item a szivárgás miatt rövid időn belül elveszítené a töltését,
                ezért időközönként (néhány ms) frissíteni kell a tartalmát

          \item nagy integráltságú, a \cix{PC}-k memóriája ilyen
        \end{itemize}
\end{itemize}

\subsection{A CPU részei}

A \cix{CPU} (Central Processing Unit -- központi feldolgozó egység) a
memóriából olvassa a végrehajtás alatt lévő program bináris utasításait.
Az utasításkészlete fontos jellemzője. A mikroprocesszor egy chipen kialakított
áramkör, mely a számítógép \cix{CPU}-jának a funkcióját látja el. Részei:
\begin{itemize}
  \item \kix{ALU} -- aritmetikai és logikai műveletek végzése
        \begin{itemize}
          \item összeadás, kivonás,
                fixpontos szorzás, osztás (léptetések),
                lebegőpontos aritmetikai műveletek (korábban koprocesszor),
                egyszerű logikai műveletek
        \end{itemize}

  \item \kix{Utasítás dekódoló és vezérlő egység}
        \begin{itemize}
          \item Felismeri, elemzi (dekódolja) a gépi nyelvű
                program utasításait, az utasítások alapján működteti
                a \cix{CPU} többi egységét, illetve képezi a szükséges címeket.
        \end{itemize}

  \item \kix{Regiszter}[ek] --
        chipen belüli, közvetlen elérésű tároló elemek
        \begin{itemize}
          \item feladatuk műveletvégzéskor az operandusok tárolása,
                illetve a címek előállítása
        \end{itemize}
\end{itemize}

\subsubsection*{8086 részei}

\begin{itemize}
  \item \kix{szegmensregiszter}[ek]
        \begin{itemize}
          \item \tc{CS} – Code Segment
                \tabto{4cm} – \tabto{4.6cm}
                kódszegmens regiszter

          \item \tc{SS} – Stack Segment
                \tabto{4cm} – \tabto{4.6cm}
                veremszegmens regiszter

          \item \tc{DS} – Data Segment
                \tabto{4cm} – \tabto{4.6cm}
                adatszegmens regiszter

          \item \tc{ES} – Extra Segment
                \tabto{4cm} – \tabto{4.6cm}
                extra adatszegmens regiszter
        \end{itemize}

  \item \kix{vezérlő regiszter}[ek]
        \begin{itemize}
          \item \tc{IP} – Instruction Pointer
                \tabto{4.8cm} – \tabto{5.4cm}
                utasítás mutató

          \item \tc{SP} – Stack Pointer
                \tabto{4.8cm} – \tabto{5.4cm}
                verem mutató

          \item \tc{BP} – Base Pointer
                \tabto{4.8cm} – \tabto{5.4cm}
                bázis mutató

          \item \tc{SI} – Source Index
                \tabto{4.8cm} – \tabto{5.4cm}
                forrás index

          \item \tc{DI} – Destination Index
                \tabto{4.8cm} – \tabto{5.4cm}
                cél index
        \end{itemize}

  \item \kix{általános célú regiszter}[ek] -- \kix{adatregiszter}[ek]
        \begin{itemize}
          \item \tc{AX} – akkumulátor regiszter
                (\tc{AH}, \tc{AL})

          \item \tc{BX} – bázis regiszter
                (\tc{BH}, \tc{BL})

          \item \tc{CX} – számláló regiszter
                (\tc{CH}, \tc{CL})

          \item \tc{DX} – adatregiszter
                (\tc{DH}, \tc{DL})

          \item műveletvégzéskor az operandusok tárolására
        \end{itemize}

  \item \cix{flag}-ek – jelzőbitek, melyek\dots
        \begin{itemize}
          \item vagy a legutóbb elvégzett aritmetikai
                műveletek eredményétől függően vesznek fel értékeket,
                vagy az processzor állapotára utalnak

          \item a \kix{feltételes ugró} utasítások
                a \cix{flag}-eket használják feltételre

          \item aritmetikai flag-ek:
                előjel flag (\cix{sign}),
                zéró flag (\cix{zero}),
                paritás flag (\cix{parity}),
                átvitel flag (\cix{carry})
                (legmagasabb helyiértéken képződött maradék)

          \item processzor állapotára utalóak:
                \cix{trap} flag (program utasításonkénti végrehajtása),
                \cix{interrupt} flag (megszakítás,
                a hardver egységek felől érkező megszakítások
                eljutnak-e a processzorhoz),
                \cix{overflow} flag (túlcsordulás),
        \end{itemize}

  \item \cix{CPU} címzése:
        \begin{itemize}
          \item \kix{memóriacím}[ek]:
                a program utasításainak beolvasására,
                adatainak írására, olvasására

          \item \kix{IO cím}[ek]:
                a perifériákkal való kommunikációra
        \end{itemize}

  \item \kix{Utasításkészlet}
        \begin{itemize}
          \item mikroprocesszorok egyik legfontosabb jellemzője,
                hogy milyen utasításokat ismernek, milyen a gépi nyelvük

          \item a gépi utasítások \kix{bináris jelsorozat}[ok]

          \item az \kix{assembly} nyelv a gépi utasításokat
                \kix{mnemonik}kal helyettesíti
        \end{itemize}

  \item \kix{assembly} utasítások
        \begin{table}[htb]
          \centering
          \begin{tabular}{|l r|}
            \hline
            \bluec{MOV} & \hspace*{1em}adatmozgatás \\
            \bluec{ADD} & összeadás                 \\
            \bluec{SUB} & kivonás                   \\
            \bluec{MUL} & szorzás${}^{1}$           \\
            \bluec{DIV} & osztás${}^{1}$            \\
            \hline
          \end{tabular}
          \begin{tabular}{|l r|}
            \hline
            \bluec{JMP} \blackc{flag} & ugrás                            \\
            \bluec{JZ} \blackc{flag}  & ugrás${}^{2}$                    \\
            \bluec{CMP}               & összehasonlítás                  \\
            \bluec{PUSH}, \bluec{POP} & verem                            \\
            \bluec{LDA}, \bluec{STA}  & \hspace*{1em}akkumulátor${}^{3}$ \\
            \hline
          \end{tabular}
          \caption{Assembly utasítások}
          \label{table:assembly}
        \end{table}
        \begin{enumerate}
          \item \bluec{MUL} $\rightarrow$
                \blackc{AL}$\cdot$\blackc{XX}$=$\blackc{AX}
                \hspace{2em}
                \bluec{DIV} $\rightarrow$
                \blackc{AX}$/$\blackc{XX}$=$\blackc{AL}

          \item \bluec{JZ} $\rightarrow$
                akkor ugrik ha a \blackc{Zero} flag aktív

          \item \bluec{LDA} $\rightarrow$
                2 byte-ot másol a memóriából az akkumulátorba
                (\tc{LoaD Accumulator})
                \\
                \bluec{STA} $\rightarrow$
                az akkumulátor tartalmát a memóriába másolja
                (\tc{STore Accumulator})
        \end{enumerate}
\end{itemize}

\subsection{Utasítás ciklus}

\begin{enumerate}
  \item \cix{fetch} (elérés)
        \begin{itemize}
          \item utasítás kód beolvasása
          \item utasítás kód értelmezése (dekódolás)
          \item operandusok beolvasása
        \end{itemize}
  \item \cix{execute} (végrehajtás)
        \begin{itemize}
          \item műveletvégzés (\cix{ALU})
          \item eredmény tárolása
          \item következő utasítás címének kiszámítása
        \end{itemize}
\end{enumerate}

\subsection{Szubrutinhívás}

\kix{Szubrutinhívás} esetén a program máshol folytatódik. (alprogramra ugrunk)
\tc{CALL}, \tc{RET} utasításpár. Ahhoz, hogy vissza tudjunk térni a megfelelő
helyre, el kell menteni a \tc{PC} (utasításszámláló) értékét a \cix{stack}-be
(\tc{PUSH})

\subsection{Interrupt}

A \kix{megszakítás} (\cix{interrupt}) egy erőltetett vezérlésátadás,
ugrás egy megszakítást kezelő \kix{rutin}[ra]. Előidézheti egy
mikroprocesszorban előforduló esemény. (zéróosztás)
Érkezhet megszakítás egy hardver egység felől is. (adatok beolvasása a
memóriába megtörtént) Program is tartalmazhat megszakítási utasítást.
(oprendszeri szolgáltatás) Fő okai:
\begin{itemize}
  \item processzor megszakítás,
  \item hardver megszakítás, (IRQ: interrupt request)
        \begin{itemize}
          \item lehet maszkolható, vagy nem maszkolható
        \end{itemize}
  \item szoftveres megszakítás.
\end{itemize}
A szubrutinhíváshoz hasonlóan az utasításszámláló értékét a \cix{stack}-be
mentjük.

\subsection{Közvetlen memória hozzáférés}

A \cix{DMA} (Direct Memory Access) a memória és egy periféria (merevlemez)
közötti közvetlen adatátvitel. A DMA vezérlő irányítja az adatforgalmat,
így a \cix{CPU} közben egy másik program kódját futtatja. DMA nélkül az
adatokat a processzoran kellene átvezetni, amely nagyon időigényes lenne.
Kezdetben szükség van az iniciálásra, de utána a folyamat a processzor
igénybevétele nélkül folytatódik.

\end{document}
