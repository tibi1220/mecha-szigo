\documentclass[../../main.tex]{subfiles}

\begin{document}

\section{Grafika II}

\begin{fulltheorem}
  Képszintézis, axonometriák
\end{fulltheorem}

\subsection{Axonometriák}

Az \kix{axonometria} párhozamos sugarakkal való vetítést jelent.
Az $xyz$ 3D-s koordináta\-renszerben lévő pontok párhozamosan vetített
képét berajzoljuk a $\xi \eta$ kordináta\-rendszerű síkbeli lapra.
Az új pontok koordinátái az alábbi egyenletekkel állapíthatóak meg:
\begin{alignat*}{9}
   & \xi  &  & = &  & - x \; q_x \cos \alpha + y \; q_y \cos \beta + z \; q_z \sin \gamma \\
   & \eta &  & = &  & - x \; q_x \sin \alpha - y \; q_y \sin \beta + z \; q_z \cos \gamma
\end{alignat*}

\subsubsection{Izometrikus axonometria}

Az \kix{izometrikus axonometria} egy olyan speciális eset, ahol $xyz$
koordináta\-rendszerben a tengelyek egymással $120^\circ$-ot zárnak be
egymással, valamint $z$ tengely $\eta$ tengellyel esik egybe. Ekkor a
paraméterek az összefüggésben:
\[
  \alpha = \beta = 30^\circ
  \quad
  \gamma = 0^\circ
  \quad
  q_x = q_y = q_z = 1
\]
\begin{figure}[H]
  \centering
  \begin{tikzpicture}[very thick]
    \coordinate (O) at (0,0);

    \draw[ultra thick, -to, yellow!50!black]
    (0,-1) -- ++(0,3.5) coordinate(eta) node[above right, black] {$\eta$};
    \draw[ultra thick, -to, yellow!50!black]
    (-2.5,0) coordinate (nxi) -- ++(5,0) coordinate(xi) node[above right, black] {$\xi$};

    \foreach \i/\c/\l in {210/x/above left, 330/y/above right, 90/z/below right}{
        \draw[-to, cyan!50!black] (\i:-.33) -- ++(\i:2.5)
        coordinate(\c) node[black, \l] {$\c$};
      }
    \draw pic[
        draw=red!50!black,
        angle radius=1.2cm,
        "$\beta$",
        angle eccentricity=.75
      ]{angle=y--O--xi};
    \draw pic[
        draw=red!50!black,
        angle radius=1.2cm,
        "$\alpha$",
        angle eccentricity=.75
      ]{angle=nxi--O--x};
  \end{tikzpicture}
  \caption{Az izometrikus axonometria}
  \label{fig:isometric}
\end{figure}

\subsubsection{Cavalier axonometria}

Az \kix{Cavalier axonometria} esetén $y$ tengely $\xi$ tengellyel, $z$ tengely
pedig $\eta$ tengellyel esik egybe. $x$ tengely a másik kettővel $135^\circ$-ot
zár be. Ekkor a paraméterek az összefüggésben:
\[
  \alpha = 45^\circ
  \quad
  \beta = \gamma = 0^\circ
  \quad
  q_x = 0,5
  \quad
  q_y = q_z = 1
\]
\begin{figure}[H]
  \centering
  \begin{tikzpicture}[very thick]
    \coordinate (O) at (0,0);

    \draw[ultra thick, -to, yellow!50!black]
    (0,-1) -- ++(0,3.5) coordinate(eta) node[above right, black] {$\eta$};
    \draw[ultra thick, -to, yellow!50!black]
    (-2.5,0) coordinate (nxi) -- ++(5,0) coordinate(xi) node[above right, black] {$\xi$};

    \foreach \i/\c/\l in {225/x/above left, 0/y/above left, 90/z/below right}{
        \draw[-to, cyan!50!black] (\i:-.33) -- ++(\i:2.5)
        coordinate(\c) node[black, \l] {$\c$};
      }
    \draw pic[
        draw=red!50!black,
        angle radius=1.2cm,
        "$\alpha$",
        angle eccentricity=.75
      ]{angle=nxi--O--x};
  \end{tikzpicture}
  \caption{Az Cavalier axonometria}
  \label{fig:cavalier}
\end{figure}

\subsection{Képszintézis}

A \kix{képszintézis} a testek megvilágításával, árnyalásával foglalkozik.

\subsubsection{Alapfogalmak}

\paragraph*{\kix{Fluxus}}
Egy adott felületen átáramló energia mennyisége, mértékegysége: Watt.
A fénysűrűség jellemzésére nem jó, mert annak mértékét a fénysugár és a felület
normálvektorának iránya, valamint a felület nagysága is befolyásolja.
\[
  \varPhi = \int_{A} \rvec v \; \dd \rvec A
\]

\paragraph*{\kix{Térszög}}
Az egységgömb felületének része, mértékegysége: szteradián.

\paragraph*{\kix{Látó-térszög}}
\[
  \dd \omega
  = \frac{\dd F}{r^2}
  = \frac{\dd A \cos \alpha}{r^2}
\]

\paragraph*{\kix{Intenzitás}}
Egy felületelemet elhagyó energia a felületelem látható nagyságára és a
térszögére vonatkoztatva, mértékegysége: $\mathrm{W m^{-2} sr^{-1}}$.
\[
  L = \frac{\dd^2 \varPhi}{\cos \alpha \, \dd  A \, \dd \omega}
  \approx \frac{\varPhi}{A \, \omega \cos \alpha}
\]

\end{document}
