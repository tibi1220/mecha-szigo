\documentclass[../../main.tex]{subfiles}

\begin{document}

\newpage

\section{Numerikus módszerek I}

\begin{fulltheorem}
  Rendezés, keresés, interpoláció, regresszió.
\end{fulltheorem}

\subsection{Rendezés}

Korábban tárgyaltuk (KIE).

\subsection{Keresés}

Korábban tárgyaltuk (KIE).

\subsection{Térgörbék leírása}

Megadási módok
\begin{itemize}
  \item Implicit megadás -- $f(x;y;z) = 0$
  \item Explicit megadás -- $x(t) ; y(t); z(t)$
\end{itemize}

\subsubsection{Elsőfokú paraméteres görbeszakasz}
\[
  \rvec Q(t) = (1-t) \rvec P_0 + t \rvec P_1
\]

\subsubsection{Harmadfokú paraméteres görbeszakasz}

Négy definiáló adat kell minden koordináta-függvényhez

\paragraph{Hermite-féle görbeszakasz}
\begin{gather*}
  \rvec Q(t)
  = \rvec P_1 + s_{P1}(t)
  + \rvec P_4 + s_{P4}(t)
  + \rvec R_1 + s_{R1}(t)
  + \rvec R_4 + s_{R4}(t)
  \\
  s_{P1} = 2t^3 - 3t^2 + 1
  \\
  s_{P4} = -2t^3 + 3t
  \\
  s_{R1} = t^3 - 2t^2 + t
  \\
  s_{R4} = t^3 - t^2
\end{gather*}

\paragraph{Bezier-féle görbeszakasz}
\begin{gather*}
  \rvec Q(t)
  = \rvec P_1 \cdot s_{P1}
  + \rvec P_2 \cdot s_{P2}
  + \rvec P_3 \cdot s_{P3}
  + \rvec P_4 \cdot s_{P4}
  \\
  s_{P1} = (1-t)^3
  \\
  s_{P2} = 3t(1-t)^2
  \\
  s_{P3} = 3t^2(1-t)
  \\
  s_{P4} = t^3
\end{gather*}

\subsubsection{Teljes görbék közelítése}

\begin{itemize}
  \item Természetes spline
  \item Kardinális spline
  \item B spline
\end{itemize}

\subsection{Interpoláció}

Egy adott $f$ valós függvény $x_i$ pontjaiban ismerjük $y_i$ függvényértékeket.
Szeretnénk megtudni tetszőleges $x$ pontban a függvény értékét a lehető
legpontosabban. Az ismert pontok elhelyezkedése lehet egyenközű, vagy változó
távolságú. Az \kix{interpoláció}[t] felhasználhatjuk:
\begin{itemize}
  \item függvény közelítésére,
  \item egyenletek megoldására,
  \item differenciálegyenletek megoldására,
  \item numerikus integrálás, deriválás esetén,
  \item diszkrét adatsor folytonossá tevésére.
\end{itemize}

\subsubsection{Lineáris interpoláció}

A \kix{lineáris interpoláció} a legegyszerűbb módszer. Két ismert függvényérték
között számítja ki a keresett függvényértékeket. A függvényt az ismert pontok
között lineárisnak feltételezi. Egyen- és váltakozó közű adatokra is működik.

\subsubsection{Polinomiális interpoláció}

A \kix{polinomiális interpoláció}[] során a függvényt a $x_0$ pont körül
valamilyen $k$-adfokú polinommal szeretnénk közelíteni. Speciális esetben
$k = 1$-nél lineáris interpolációt kapunk. A számításhot $k + 1$ darab
ismert pontra van szükségünk $x_0$ körül. A polinom ezeken a pontokon
boztosan át fog menni. Fajtái:
\begin{itemize}
  \item Lagrange-féle interpoláció
  \item Newton-féle interpoláció
  \item Hermite-féle interpoláció
        \begin{itemize}
          \item Ebben az esetben nem csak azt kívánjuk meg hogy az interpolációs
                polinom átmenjen a pontokon, hanem a deriváltakra is vannak
                megkötéseink.
          \item A sima interpoláció esetén nulladrendű illeszkedésről beszélünk,
                ha az első deriváltakat is figyelembe vesszük akkor elsőrendű,
                és így tovább.
        \end{itemize}
  \item Természetes spline
        \begin{itemize}
          \item A Spline az angol neve annak az acélszalag vonalzónak, amellyel
                a műszaki rajzolók előre kitűzött pontokon keresztül
                görbevonalat rajzolnak.
          \item Az így keletkező függvény folytonos, törésmentes görbülettel
                rendelkezik és kétszer deriválható.
        \end{itemize}
\end{itemize}

\subsection{Regresszió}

Később tárgyaljuk (18).

\end{document}
