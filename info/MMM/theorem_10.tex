\documentclass[../../main.tex]{subfiles}

\begin{document}

\section{Numerikus módszerek VI}

\begin{fulltheorem}
  Mátrixokkal kapcsolatos numerikus módszerek.
\end{fulltheorem}

\subsection{Lineáris egyenletrendszerek}

A \kix{lineáris egyenletrendszer}[ek] általános alakja:
\[
  \rmat A \rvec x = \rvec b
\]

A legismertebb megoldási módszer a Gauss-elimináció. Probléma adódhat, ha
zérus van a főátlóban, hiszen ekkor a determináns nulla (háromszög esetben),
illtve ha rosszul kondícionált a mátrix, pl.:
\begin{gather*}
  \begin{bmatrix}
    1     & 1 \\
    1,001 & 1
  \end{bmatrix} \begin{bmatrix}
    x \\ y
  \end{bmatrix} = \begin{bmatrix}
    2 \\ 2,001
  \end{bmatrix} \; \Rightarrow \; \begin{bmatrix}
    x \\ y
  \end{bmatrix} = \begin{bmatrix}
    1 \\ 1
  \end{bmatrix}
  \\
  \begin{bmatrix}
    1     & 1 \\
    0,999 & 1
  \end{bmatrix} \begin{bmatrix}
    x \\ y
  \end{bmatrix} = \begin{bmatrix}
    2 \\ 1,998
  \end{bmatrix} \; \Rightarrow \; \begin{bmatrix}
    x \\ y
  \end{bmatrix} = \begin{bmatrix}
    0 \\ 2
  \end{bmatrix}
\end{gather*}

Megoldhatjuk a problémát iterációval:
\begin{gather*}
  \rmat A = \rmat B - \rmat C
  \quad \rightarrow \quad
  \rmat B \rvec x = \rmat C \rvec x + \rvec b
  \\
  \rvec x
  = \rmat B^{-1} \rmat C \rvec x + \rmat B^{-1} \rvec b
  = \rmat D \rvec x + \rvec d
  \\
  \text{Iteráció: }
  \rvec x_{n+1} = \rmat D \rvec x_n + \rvec d
\end{gather*}
A módszer csak akkor használható, ha $\rmat B$ invertálható, azaz a normája
kisebb, mint 1.

\subsection{Determináns}

Háromszög mátrix esetében a \kix{determináns}[] a főátlóban lévő elemek szorzata:
\[
  \begin{vmatrix}
    \lambda_1 & \cdot     & \cdot     \\
    0         & \lambda_2 & \cdot     \\
    0         & 0         & \lambda_3
  \end{vmatrix}
  = \lambda_1 \, \lambda_2 \, \lambda_3
\]

\subsection{Mátrixinvertálás}

A klasszikus \kix{mátrixinvertálás}[i] műveletek memóriaigénye ugrásszerűen nő,
egy $12 \times 12$-es mátrix invertálásának memóriaigénye
$12! / 2 \cdot 8 = 2\,\mathrm{GB}$.

A Gauss eliminációhoz hasonló módszerrel fogunk invertálni. Itt egységmátrixokhoz
hasonló transzlációs mátrixokkal fogunk szorozni. Pl.:
\begin{gather*}
  \rmat A = \begin{bmatrix}
    1 & 3 \\ 2 & 5
  \end{bmatrix}
  \\
  \begin{bmatrix}
    1 & 0 \\ -2 & 1
  \end{bmatrix} \rmat A = \begin{bmatrix}
    1 & 3 \\ 0 & -1
  \end{bmatrix}
  \\
  \begin{bmatrix}
    1 & 3 \\ 0 & 1
  \end{bmatrix} \rmat T_1 \rmat A = \begin{bmatrix}
    1 & 0 \\ 0 & -1
  \end{bmatrix}
  \\
  \begin{bmatrix}
    1 & 0 \\ 0 & -1
  \end{bmatrix} \rmat T_2 \rmat T_1 \rmat A = \begin{bmatrix}
    1 & 0 \\ 0 & 1
  \end{bmatrix}
  \\[2mm]
  \underbrace{\rmat T_3 \rmat T_2 \rmat T_1}_{\rmat A^{-1}} \rmat A = \rmat I
\end{gather*}

A lépések szövegesen:
\begin{enumerate}
  \item Hozzunk létre az $\rmat A$ mátrixszal azonos méretű egységmátrixot.
  \item Sorozatos sorműveletekkel alakítsuk át az $\rmat A$ mátrixot felső
        háromszög mátrixszá, mint a Gauss eliminációnál, és minden sorműveletet
        végezzünk el az egységmátrixon is.
  \item Ismételt sorműveletekkel elimináljuk a főátló feletti részt is, és
        tegyünk ugyanígy az egységmátrixszal is.
  \item Az $\rmat A$ mátrixban már csak a főátlóban vannak nem nulla értékek,
        ezért osszuk végig a sorokat a főátló elemeivel, így végül megkapjuk
        $\rmat A$-ban az egységmátrixot, az eredeti egységmátrix helyén pedig
        az inverz mátrixot.
\end{enumerate}

\subsection{Bázisfaktorizáció}

A \kix{bázisfaktorizáció} célja a független egyenletek, vagy független vektorok
kigyűjtése. Itt is felső háromszög mátrix létrehozása a cél. A Purcell-módszerrel
is lehet bázisfaktorizációt végezni, amely a projektor mátrixokra épít.

\subsection{Nemlineáris egyenletrendszer}

Megadási módjai:
\[
  \rvec f(\rvec x) = \mathbf{0}
  \qquad
  \rvec \varphi(\rvec x) = \rvec x
\]

A \kix{nemlineáris egyenletrendszer}[eket] szintén iterációval tudjuk megoldani.
Az iterációs sorozat az alábbi alakban írható fel:
\[
  \rvec x_{n+1} = \rvec \varphi(\rvec x_n)
\]

feltéve, hogy $\rvec x_i$ vektorok benne maradnak $\rvec
  \varphi$ értelmezési tartományában. A konvergencia feltétele:
\[
  \left\| \rvec \varphi(\rvec x_i) - \rvec \varphi(\rvec x_{i+1}) \right\|
  \leq \left\| \rvec x_i - \rvec x_{i+1} \right\| \cdot q
  \text{, ahol } q \in [0; 1).
\]

Ha leosztunk a jobb oldali normával, akkor az egyenlőtlenség
differenciálhányadoshoz hasonló alakot vesz fel:
\[
  \left\| \rvec \varphi'(x) \right\| \leq q < 1
\]

Az iteráció ekkor az alábbi alakban írható fel:
\[
  \rvec x_{n+1}
  = \rvec x_n
  - \left( \rvec f'(\rvec x_0) \right)^{-1} \rvec f(\rvec x_n)
\]

Az érintőmódszer általánosításával: (ez nem mindig konvergens, és minden
lépésben mátrixot kell invertálni, viszont ha konvergens, akkor gyors)
\[
  \rvec x_{n+1}
  = \rvec x_n
  - \left( \rvec f'(\rvec x_n) \right)^{-1} \rvec f(\rvec x_n)
\]

Az érintőmódszert tovább gyorsíthatjuk, ha lineáris egyenletrendszerként oldjuk
meg.
\begin{gather*}
  \rmat A \rvec x_{n+1} = \rvec b \\
  \rmat A = \rvec f'(\rvec x_n)   \\
  \rvec b = \rvec f'(\rvec x_n) \rvec x_n - \rvec f(\rvec x_n)
\end{gather*}

\end{document}
