\documentclass[../../main.tex]{subfiles}

\begin{document}

\clearpage

\section{LabView I}

\begin{fulltheorem}
  Adatfolyam programozás és problémái.
\end{fulltheorem}

\subsection{Az adatfolyam programozás}

A \kix{Labview}[ban] használatos adatfolyam paradigma (néha G-nek is szokás nevezni)
az adatok elérhetőségén alapul. Ha egy \kix{blokk} kiértékeléséhez elegendő adat áll
rendelkezésre, akkor az ki is lesz értékelve. A végrehajtási sorrendet egy
grafikus blokk-diagram (Labview Source Code) diktálja, melyen a blokkok
vezetékekkel vannak összekötve. Ezek a \kix{vezeték}[ek] biztosítják a változók
áramlását, hogy bármely blokk azonnal kiértékelésre kerüljön, amint minden
bemenetére adat érkezett. Mivel egyszerre akár több blokk esetében is
teljesülhet ez a feltétel, ezért a LabView képes a párhuzamos szálak futtatására.
A Multi-processing és multi-threading hardvert automatikusan felhasználja a
beépített ütemető, amely több operációsrendszer-szálat multiplexel a
végrehajtásra váró blokkokra.

\subsection{A LabView program és részei}

A LabView programokat  virtuális műszereknek (\cix{VI} -- \kix{Virtual Instrument})
szokás nevezni. Minden \tc{VI} 3 részből áll:
\begin{itemize}
  \item \tc{Front Panel} --
        Bemenetekből (\tc{control}) és kimenetekből (\tc{indicator}) épül fel.
  \item \tc{Block Diagram} --
        A grafikus forráskódot tartalmazza. A \tc{Front Panelen} lévő változók
        itt terminálként fognak megjelenni. Struktúrákat és függvényeket is
        tartalmaz, melyek a \tc{Functions Palette}-en találhatóak.
  \item \tc{Connector Pane} --
        A \tc{VI} blokkdiagrammon való reprezentációját határozza meg.
\end{itemize}

\subsection{Problémák}

LabView használata során az alábbi problémák merülhetnek fel:
\begin{itemize}
  \item A párhuzamos futás miatt nem tudhatjuk, hogy pontosan mikor,
        melyik blokk lesz kiértékelve, akár egymást követő futások során
        is más futási sorrendet kaphatunk.
  \item Nehéz a vezetékeket esztétikusan elrendezni, ezen az auto-formatter
        sem szokott segíteni.
  \item Nem lehet nagyítani.
\end{itemize}

\end{document}
