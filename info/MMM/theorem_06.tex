\documentclass[../../main.tex]{subfiles}

\begin{document}

\section{Numerikus módszerek II}

\begin{fulltheorem}
  Hibafajták a numerikus módszerek esetén, érdekességek.
\end{fulltheorem}

{\color{red}\textbf{Disclamer:}} A következő fejezetekben írt konkrét számadatok
\cix{lua} nyelvvel számolt numerikus adatok.
(64-bites lebegőpontos ábrázolással (IEEE 754))

\subsection{Hibafajták}

Beszélhetünk relatív ($|x - a|$) egy abszolút ($|x - a| / a$) hibáról.
A hibák okati lehetnek kerekítési-, öröklött-, vagy képlethiba.

\subsubsection{Kiegyszerűsödés}

Ez a jelenség akkor fordul elő, ha két, közel azonos számot vonunk ki egymásból.
Ilyenkor az eredmény sokkal kevesebb értékes jegyből fog állni, ami
információvesztést jelent. A deriváltak numerikus közelítésénél is ez a
jelenség okozza a legtöbb gondot.
\begin{luacode*}
  d = math.sqrt(150000^2 - 4)
  d2 = d/2
  x1 = 75000 + d/2
  x2 = 75000 - d/2
  x3 = 1 / x1
  s1 = x1^2 - 150000*x1 + 1
  s2 = x2^2 - 150000*x2 + 1
  s3 = x3^2 - 150000*x3 + 1
\end{luacode*}
\begin{gather*}
  x^2  - 150000 x + 1 = 0 \\
  x_{12} = 75000 \pm \Nlv{d2} = \begin{cases}
    \Nlv{x1} \\
    \Nlv{x2}
  \end{cases}
\end{gather*}
Visszahelyettesítés után:
\[
  x_{12}^2 - 150000 x_{12} + 1 = \begin{cases}
    \Nlv{s1}[0] \\
    \Nlv{s2}
  \end{cases}
\]
Ha az alábbi azonossággal számoljuk ki $x_2$-t, akkor elkerülhetjük a problémát.
\[
  x_1 x_2 = \frac{c}{a}
  \quad \rightarrow \quad
  x_2 = \Nlv{x3}
\]
Ha ezt helyettesítjük vissza, akkor láthatjuk, hogy a hiba el lett kerülve.
\[
  x_2^2 + 150000 x_2 + 1 = \Nlv{s3}[0]
\]

\subsubsection{Kiejtés}

A kiejtés általában sorok összegzésekor fordul elő, akkor ha a részösszegek
számításkor a kiegyszerűsödés jelenségével találkozunk. Tulajdonképpen úgy is
felfoghatjuk, mint az előző fejezetben tárgyalt fogalom általánosítását.

Leggyakrabban alternáló sorok összegzésekor találkozunk vele. Szemléltetésként
nézzük a következő példát! Az $e^x$ függvény nulla körüli Taylor sora a
következő:
\[
  e^x = \sum_{k=0}^\infty \frac{x^k}{k!}
\]
Negatív exponens esetén érdemes az abszolútértékével kiszámolni a sort, majd
venni a reciprokát:
\begin{luacode*}
  function factorial(n)
  fact = 1.0
  for i = 1, n, 1
  do
  fact = fact * i
  end
  return fact
  end
  x1=-22; x2=22
  s1=0; s2=0
  for k=0,100,1
  do;
  s1 = s1 + x1^k / factorial(k)
  s2 = s2 + x2^k / factorial(k)
  end
  s2 = 1 / s2
\end{luacode*}
\begin{align*}
  e^{-22}     & = \Nlv{s1} \\
  1 / e^{+22} & = \Nlv{s2}
\end{align*}
Itt nem csak az okozza az eltérést, hogy az előjelek alternáltak, hanem az is
hogy a tagok nagyságrendje folyamatosan nőtt. ($k$=100-ig ment a \cix{for} ciklus)

\subsubsection{Numerikus instabilitás}

\kix{Instabilitás}[ról] akkor beszélünk, ha a közbenső hibák nagymértékben
befolyásolják az eredményt.
\begin{gather*}
  y_n = \int_0^1 x^n \, e^x \, \dd x
  \\
  y_n
  = \int_0^1 x^n \, e^x \, \dd x
  = \left. x^n \, e^x \right|_0^1 - n \int_0^1 x^{n-1} \, e^x \, \dd x
  = e - n \, y_{n-1}
\end{gather*}

Ha rekurzívan szeretnénk a határértéket kiszámolni, akkor hiba fog keletkezni,
amely a visszafele számoláskor ki fog derülni, hiszen a rekurzív alak miatt
a képlet az alábbi alakra írható át
\[
  y_n = e - n \, y_{n-1}
  \quad \rightarrow \quad
  y_{n-1} = \frac{e - y_n}{n}
\]
\begin{table}[htb]
  \centering
  \includegraphics{../../static/tex/build/table-instability.pdf}
  \caption{Numerikus instabilitás szemléltetése}
  \label{table:instability}
\end{table}

\subsubsection{Rosszul kondicionáltság}

Egy feladatot akkor hívunk rosszul kondicionáltnak, ha a paramétereinek csekély
megváltoztatása az eredmény nagymértékű módosulását eredményezi. Ilyen
kismértékű változás a számítástechnikában szinte mindig előfordul, elég csak
kerekítési hibákra gondolnunk.
\begin{alignat*}{9}
  (x - 1)^6 & = & 0       & \quad \rightarrow \quad & x = 1   \\
  (x - 1)^6 & = & 10^{-6} & \quad \rightarrow \quad & x = 1,1
\end{alignat*}
Megfigyelhetjük, hogy a paraméterek kis változtatásával a végeredmény
$10\%$-kal nőtt.

\end{document}
