\documentclass[../../main.tex]{subfiles}

\begin{document}

\section{Numerikus módszerek IV}

\begin{fulltheorem}
  Numerikus deriválás alapjai.
\end{fulltheorem}

A deriválás matematikai definíciója:
\[
  f'(a)
  = \lim_{h \rightarrow 0} \frac{f(a + h) - f(a)}{h}
  = \lim_{h \rightarrow 0} \frac{f(a) - f(a - h)}{h}
\]

Ezt a legkönnyebben a differenciahányadossal közelíthetjük:
\[
  f_i' \approx \frac{f_{i + 1} - f_i}{\Delta t}
  \qquad
  f_i' \approx \frac{f_i - f_{i - 1}}{\Delta t}
\]

Gyakran nem ismerjük a függvényt, viszont a mérés segítségével egy másodfokú
polinommal tudjuk közelíteni:
\[
  f(x) = ax^2 + bx + c
\]

Ekkor a deriváltat az alábbi összefüggéssel közelíthetjük:
\[
  f_i' = \frac{f_{i+1} - f_{i-1}}{2 \Delta t}
\]

Taylor-sorok alapján is előállíthatóak formulák:
\[
  f(x)
  = \sum_{k=0}^{n - 1} \frac{f^{(i)}(x_0)}{k!} (x - x_0)^k
  = f(a)
  + f'(a) (x - x_k)
  + \frac{f''(a)}{2} (x - x_k)^2
  + \ldots
  + \frac{f^{(n)}(\xi)}{n!} (x - x_k)^n
\]

Másodrendű derivált a maradéktagok elhagyásával:
\[
  \begin{rcases}
    f_{i-1} = f_i + f'(t_{i-1} - t_{i}) + \frac{f''_i}{2}(t_{i-1} - t_i) \\
    f_{i+1} = f_i + f'(t_{i+1} - t_{i}) + \frac{f''_i}{2}(t_{i+1} - t_i)
  \end{rcases} \Rightarrow \;
  f_{i+1} + f_{i-1} \approx 2 f_i + f''_i \Delta t^2
\]\[
  f''_i = \frac{f_{i+1} - 2 f_{i} + f_{i-1}}{\Delta t^2}
\]

Ekkor a hibatag:
\[
  \begin{rcases}
    f_{i-1}
    = f_i
    - f_i' \Delta t
    + \frac{f_i''}{2} \Delta t^2
    - \frac{f_i'''}{6} \Delta t^3
    + \frac{f_i''''(\zeta)}{24} \Delta t^4
    \\
    f_{i+1}
    = f_i
    + f_i' \Delta t
    + \frac{f_i''}{2} \Delta t^2
    + \frac{f_i'''}{6} \Delta t^3
    + \frac{f_i''''(\xi)}{24} \Delta t^4
  \end{rcases} \Rightarrow \;
  H(\eta \in (\zeta ; \xi)) = \frac{-f''''(\eta)}{12} \Delta t^2
\]
\end{document}
