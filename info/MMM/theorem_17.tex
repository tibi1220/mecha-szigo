\documentclass[../../main.tex]{subfiles}

\begin{document}

\section{Programozás VI}

\begin{fulltheorem}
  Evolúciós algoritmusok alapjai, genetikus algoritmusok.
\end{fulltheorem}

\subsection{Evolúciós algoritmusok}

Az \kix{evolúciós algoritmus}[ok] alapelve a megoldások egy populációján
történő keresés, melyet a biológiából megismert törvényszerűségek vezérelnek:
\begin{itemize}
  \item A populáció egyedei a feladat egy-egy megoldását jelentik.
  \item A populáció fejlődik, egyre jobb egyedeket kapunk.
\end{itemize}

\subsubsection{Alapfogalmak}

\begin{itemize}
  \item \kw{Gén} -- Funkcionális entitás, az egyed egy tulajdonságát kódolja.
        (pl. hajszín)
  \item \kw{Allél} -- A gén értéke. (pl. szőke)
  \item \kw{Egyed} -- kromoszóma, egy megoldás jelölt a problémára.
  \item \kw{Genotípus} -- Egy egyed alléljainak egy speciális kombinációja.
  \item \kw{Fenotípus} -- Az egyed külső-belső tulajdonságainak összessége.
  \item \kw{Locus} -- Egy gén pozíciója a kromoszómán belül.
  \item \kw{Populáció} -- Egyszerre együtt élő egyedek összessége.
\end{itemize}

\subsection{Genetikus algoritmusok}
\begin{figure}[H]
  \centering
  \includegraphics{../../static/tex/build/genetic-algs.pdf}
  \caption{A genetikus algoritmusok folyamata}
  \label{fig:genalg}
\end{figure}

\subsubsection{Az egyedek rangsorolása}
\begin{itemize}
  \item \kix{Fitness érték} alapján
        \begin{itemize}
          \item Alkalmassági érték --  az egyedeket valamilyen kritérium
                szerint értékeljük ki, aszerint, hogy mennyire jó megoldást
                adnak a feladatra.
          \item Jobb egyednek nagyobb a fitnesz értéke, és nagyobb eséllyel
                éli túl.
        \end{itemize}
  \item Szelekciós módszerek
        \begin{itemize}
          \item Minél jobb az egyed, annál nagyobb az esély a kiválasztására.
          \item Rulett kerék szelekció: a gurításnál a nagyobb fitnesz értékű
                egyedek nagyobb eséllyel választódnak ki.
        \end{itemize}
\end{itemize}

\end{document}
