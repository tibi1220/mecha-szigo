\documentclass[../../main.tex]{subfiles}

\begin{document}

\section{Programozás III}

\begin{fulltheorem}
  Neurális hálózatok alapjai, a Perceptron, a Perceptron tanítása.
\end{fulltheorem}

A \kix{neurális háló}[k] neuronjainak mintája az agyi neuronok viselkedése.
Minták formájában reprezentált tudás megtanulására képesek. A tudás típusa
lehet:
\begin{itemize}
  \item analitikus, (pl. matematikai egyenletek)
  \item szabályalapú, (pl. fuzzy rendszerek)
  \item tapasztalati. (minták, megfigyelések)
\end{itemize}

\subsection{Tanítási módszerek}

\paragraph{Felügyeletlen tanulás}

Ilyen eset pl. ha címkézetlen képek alapján kell azonosítani az állatok fajtáját.

\begin{figure}[H]
  \centering
  \includegraphics{../../static/tex/build/neural-learning-1.pdf}
  \caption{Felügyelet nélküli tanulás}
  \label{fig:learning1}
\end{figure}

\paragraph{Felügyelt tanulás}

Ilyen eset pl. ha címkézett képek alapján kell azonosítani az állatok fajtáját.

\begin{figure}[H]
  \centering
  \includegraphics{../../static/tex/build/neural-learning-2.pdf}
  \caption{Felügyelt tanulás}
  \label{fig:learning2}
\end{figure}

\paragraph{Hagyományos tanulás}\;

\begin{figure}[H]
  \centering
  \includegraphics{../../static/tex/build/neural-learning-3.pdf}
  \caption{Hagyományos tanulás}
  \label{fig:learning3}
\end{figure}

\subsection{A mesterséges neuron}

A \kix{mesterséges neuron}[] egy információ feldolgozó egység, amely alapvető
szerepet játszik a mesterséges neurális hálózatokban. A biológiai neuront
utánozó model. Matematikai leírása:
\[
  y = \varPhi \left( \sum_{i=0}^n x_i \, w_i \right)
\]
\begin{center}
  \begin{tabular}{c c l}
    $x_i$     & -- & bemenetek           \\
    $w_i$     & -- & súlyok              \\
    $y$       & -- & kimeneti jel        \\
    $\sum$    & -- & összegző            \\
    $\varPhi$ & -- & aktivációs függvény
  \end{tabular}
\end{center}
\begin{figure}[H]
  \centering
  \includegraphics{../../static/tex/build/neuron.pdf}
  \caption{Egy mesterséges neuron modellje}
  \label{fig:neuron}
\end{figure}

\subsection{Aktivációs függvény}

Az \kix{aktivációs függvény} behatárolja a kimenetet két aszimptota közé,
és ezáltal a neuron kimenete mindig egy észszerű dinamikatartományban fog
elhelyezkedni.

\subsubsection{Szigmoid}
\[
  \sigma(x) = \frac{1}{1 + e^{-x}}
  \qquad
  \sigma'(x) = \sigma(x) \left( 1 - \sigma(x) \right)
\]
\begin{figure}[H]
  \centering
  \includegraphics{../../static/tex/build/sigmoid.pdf}
  \caption{A szigmoid függvény és deriváltja}
  \label{fig:sigmoid}
\end{figure}

\subsubsection{Rectifier Linear Unit}
\[
  \mathrm{ReLU}(x) = \begin{cases}
    0, & \text{ ha } x < 0    \\
    x, & \text{ ha } x \geq 0
  \end{cases}
  \qquad
  \mathrm{ReLU}'(x) = \begin{cases}
    0, & \text{ ha } x < 0    \\
    1, & \text{ ha } x \geq 0
  \end{cases}
\]
\begin{figure}[H]
  \centering
  \includegraphics{../../static/tex/build/relu.pdf}
  \caption{A ReLU függvény és deriváltja}
  \label{fig:relu}
\end{figure}

\subsubsection{Hiperbolikus tangens}
\[
  \tanh(x) = \frac{e^{2x} - 1}{e^{2x} + 1}
  \qquad
  \tanh'(x) = 1 - \tanh^2(x)
\]
\begin{figure}[H]
  \centering
  \includegraphics{../../static/tex/build/tanh.pdf}
  \caption{A $\tanh(x)$ függvény és deriváltja}
  \label{fig:tanh}
\end{figure}

\subsubsection{Leaky ReLU}
\[
  \mathrm{LReLU}(x) = \begin{cases}
    x/10, & \text{ ha } x < 0    \\
    x,    & \text{ ha } x \geq 0
  \end{cases}
  \qquad
  \mathrm{LReLU}'(x) = \begin{cases}
    1/10, & \text{ ha } x < 0    \\
    1,    & \text{ ha } x \geq 0
  \end{cases}
\]
\begin{figure}[H]
  \centering
  \includegraphics{../../static/tex/build/lrelu.pdf}
  \caption{Az LReLU függvény és deriváltja}
  \label{fig:lrelu}
\end{figure}

\end{document}
