\documentclass[../../main.tex]{subfiles}

\begin{document}

\section{Grafika I}

\begin{fulltheorem}
  Grafikusadattárolás (vektor, raszter)
\end{fulltheorem}

\subsection{Vektoros adattárolás}

\kix{Vektoros adattárolás}[] esetén az adatokat geometriai alakzatok segítsével
tároljuk (pl. egyenes, kör, négyszög, stb.) Ilyen adattárolási stratégiát használnak
például a számítógépes \kix{betűtípus}[ok], (\cix{TTF} -- True Type Font)
melyet az Apple Inc. fejlesztett ki az 1980-as évek elején.

A geometriai alakzatokat többféle módszerrel tárolhatjuk:

\begin{enumerate}
  \item \kix{Lineáris lista}
        \begin{table}[H]
          \centering
          \begin{tabular}{| c | c | c |}
            \hline
            Sorszám & $x$ & $y$
            \\ \hline \hline
            1       & 0   & 0
            \\ \hline
            2       & 40  & 0
            \\ \hline
            3       & 40  & 20
            \\ \hline
            4       & 0   & 20
            \\ \hline
            5       & 0   & 0
            \\ \hline
          \end{tabular}
          \caption{Téglalap reprezentációja lineáris listával}
          \label{table:rec-linlist}
        \end{table}

  \item \kix{Láncolt lista}
        \begin{table}[H]
          \centering
          \begin{tabular}{| c | c | c | c |}
            \hline
            Sorszám & $x$ & $y$ & Következő csúcs
            \\ \hline \hline
            1       & 0   & 0   & 2
            \\ \hline
            2       & 40  & 0   & 3
            \\ \hline
            3       & 40  & 20  & 4
            \\ \hline
            4       & 0   & 20  & 5
            \\ \hline
            5       & 0   & 0   & 0
            \\ \hline
          \end{tabular}
          \caption{Téglalap reprezentációja láncolt listával}
          \label{table:rec-forwardlist}
        \end{table}

  \item \kix{Keresztláncolt lista}
        \begin{table}[H]
          \centering
          \begin{tabular}{| c | c | c | c |}
            \hline
            Sorszám & $x$ & $y$ & Következő csúcs
            \\ \hline \hline
            1       & 0   & 0   & 2
            \\ \hline
            2       & 40  & 0   & 3
            \\ \hline
            3       & 40  & 20  & 4
            \\ \hline
            4       & 0   & 20  & 0
            \\ \hline
          \end{tabular}
          \hspace{1em}
          \begin{tabular}{| c | c | c | c |}
            \hline
            Sorszám & KP & VP & Következő él
            \\ \hline \hline
            1       & 1  & 2  & 2
            \\ \hline
            2       & 2  & 3  & 3
            \\ \hline
            3       & 3  & 4  & 4
            \\ \hline
            4       & 4  & 0  & 0
            \\ \hline
          \end{tabular}
          \caption{Téglalap reprezentációja keresztláncolt listával}
          \label{table:rec-crosslist}
        \end{table}
\end{enumerate}

\subsection{Raszteres adattárolás}

A \kix{raszteres adattárolás}[] során  nem geometriai alakzatokat, hanem
képpontokat tárolunk el. A képontokat \kix{fa} segítségével tárolhatjuk.
A 2D-s képeket 4-es fákkal, a 3D-s képeket nyolcas fákkal tárolhatjuk.

Raszteres formátumok
\begin{itemize}
  \item \cix{JPEG} -- veszteséges tömörítés
  \item \cix{BMP} -- veszteségmentes tömörítés
  \item \cix{GIF} -- 256 bites színpalettával rendelkezik
  \item \cix{PNG} -- tartalmazhat alfa csatornát is
  \item \cix{RAW} -- adatok minimálisan vannak feldolgozva
  \item \cix{XCF} -- GIMP saját formátuma
\end{itemize}

\subsection{Grafikus CAD tárgymodellek}

\begin{enumerate}
  % noindent
  \item Pont \\
        (sorszám, $x$, $y$, $z$, következő)
  \item Él \\
        (sorszám, él egyenlete, KP, VP, $z$, következő)
  \item Felület \\
        (sorszám, felület egyenlete, határélek, következő)
  \item Boudary Representation \\
        (sorszám, felület egyenlete, határélek, $n_x$, $n_y$, $n_z$, következő)
  \item CSG (Constructive Solid Geometry) \\
        (alaksajátosság, paraméteres elem, technológia (unió, metszet és különbség))
  \item STL (Standard Tessallation Language) \\
        (csúcsok, normális)
  % indent
\end{enumerate}

\end{document}
