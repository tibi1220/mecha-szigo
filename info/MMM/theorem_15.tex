\documentclass[../../main.tex]{subfiles}

\begin{document}

\section{Programozás IV}

\begin{fulltheorem}
  Mesterséges neurális hálózatok, felügyelt tanulás.
\end{fulltheorem}

\subsection{Mesterséges neurális hálózat}

A \kix{mesterséges neurális hálózat}[ok] (\kix{ANN}) adat feldolgozó rendszerek,
melyek nagy számú egyszerű, összekapcsolt feldolgozó elemek (mesterséges
neuronok) architektúrája, melyet az agy komplex viselkedésre képes szerkezete
inspirált.
Az \kix{ANN}-ben a neuronok rétegek sorozatába vannak szervezve teljes vagy
részleges kapcsolattal az egyes rétegek között.
Az egyszerűség kedvéért, általában az ugyanabban a rétegben lévő neuronok
aktivációs függvénye megegyezik.

A mesterséges neurális hálókat sűrűségük szerint 2 csoportba sorolhatjuk be:
\begin{itemize}
  \item \kix{Tejlesen kapcsolt ANN}
        \begin{itemize}
          \item Egy adott réteg összes neuronja kapcsolatban áll a következő
                réteg összes neuronjával.
          \item Előnye: egyszerű felépítés,
                lehetőségeket maximálisan kihasználja.
          \item Hátránya: nagy memória és számítási igény.
        \end{itemize}
  \item \kix{Részlegesen kapcsolt ANN}
        \begin{itemize}
          \item Egy adott réteg összes neuronja a következő rétegből csak
                néhány neuronnal áll kapcsolatban.
        \end{itemize}
\end{itemize}

\begin{figure}[H]
  \centering
  \includegraphics[scale=1]{../../static/tex/build/ann-fully-connected.pdf}
  \caption{Teljesen kapcsolt ANN}
  \label{fig:ann-fc}
\end{figure}


Jelfolyam szerinti csoportosítás:
\begin{enumerate}
  \item \kix{Előrecsatolt}
        \begin{itemize}
          \item Időben független adatok gyors feldolgozása.
          \item Időben függő adatok nem valós idejű pontos feldolgozása.
        \end{itemize}
  \item \kix{Celluláris}
        \begin{itemize}
          \item Csak a szomszédos egységek között van kapcsolat.
          \item Főleg képfeldolgozásra használatos.
        \end{itemize}
  \item \kix{Visszacsatolt}
        \begin{itemize}
          \item Időben függő adatok gyors feldolgozása.
        \end{itemize}
\end{enumerate}

\begin{figure}[H]
  \centering
  \hfill
  \begin{subfigure}[c]{.35\textwidth}
    \centering
    \includegraphics[scale=.6]{../../static/tex/build/ann-forward.pdf}
    \subcaption{1 -- előrecsatolt}
  \end{subfigure}
  \hfill
  \begin{subfigure}[c]{.25\textwidth}
    \centering
    \includegraphics[scale=.6]{../../static/tex/build/ann-cellular.pdf}
    \subcaption{2 -- celluláris}
  \end{subfigure}
  \hfill
  \begin{subfigure}[c]{.35\textwidth}
    \centering
    \includegraphics[scale=.6]{../../static/tex/build/ann-backward.pdf}
    \subcaption{3 -- visszacsatolt}
  \end{subfigure}
  \hfill
  \caption{Jelfolyam szerinti csoportosítás}
  \label{fig:ann-flow}
\end{figure}

\subsection{Teljesen kapcsolt ANN tanítása}

Egy ciklusban $P$ számú bemenetet és kimenetet vizsgálunk.
A kimenet és az elvárt kimenetet összehasonlítjuk, és definiálunk egy
hibafüggvényt. A hiba visszaterjesztését rétegről rétegre végezzük visszafelé
haladva kihasználva, hogy a segédparamétert iteratívan tudjuk számolni.


\end{document}
