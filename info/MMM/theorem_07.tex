\documentclass[../../main.tex]{subfiles}

\begin{document}

\section{Numerikus módszerek III}

\begin{fulltheorem}
  Gyökkeresés módszerei: polinomok, egyéb egyenletek, egyenletrendszerek esetén.
\end{fulltheorem}

\subsection{Intervallum keresés}

Bolzano-tétel: $!f: \mathbb R \rightarrow \mathbb R$ folytonos függvény.
Ha $f(a) f(b) < 0$, akkor $\exists \xi \in (a;b)$, hogy $f(\xi) = 0$,
vagyis a függvénynek létezik ezen intervallumon zérushelye.

\subsection{Általános algoritmusok}

\paragraph{Intervallum-felezés}

Talán ez a legegyszerűbb módszer. Előnye, hogy mindig használható, a gyök
tetszőleges pontosságú megközelítésére biztosít lehetőséget. Hátránya a
közelítés sebessége.
\begin{gather*}
  \xi_i = \frac{a_i + b_i}{2}
  \\
  a_{i+1} = \begin{cases}
    a_i \text{, ha } f(a_i)f(\xi_i) < 0   \\
    \xi_i \text{, ha } f(a_i)f(\xi_i) > 0 \\
  \end{cases}
  \qquad
  b_{i+1} = \begin{cases}
    b_i \text{, ha } f(b_i)f(\xi_i) > 0   \\
    \xi_i \text{, ha } f(b_i)f(\xi_i) < 0 \\
  \end{cases}
\end{gather*}

\paragraph{Húrmódszer}

Az algoritmus lépései sokban hasonlítanak az előzőekben leírt
intervallumfelezéshez, csak a $\xi_i$ pontok meghatározása történik másként.
Húzzunk egyenest, amely átmegy a $(a_i; f(a_i))$ és $(b_i; f(b_i))$
pontokon. $\xi_i$ pont legyen ennek az egyenesnek az $x$ tengelymetszete.
\[
  \xi_i = b_i - f(b_i) \frac{b_i - a_i}{f(b_i) - f(a_i)}
\]
Az új intervallumok meghatározása az előzőekhez analóg módon történik.

\paragraph{Iteráció}

Ez a módszer nem mindig konvergens, de könnyen használható és gyorsan eredményt
ad. Az egyenlet $f(x) = 0$ alakban adott, a gyököket továbbra is az $(a; b)$
intervallumon keressük.
\begin{gather*}
  f(x) + x = x \\
  g(x):= f(x) + x = x \\
  x_1 = g(x_0) \\
  \vdots \\
  x_n = g(x_{n-1}) \\
  \vdots \\
  x_\infty = x_\infty + f(x_\infty) \\
  f(x_\infty) = 0
\end{gather*}

A konvergencia feltétele, hogy $|g'(x)| < 1$. $f(x) = 1 - \sin x$ esetén
divergens.

Az iteráció gyorsítható az intervallumban monoton növekvő függvény esetén.
\begin{gather*}
  f(x) = 0 \; \Rightarrow \alpha f(x) = 0 \\
  f'(x) \; \Rightarrow \alpha f'(x) \\
  |g'(x)| < 1 \; \Rightarrow |f'(x) + 1| < 1 \\
  |f'(\xi)| = \max \left\{ f'(x); x \in (a;b) \right\} \\
  M := f'(x) \quad \alpha := -1/m \\
  \left| -\frac{f'(x)}{M} + 1 \right| < 1 \text{ mindig teljesül} \\
  x_n = x_{n - 1} - \frac{f(x_{n-1})}{M}
\end{gather*}

\paragraph{Érintőmódszer (Newton--Rapson)}

Vegyünk egy tetszőleges $x_0 \in (a; b)$ pontot az intervallumon és húzzuk
be az érintőjét. Az érintő $x$ tengelymetszete lesz a következő pont,
melynek a meredekségével iterálunk.
\[
  y - f(x_n) = f'(x_n)(x - x_n)
  \quad \rightarrow \quad x_n = x_{n-1} - \frac{f(x_{n-1})}{f'(x_{n-1})}
\]

Probléma lehet a körbejárás, illetve felléphet az is, hogy kilépünk az
intervallumból.

\subsection{Polinomok}

Mivel egy polinomot is kezelhetünk függvényként, az előző fejezetben közölt
gyökkereső eljárások itt is alkalmazhatók, viszont a polinomoknak általában
nem csak egy gyökre van szükségünk az adott környezetben, hanem az összesre.

\paragraph{Horner-séma}

Az egyik leggyakoribb tevékenység, amit egy polinommal el kell végeznünk,
hogy meghatározzuk egy adott helyen a helyettesítési értékét. A Horner-séma
ennek a számításnak a műveletek számára optimalizált eljárása, azaz itt kell a
legkevesebbet számolnunk. Gyakorlatilag csak egy átrendezésről van szó.
Legyen a polinom:
\begin{gather*}
  p(x)
  = a_n x^n
  + a_{n-1} x^{n-1}
  + a_{n-2} x^{n-2}
  + \dots
  + a_1 x
  + a_0
  = 0
  \\
  p(x)
  =
  a_0 + x (
  a_1 + x (
  a_2 + \dots + (
  a_{n-2} + x (
  a_{n-1} + x a_n
  )) \dots ))
\end{gather*}

\paragraph{A gyököket tartalmazó intervallum}

Az alábbi egyenlőtlenség mindig teljesül:
\begin{gather*}
  A = \max\left\{|a_i|; i = \{0;1;\dots;n-1\}\right\} \\
  B = \max\{|a_i|; i = \{1;2;\dots;n\}\} \\
  \frac{1}{1 + B / |a_0|} < |x_i| < 1 + \frac{A}{|a_n|}
\end{gather*}

\subsection{Egyenletrendszerek}

Későbbi tételben tárgyalva van. (Mátrixok)

\end{document}
