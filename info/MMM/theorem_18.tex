\documentclass[../../main.tex]{subfiles}

\begin{document}

\section{Programozás VII}

\begin{fulltheorem}
  Evolúció stratégiák, bakteriális evolúciós algoritmusok, genetikus programozás.
\end{fulltheorem}

\subsection{Evolúciós stratégiák}

\subsubsection{Keresztezés -- Crossover}
\begin{itemize}
  \item Véletlenszerű keresztezési pont kiválasztása a két szülőn.
  \item Utódok létrehozása az információ kicserélődésével a keresztezési
        pont alapján.
\end{itemize}
\begin{table}[H]
  \centering
  \begin{tabular}{|*{20}{>{\centering\arraybackslash}p{.2cm}|}}
    \hline
    \rowcolor{gray!10}
    \luaexec{for i=1,19,1 do tex.sprint(0 .. " & ") end} 0
    \\ \hline
    \luaexec{for i=1,19,1 do tex.sprint(1 .. " & ") end} 1
    \\ \hline
  \end{tabular}
  \\[2mm]
  $\Downarrow$
  \\[2mm]
  \begin{tabular}{|*{20}{>{\centering\arraybackslash}p{.2cm}|}}
    % noindent
    \hline
    \luaexec{for i=1,5,1 do tex.sprint("\\cellcolor{gray!10}" .. 0 .. "&") end} \cellcolor{gray!10} 0
    \luaexec{for i=1,14,1 do tex.sprint("&" .. 1) end}
    \\ \hline
    \luaexec{for i=1,5,1 do tex.sprint(1 .. "&") end} 1
    \luaexec{for i=1,14,1 do tex.sprint("&\\cellcolor{gray!10}" .. 0) end}
    \\ \hline
    % indent
  \end{tabular}
  \caption{A keresztezés szemléltetése}
  \label{table:crossover}
\end{table}

\subsubsection{Mutáció}
A gén értékének megváltoztatása $p_m$ valószínűséggel. (mutációs arány)
\begin{table}[H]
  \centering
  \begin{tabular}{|*{20}{>{\centering\arraybackslash}p{.2cm}|}}
    \hline
    \luaexec{for i=1,19,1 do tex.sprint(1 .. " & ") end} 1
    \\ \hline
  \end{tabular}
  \\[2mm]
  $\Downarrow$
  \\[2mm]
  \begin{tabular}{|*{20}{>{\centering\arraybackslash}p{.2cm}|}}
    % noindent
    \hline
    \luaexec{
      for i=1,19,1 do
      if (math.fmod(i,3) == 0 or math.fmod(i,7) == 0) then
        tex.sprint(1 .. "&") 
      else
        tex.sprint("\\cellcolor{gray!10}" .. 0 .. "&") 
      end
      end
    } 1
    \\ \hline
    % indent
  \end{tabular}
  \caption{A mutáció szemléltetése}
  \label{table:mutation}
\end{table}

\subsection{Bakteriális algoritmusok}

A \kix{bakteriális algoritmus} egy természetből ellesett optimalizációs
technika. A baktériumok evolúciós folyamatán alapul. Alkalmas bonyolult
optimalizációs problémák megoldására. Az algoritmus lépései:
\begin{itemize}
  \item Kezdeti populáció véletlenszerű létrehozása.
  \item Bakteriális mutáció végrehajtása minden egyeden.
  \item Génátadás végrehajtása a populációban.
  \item Ha megfelelő eredményt értünk el, akkor megállunk, különben folytatjuk
        a bakteriális mutációs lépéssel.
\end{itemize}

A \kix{bakteriális mutáció} lépései:
\begin{itemize}
  \item Egy rész véletlenszerű kiválasztása.
  \item Az $i$-edik részt változtatjuk az $N_\text{klón}$ számú másolatban,
        de az eredeti baktériumban nem.
  \item A legjobb baktérium átadja az $i$-edik részt a többi baktériumnak.
  \item Ismételjük addig, amíg az összes részt ki nem választottuk.
\end{itemize}

A \kix{génátadás} módja:
\begin{itemize}
  \item A populációt 2 részre osztjuk, jó egyedekre, és rossz egyedekre.
  \item Egy baktériumot véletlenszerűen kiválasztunk a jobbik alpopulációból
        (forrásbaktérium) egy másikat pedig a rossz egyedek közül (célbaktérium).
  \item A forrásbaktérium egy része felülírja a célbaktérium egy részét.
  \item Ez az algoritmus ismétlődik $N_\text{inf}$-szer
\end{itemize}

Az algoritmus paraméterei:
\begin{itemize}
  \item $N_\text{gen}$: generációk száma.
  \item $N_\text{ind}$: egyedek száma.
  \item $N_\text{klón}$: másolatok (klónok) száma a bakteriális mutációban.
  \item $N_\text{inf}$: génátadások (infekciók) száma a génátadásnál.
\end{itemize}

\subsection{Genetikus programozás}

A \kix{genetikus programozás} John Koza nevéhez fűződik.
A genetikus programozás a genetikus algoritmusok alapötletét alkalmazza a
lehetséges programok terére. Különbözőnek tűnő problémák különböző területekről
átfogalmazhatók egy számítógépes program-keresési feladattá.
A genetikus operátorok:
\begin{itemize}
  \item Keresztetés
  \item Mutáció
  \item Reprodukció
        \begin{itemize}
          \item Szülő kiválasztása fitness érték alapján, majd válzottatás
                nélküli lemásolása.
        \end{itemize}
\end{itemize}

\paragraph{Példa} Véletlen programok létrehozása
\begin{itemize}
  \item Rendelkezésre álló függvények:
        $F = \left\{ +, -, *, \%, \texttt{IF} \right\}$
  \item Rendelkezésre álló terminánsok:
        $T = \left\{ X, Y, \text{consts} \right\}$
  \item A véletlen programok szintaktikailag érvényesek és végrehajthatóak.
  \item A fák különböző méretűek és alakúak lehetnek.
\end{itemize}

\paragraph{Példa} Szimbolikus regresszió
\begin{itemize}
  \item Cél meghatározása.
        \begin{itemize}
          \item Találjunk egy egybemenetű ($X$ független változó) számítógépes
                programot, amelynek a kimenete megegyezik a kívánt kimenettel.
        \end{itemize}
  \item A terminálisok halmazának meghatározása.
        \begin{itemize}
          \item $T = \left\{ X, \mathrm{consts} \right\}$
        \end{itemize}
  \item A függvények halmazának meghatározása.
        \begin{itemize}
          \item $F = \left\{ +, -, *, \% \right\}$
        \end{itemize}
  \item A fitness értékek meghatározása.
        \begin{itemize}
          \item A program kimenetei és a kívánt kimenetek közötti különbségek
                abszolútértékeinek összege.
        \end{itemize}
  \item A futtatás paramétereinek meghatározása.
        \begin{itemize}
          \item A poluláció mérete -- $M = 4$.
        \end{itemize}
  \item A megállási feltétel meghatározása.
        \begin{itemize}
          \item Ha kialakul egy olyan egyed, amelynél az abszolút hibák összege
                kisebb, mint $0,1$.
        \end{itemize}
\end{itemize}

\end{document}
