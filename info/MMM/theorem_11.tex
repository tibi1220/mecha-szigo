\documentclass[../../main.tex]{subfiles}

\begin{document}

\section{Numerikus módszerek VII}

\begin{fulltheorem}
  Kezdetiérték feladatok közelítő megoldási lehetőségei
  (lineáris-nem lineáris esetekben, állapottér modell).
\end{fulltheorem}

\subsection{Tartály feladat}
\[
  \dot h = -K \sqrt{h}
  \qquad
  K = \frac{d^2 \pi \sqrt{2g}}{4F}
\]

\bgroup
\def\arraystretch{2}
\arrayrulecolor{cyan!20!black}
\setlength{\arrayrulewidth}{1pt}
\begin{table}[H]
  \centering\begin{tabular}{|*{4}{c|}}
    \hline
    \rowcolor{yellow!70!gray}
    $i$ & $t_0$                  & $h_i$                                                   & $\dot h_i$
    \\ \hline
    $0$ & $t_0$                  & $h_0$                                                   & $\dot h_0 = -K \sqrt{h_0}$
    \\  \rowcolor{gray!10}
    $1$ & $t_1 = t_0 + \Delta t$ & $h_1 = h_0 + \dot h_0 \Delta t$                         & $\dot h_1 = -K \sqrt{h_1}$
    \\
    $2$ & $t_2 = t_1 + \Delta t$ & $h_2 = h_1 + \dfrac{3 \dot h_1 - \dot h_0}{2} \Delta t$ & $\dot h_2 = -K \sqrt{h_1}$
    \\  \rowcolor{gray!10}
    $3$ & $t_3 = t_2 + \Delta t$ & $h_3 = h_2 + \dfrac{3 \dot h_2 - \dot h_1}{2} \Delta t$ & $\dot h_3 = -K \sqrt{h_1}$
    \\ \hline
  \end{tabular}
  \caption{Tartályos feladat megoldása közelítéssel}
  \label{fig:int1}
\end{table}
\egroup

\subsection{Rezgő feladat}
\[
  m \ddot x + b \dot x + k x = F
  \qquad
  \ddot x = \frac{F}{m} - \frac{b}{m} \dot x - \frac{k}{m} x
\]

\bgroup
\def\arraystretch{2}
\arrayrulecolor{cyan!20!black}
\setlength{\arrayrulewidth}{1pt}
\begin{table}[H]
  \small
  \centering\begin{tabular}{|*{5}{c|}}
    \hline
    \rowcolor{yellow!70!black}
    $i$ & $t_0$                  & $x_i$                                                 & $\dot x_i$                                                         & $\ddot x_i$
    \\ \hline
    $0$ & $t_0$                  & $x_0$                                                 & $\dot x_0$                                                         & $\ddot x_0 = \dfrac{F}{m} - \dfrac{b}{m} \dot x_0 - \dfrac{k}{m} x_0$
    \\  \rowcolor{gray!10}
    $1$ & $t_1 = t_0 + \Delta t$ & $x_1 = x_0 + \dfrac{\dot x_1 - \dot x_0}{2} \Delta t$ & $\dot x_1 = \dot x_0 + \ddot x_0 \Delta t$                         & $\ddot x_1 = \dfrac{F}{m} - \dfrac{b}{m} \dot x_1 - \dfrac{k}{m} x_1$
    \\
    $2$ & $t_2 = t_1 + \Delta t$ & $x_2 = x_1 + \dfrac{\dot x_2 - \dot x_1}{2} \Delta t$ & $\dot x_2 = \dot x_1 + \dfrac{3\ddot x_1 - \ddot x_0}{2} \Delta t$ & $\ddot x_2 = \dfrac{F}{m} - \dfrac{b}{m} \dot x_2 - \dfrac{k}{m} x_2$
    \\  \rowcolor{gray!10}
    $3$ & $t_3 = t_2 + \Delta t$ & $x_3 = x_2 + \dfrac{\dot x_3 - \dot x_2}{2} \Delta t$ & $\dot h_3 = \dot x_2 + \dfrac{3\ddot x_2 - \ddot x_1}{2} \Delta t$ & $\ddot x_3 = \dfrac{F}{m} - \dfrac{b}{m} \dot x_3 - \dfrac{k}{m} x_3$
    \\ \hline
  \end{tabular}
  \caption{Rezgő feladat megoldása közelítéssel}
  \label{fig:int2}
\end{table}
\egroup

\subsection{Prediktor-Corrector alkalmazása}

A legtöbbet az első lépésnél segít, csökkenti az önindítás hiánya miatti hibát.
Az iterációt általában egy lépésnél tovább nem célszerű alkalmazni, mert nem
javít jelentősen, de növeli a számítási időt. A tartályos feladat kivétel,
hiszen ott a parabolát egyenessel közelítjük.
\begin{itemize}
  \item Adams Multon -- nem léptet, hanem pontosít -- corrector
  \item Adams Bashforth -- téglány első lépésben -- prediktor
\end{itemize}

\subsection{Állapottér modell}

Az \kix{állapottér modell} általános alakja:
\begin{gather*}
  \dot{\rvec q} = \rmat A \rvec q + \rmat B \rvec u \\
  \rvec x = \rmat C \rvec q + \rmat D \rvec u
\end{gather*}

A rezgő feladat felírása állapottér alakban:
\begin{gather*}
  q_1 = x, \quad q_2 = \dot x, \quad \dot q_1 = q_2 \\
  \begin{bmatrix}
    \dot q_1 \\ \dot q_2
  \end{bmatrix} = \begin{bmatrix}
    0 & 1 \\ -b/m & -k/m
  \end{bmatrix} \begin{bmatrix}
    q_1 \\ q_2
  \end{bmatrix} + \begin{bmatrix}
    0 \\ 1/m
  \end{bmatrix} F \\
  x = \begin{bmatrix}
    1 \\ 0
  \end{bmatrix}^\intercal \begin{bmatrix}
    q_1 \\ q_2
  \end{bmatrix} + \begin{bmatrix}
    0 \\ 0
  \end{bmatrix} F
\end{gather*}

A modell megoldása:
\begin{gather*}
  \dot{\rvec{q}}_i \approx \frac{\rvec q_{i+1} - \rvec q_i}{\Delta t}
  = \rmat A \rvec q_i + \rmat B \rvec qu
  \\
  \rvec q_{i+1} = \rvec q_i + \left( \rmat A \rvec q_i + \rmat B \rvec u \right) \Delta t
  \\
  \rvec q_{i+1}
  = \underbrace{\left(\rmat I + \Delta t \rmat A\right)}_{\rmat M} \rvec q_i
  + \underbrace{\Delta t \rmat B \rvec u}_{\rmat N}
\end{gather*}

A megoldás ezen mátrixok segítségével felírható:
\begin{align*}
  \rvec q_{i+1} & = \rmat M \rvec q_i + \rmat N         \\
  \rvec x_{i+1} & = \rmat C \rvec q_i + \rmat D \rvec u
\end{align*}


\end{document}
