\documentclass[../../main.tex]{subfiles}

\begin{document}

\section{Numerikus módszerek V}

\begin{fulltheorem}
  Numerikus integrálás.
\end{fulltheorem}

Nincsen valós differenciáló tag, viszont integráló létezik.
A differenciálegyenletek megoldása során is integrálunk.
A differenciálás erőltetése kiegyszerűsödéssel áll bosszút.
($\Delta t$ nagyságrendileg eltér a számítandó értékektől)

Az határozott integrál matematikai definíciója:
\[
  \int_a^b f(t) \, \dd t = F(b) - F(a)
\]

A primitív függvény azonban nem mindig áll a rendelkezésünkre, vagy mert a
matematikai ismereteink nem elégségesek a meghatározásához, vagy mert nem lehet
előállítani. Ekkor numerikus módszereket alkalmazunk.

\subsection{Numerikus integrálás során fellépő hibafajták}

\paragraph{Kerekítési hiba}

A számítógépes számábrázolásból adódó hiba.

\paragraph{Csonkítási hiba}

A függvény közelítéséből adódóan.

\paragraph{Kumulatív hiba}

Halmozott hiba, a kerekítési és csonkítási hiba eredője.
Amennyiben ez a hibatípus nem korlátos, akkor az integrálási folyamat
nem lesz stabil.

\paragraph{A hiba felírása}
\begin{enumerate}
  \item Hiba felírása.
  \item A függvényt Taylor-sorral közelítjük.
  \item $H$-ra rendezve polinomsort kapunk
  \item Az első $n$ elemet 0-nak vesszük, lineáris egyenletrendszert kapunk.
  \item Megoldjuk az egyenletrendszert.
  \item $H$ maradékának felhasználásával hibabecslést készítünk.
\end{enumerate}
\begin{gather*}
  H
  = \int_{t_0}^{t_m} y(t) \, \dd t
  - \sum_{i=0}^n c_i y(t_i)
  \\
  T(t)
  = \sum_{i=0}^\infty \frac{(t - t_k)^i}{i!}y^{(i)}(t_k)
\end{gather*}

\subsection{Zárt Newton-Cotes formulák}

A \kix{Newton-Cotes} formulák az intervallum belső pontjait használják
fel a közelítéshez.

\paragraph{Trapéz}
\[
  y_{n + 1}
  = y_n
  + \frac{\dot y_n + \dot y_{n + 1}}{2} \Delta t
  - \frac{\dddot y(\xi)}{12} \Delta t^3
\]

\paragraph{Simson}
\[
  y_{n + 1}
  = y_n
  + \frac{\dot y_n +  4 \dot y_{n + 1/2} + \dot y_{n + 1}}{6} \Delta t
  - \frac{y^{(\mathrm{V})}(\xi)}{90} \Delta t^5
\]

\subsection{Adams-Bashforth formulák}

Az \kix{Adams-Bashforth} formulák az intervallum előtti pontokat használják
fel a közelítéshez.

\paragraph{Téglány (AB1)}

A \kix{téglány} a legegyszerűbb integrálközelítő módszer. Mivel egylépéses ezért
önindító. Az első lépésben szokás használni, utána az eredménnyel magasabbrendű
közelítő módszereket alkalmazunk.
\[
  y_{n + 1}
  = y_n + \dot y_n \Delta t
  + \frac{\ddot y(\xi)}{2} \Delta t^2
\]
\begin{figure}[H]
  \centering
  \includegraphics{../../static/tex/build/AB1.pdf}
  \caption{Integrálközelítés téglánnyal}
  \label{fig:AB1}
\end{figure}

\paragraph{Trapéz (AB2)}

Az AB2 \kix{trapéz formula} a téglánnyal hasonlóan csak az előző értékekre épít,
ezért \kix{prediktor}[nak] is szokás nevezni, hiszen az ilyen formulák jósolnak.
\[
  y_{n+1}
  = \frac{3 \dot y_n - \dot y_{n-1}}{2} \Delta t
  + \frac{5 \dddot y(\xi)}{12} \Delta t^3
\]
Megfigyelhetjük, hogy a formula tulajdonképpen egy \kix{súlyozott átlag}.

\begin{figure}[H]
  \centering
  \includegraphics{../../static/tex/build/AB2.pdf}
  \caption{Integrálközelítés trapézzal (AB2)}
  \label{fig:AB2}
\end{figure}

\subsection{Adams-Moulton formulák}

Az \kix{Adams-Moulton} formulák az intervallum vége és az az előtti pontokat
használják fel a közelítéshez.

\paragraph{Téglány}
\[
  y_{n + 1}
  = y_n
  + \dot y_{n + 1} \Delta t
  - \frac{\ddot y(\xi)}{2} \Delta t^2
\]

\paragraph{Trapéz (AM2)}
\[
  y_{n + 1}
  = y_n
  + \frac{\dot y_n + \dot y_{n + 1}}{2} \Delta t
  - \frac{\dddot y(\xi)}{12} \Delta t^3
\]

\begin{figure}[H]
  \centering
  \includegraphics{../../static/tex/build/AM2.pdf}
  \caption{Integrálközelítés trapézzal (AM2)}
  \label{fig:AM2}
\end{figure}


\subsection{Runge-Kutta formulák}

A \kix{Runge-Kutta} formulák csak a $\Delta t$ nagyságú intervallumokat
használnak, viszont ezek belső pontjaira is szükségük van. A függvény
közelítésére Taylor-polinomokat használnak. Mivel csak egy intervallumot
használnak, ezért egylépéses módszernek is nevezik őket.

\end{document}
