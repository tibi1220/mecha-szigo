\documentclass[../../main.tex]{subfiles}

\begin{document}

\section{Grafika III}

\begin{fulltheorem}
  Láthatóság, árnyalás, megvilágítás, színmodellek, anyagmodellek, konvolúció, élkeresés
\end{fulltheorem}

\subsection{Megjelenítési algoritmusok -- láthatóság}

\subsubsection{Deapth Sort algoritmus}

A mélységi rendezés (\kix{Depth Sort}) algoritmus lényege, hogy rendezzük a
képernyőtől (leképezési sík) való távolságuk szerinti sorrendbe a kirajzolandó
objektumok felületét alkotó háromszögeket, majd hátulról előre rajzoljuk
ki őket. Így amit később rajzolunk ki az átfesti (takarja) a hátrébb lévő elemeket.

A háromszögekkel közelített felület elemeit a megjelenítési síktól mért
legtávolabbi pontjuk távolsága szerint rendezzük sorba. A sorrend megfelelő, ha:
\begin{enumerate}
  \item Ha $\triangle_A$ legtávolabbi pontja közelebb van a síkhoz, mint
        $\triangle_B$ legközelebbi pontja.
  \item A háromszögek befoglaló téglalapjai nem metszik egymást.
  \item Az $\triangle_A$ vetített képe nem metszi $\triangle_B$ vetített képét.
  \item a $\triangle_B$ mindhárom sarokpontja az $\triangle_A$ síkja mögött
        helyezkedik el.
\end{enumerate}
Ha ezek egyike sem áll fenn, akkor az egyik háromszöget két részre kell vágni,
és folytatni a módszert egészen addig, míg az összes háromszög sorbarendezhető.

\subsubsection{Area subdivision}

Az területfelosztásos (\kix{Area Subdivision}) technika lényege az, hogy a
\kix{Viewport}[ot] négy egyenlő részre bontjuk. A levetített háromszögek
vizsgálata során megállapíthatjuk, hogy készen vagyunk a takarással, ha:
\begin{enumerate}
  \item a részt egyetlen háromszög foglalja be,
  \item egyetlen háromszög sincs a negyedben,
  \item a negyed egyetlen háromszöget metsz, vagy foglal magában,
  \item valamelyik háromszögről tudjuk, hogy takar.
\end{enumerate}

\subsubsection{Z-puffer}

Manapság a takart részek eltávolításának legelterjedtebb módszere a
\kix{Z-puffer} algoritmus. A grafikus kártyákba be van építve.
Az algoritmus lényege, hogy minden képernyőpixelhez a Z-puffer tartalmaz
egy mélységi adatot. Ha az aktuális pont távolsága kisebb mint a Z-pufferben
tárolt érték, akkor a pont kifestődik, és a Z-pufferbe a z-koordináta lesz az
új, aktuális érték.

\subsection{Árnyalás}

A különböző anyagok tükrözik, szétszórva visszaverik, megtörik vagy elnyelik a
fénysugarakat. Az anyagok ilyen viselkedését modellezhetjük a kétirányú
visszaverődés-eloszlás függvénnyel. Ennek angol rövidítése a \kix{BRDF}.

\subsection{Megvilágítás}

\subsubsection{Fényforrások}

\begin{itemize}
  \item Pontszerű fényforrás
  \item Irányított fényforrás (iránya az intenzitással azonos)
  \item Ambiens fényforrás (állandó intenzitású)
  \item Szpotlámpa (határkúp, iránytól függő intenzitás)
  \item Égboltfény (irány- és tárgyfüggő is lehet)
\end{itemize}

\subsection{Színmodellek}

\begin{itemize}
  \item \kix{RGB} -- Red, Green, Blue
  \item \kix{CMY} -- Cyan, Magenta, Yellow
  \item \kix{HSL} -- Hue, Saturation, Lightness
\end{itemize}

\subsection{Anyagmodellek}

\paragraph*{Diffúz anyagok}
A beérkező, adott hullámhosszú fénysugár esetén minden irányban egyforma
fényáram távozik. Az, hogy mi mennyire látjuk a tárgyat világosnak, az csak
attól függ, hogy honnan világítjuk meg. Ilyen anyag pl. a fal.
A Lambert-törvény írja le a diffúz felületek megvilágítás hatására történő
viselkedését. A diffúz felületek BRDF kifejezése adott hullámhosszon csak a
fény hullámhosszától függ.
\begin{figure}[H]
  \centering
  \includegraphics[]{../../static/tex/build/model-diff.pdf}
  \caption{A diffúz anyagmodell}
  \label{fig:model-diff}
\end{figure}

\paragraph*{Reflexív anyagok}
Ha egy tükröt egy fénysugárral megvilágítunk, a tükör azt a fénysugarat úgy
veri vissza, hogy a visszavert sugár benne fekszik a beeső fény és a beesési
merőleges által meghatározott síkban és ugyanakkora szöget zár be a beesési
merőlegessel, mint a bejövő fénysugár.
\begin{figure}[H]
  \centering
  \includegraphics[]{../../static/tex/build/model-reflex.pdf}
  \caption{A reflexív anyagmodell}
  \label{fig:model-reflex}
\end{figure}

\paragraph*{Ideális fénytörés}
A fénytörésről szóló Snellius-Descartes-törvény szerint a beeső fénysugár,
a beesési merőleges és a megtört fénysugár egy síkban helyzkednek el.
A merőlegesen beeső fénysugár nem törik meg.
Végül a beesési szög szinuszának és a törési szög szinuszának aránya a
közegekben mért terjedési sebességek arányával egyenlő, ami megegyezik a két
közeg relatív törésmutatójával.
\begin{figure}[H]
  \centering
  \includegraphics[]{../../static/tex/build/model-break.pdf}
  \caption{Az ideális törés modellje}
  \label{fig:model-break}
\end{figure}

\paragraph*{Spekuláris visszaverődés}
A fényvisszaverő anyagok általában sem tisztán diffúz módon, sem tisztán
reflexív módon verik vissza a fényt, hanem a kettő között valahogy. Az ilyen
felületeket spekuláris felületeknek nevezzük, és a visszaverődés leírása
Phong nevéhez köthető. Phong modellje szerint a visszaverődés nagy része az
elméleti visszaverődés irányában és annak környezetében történik.
\begin{figure}[H]
  \centering
  \includegraphics[]{../../static/tex/build/model-phong.pdf}
  \caption{A Phong modell}
  \label{fig:model-phong}
\end{figure}


\subsection{Konvolúció}

A képpont és környezete világosságának súlyozott összege.
\[
  q(k; l)
  = t \otimes g(k; l)
  = \sum_{r=-R}^R \sum_{s=-S}^S t(r; s) \cdot g(k+r; l+s)
\]

\begin{center}
  \begin{tabular}{c c l}
    $g$ & -- & a képpont eredeti világossága
    \\
    $q$ & -- & a képpont új világossága
    \\
    $t$ & -- & a súlyok, rendszerint páratlanszámú sort és oszlopot tartalmaz
  \end{tabular}
\end{center}

Tulajdonságai: szimmetrikus, asszociatív.

Az \kix{elmosás} konvolúciós mátrixok segítségével valósítható meg.
($\rmat K$ -- blur, $\rmat G$ -- Gauss blur)
\[
  \rmat K = \frac{1}{\mathrm{width} \cdot \mathrm{height}}
  \begin{bmatrix}
    1      & 1      & 1      & \cdots & 1      \\
    1      & 1      & 1      & \cdots & 1      \\
    1      & 1      & 1      & \cdots & 1      \\
    \vdots & \vdots & \vdots & \ddots & \vdots \\
    1      & 1      & 1      & \cdots & 1
  \end{bmatrix}
  \qquad
  \rmat G = \frac{1}{159}
  \begin{bmatrix}
    2 & 4  & 5  & 4  & 2 \\
    4 & 9  & 12 & 9  & 4 \\
    5 & 12 & 15 & 12 & 9 \\
    4 & 9  & 12 & 9  & 4 \\
    2 & 4  & 5  & 4  & 2
  \end{bmatrix}
\]

\subsection{Élkeresés}

\paragraph{Élkeresés differencia kereséssel -- differential}

Vízszintes és függőleges irányú változásokat a szürkeségi differencia képzésével
emelhetjük ki.

\paragraph{Másodrendű differencia -- gradient}

Vízszintes és függőleges irányú változásokat (másodrendű) parciális differencia
képzésével emelhetjük ki.

\paragraph{Élkeresés a változás sebessége alapján -- Sobel}

A változás sebessége a vízszintes és a függőleges irányban áltagoló
differenciál operátorral.

\paragraph{Élkeresés a változás gyorsulása alapján -- Laplace}

A Laplace operátort használja élkeresésre. ($\Div \Grad f$)

\paragraph{Élkeresés a Canny módszerével}

Gauss zajszűrés, szürkeárnyalatos kép, vízszintes és függőleges gradiens
komponensek, közeli nem maximumok eltűntetése, gradiens küszöb.

\end{document}
