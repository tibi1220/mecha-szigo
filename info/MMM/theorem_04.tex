\documentclass[../../main.tex]{subfiles}

\begin{document}

\section{Grafika IV}

\begin{fulltheorem}
  Képfeldolgozás, szegmentálás, alakfelismerés.
\end{fulltheorem}

\subsection{Szegmentálás}

\kix{Szegmentálás} során célunk az összetartozó jellemzők összegyűjtése,
csoportosítása.

Szegmentálhatunk felülről lefele, (Top Down -- globális szegmentálás
küszöbértékekkel -- közös ismert objektum hasonló pixeleinek keresése)
vagy lentről felfelé is (Bottom Up -- elárasztás
statisztikával, pl. tartományok növelése, kijelölése)

\subsection{Alakfelismerés}

\kix{Alakfelismerés} a korrelációs együttható (kovariancia és varancia hányadosa)
segítségével lehet.


\end{document}
