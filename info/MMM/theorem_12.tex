\documentclass[../../main.tex]{subfiles}

\begin{document}

\newpage

\section{Programozás I}

\begin{fulltheorem}
  Fuzzy halmazok alapjai, műveletek fuzzy halmazokon.
\end{fulltheorem}

\subsection{Homokkupac paradoxon}
\[
  \mathrm{homokkupac - homokszem = homokkupac}
\]
Ebből az következik, hogy $\mathrm{homokkupac} = 0$.
Ennek oka, hogy a homokkupacot nem definiáltuk elég pontosan.
A gond az, hogy a matematikai precizitás nincsen összhangban a homokkupac
hétköznapi fogalmával, vagyis a precíz matematika nem alkalmas pontatlan
fogalmak formális kezelésére. A megoldás erre a \kix{fuzzy logika}.

\subsection{A fuzzy halmazelmélet}

A \kix{fuzzy halmazelmélet} Mamdahi és Zadeh nevéhez fűződik, a szó jelentése:
homályos, életlen. A halmazok között \kix{elmosódott határ}[ok] vannak, pl.:
magas emberek. A \kix{részleges tagság}[] 0 és 1 közötti értéket vehet fel,
melyet a \kix{tagsági függvény}[] jellemez.

\paragraph{Példa} Egy hallgatói csoport
\begin{itemize}
  \item Alaphalmaz: $X$ -- Összes hallgató
  \item Részhalmaz: $A$ -- Hallgatók jogosítvánnyal
  \item Karakterisztikus függvény: $\chi_A(X)$
  \item Taggsági függvény: $\mu(X)$ -- Hallgatók, akik jól vezetnek
\end{itemize}

\paragraph{Tulajdonságok}

A fuzzy halmazok az alábbi paraméterekkel jellemezhetőek:
\begin{center}
  \begin{tabular}{l l}
    mag                    & $\core A = \left\{ x \in X | \mu_A(x) = 1 \right\}$          \\
    tartó                  & $\supp A = \left\{ x \in X | \mu_A(x) > 0 \right\}$          \\
    $\alpha$-vágat         & $A_\alpha = \left\{ x \in X | \mu_A(x) \geq \alpha \right\}$ \\
    szigorú $\alpha$-vágat & $A_{\alpha+} = \left\{ x \in X | \mu_A(x) > \alpha \right\}$ \\
    magasság               & $h(A) = \sup_{x \in X} \mu_A(x)$                             \\
  \end{tabular}
\end{center}

\paragraph{Műveletek}
A halmazokhoz hasonlóan a futtz halmazokon is végezhetünk matematikai
műveleteket.
\begin{center}
  \begin{tabular}{l l}
    Unió      & $\mu_{A \cup B}(x) = \max \left\{ \mu_A(x); \mu_B(x) \right\}$ \\
    Metszet   & $\mu_{A \cap B}(x) = \min \left\{ \mu_A(x); \mu_B(x) \right\}$ \\
    Különbség & $\mu_{\overline A}(x) = 1 - \mu_A(x)$                          \\
  \end{tabular}
\end{center}

\end{document}
