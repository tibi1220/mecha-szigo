\documentclass[../../main.tex]{subfiles}

\begin{document}

\section{Programozás V}

\begin{fulltheorem}
  Felügyelet nélküli tanulás, K-means klaszterezés.
\end{fulltheorem}

\subsection{Klaszterezés}
% http://www.cs.bme.hu/~csima/dm14/cluster.pdf

A \kix{klaszterezés} célja valamilyen dolgokat úgy csoportokba osztani, hogy a
hasonlók kerüljenek egy csoportba. A \kix{felügyelet nélküli tanulás} során
is használhatunk klaszterezést. Ilyen tanulás során nincsen címkézés, amely
segítene a besorolásban, az attribútumértékek egymáshoz való viszonya alapján
kell csoportosítani.

\subsection{K-means klaszterezés}

A \kix{K-means klaszterezés} egy particionális (minden dolog egy klaszterbe
kerülhet), prototípus-alapú (minden klaszternek van egy reprezentánsa, minden
dolog abba a klaszterbe kerül, aminek a reprezentánsához a legközelebb van)

\paragraph{K-means algoritmus}

Adott egy $K$ szám, ennyi csoportot szeretnénk létrehozni.
\begin{enumerate}
  \item $K$ darab kezdő centroid (átlag, középpont) kiválasztása az $n$
        dimenziós térből (nem feltétlenül adatpont).
  \item Minden adatpont hozzácsatolása a hozzá legközelebb álló centroidhoz.
  \item Kapott csoportokra újraszámoljuk a centroidokat.
  \item (2)-es és (3)-as pont ismétlése, amíg már nincsen további változás
\end{enumerate}



\end{document}
